\documentclass{article}

\usepackage[polish]{babel}
\usepackage[utf8]{inputenc}
\usepackage{polski}
\usepackage[T1]{fontenc}
 
\usepackage[margin=1.5in]{geometry} 

\usepackage{color} 
\usepackage{amsmath}                                                                    
\usepackage{amsfonts}                                                                   
\usepackage{graphicx}                                                             
\usepackage{booktabs}
\usepackage{amsthm}
\usepackage{pdfpages}
\usepackage{wrapfig}

\theoremstyle{definition}
\newtheorem{de}{Definicja}[subsection]

\theoremstyle{definition}
\newtheorem{tw}{Twierdzenie}[subsection]

\theoremstyle{definition}
\newtheorem{pk}{Przykład}[subsection]

\theoremstyle{definition}
\newtheorem*{fakt}{FAKT}

\author{Rafal Wlodarczyk}
\title{Algebra - Notatki z wykladu}  
\date{INA 1 Sem.}

\begin{document}

\maketitle

\section{Wyklad Pierwszy}

\subsection{Symbole}

\paragraph{Logika}

$\neg, \land, \lor, \implies, \iff$

\paragraph{Zbiory}

$x\in A, A \cap B, A \cup B, A - B, A\backslash B, A^C, B^C, A\subseteq B, A\times B$

\paragraph{Funkcje}

$f: X\rightarrow Y, f: X\times Y\rightarrow A$ funkcja dwuargumentowa

\paragraph{Własność}

$\\\text{Dla } \mathbb{N}$
$W(n) \forall_{x} (x|n)\implies x=1 \lor x=n$
\\Jest to definicja liczb pierwszych.

\subsection{Definicje}

\begin{de}
Niech $X$ - Zbiór. Działaniem na $X$ nazywamy każdą funkcję $f:X\cdot X\rightarrow X$
\end{de}

\begin{pk}
\begin{itemize}
Rozważmy następujące przykłady:

\item $f(x,y)=x\cdot y$ Jest działaniem na $\mathbb{R}$ - tak
\item $f(x,y)=x-y$ Jest działaniem na $\mathbb{N}$? - nie, ponieważ $\exists_{x,y} f(x,y)\notin \mathbb{N}$
\end{itemize}
\end{pk}

\paragraph{Oznaczenie}

$f(x,y)\iff x+y, x\cdot y, x\circ y$ - Działanie ogólne

\begin{de}
Niech $X$ - Zbiór. Działanie $\circ$ nazywamy łącznym, gdy:
$\forall_{x,y,z\in X} (x\circ y)\circ z = x\circ (y\circ z)$
Działanie $\circ$ nazywamy przemiennym, gdy:
$\forall_{x,y\in X} x\circ y = y\circ x$
\end{de}

\begin{pk}
\begin{itemize}
.
\item $+$ na $\mathbb{R}$ jest łączne i przemienne
\item $-$ na $\mathbb{R}$ nie jest ani łączne, ani nieprzemienne
\end{itemize}
\end{pk}

\begin{de}
Niech $\circ$ - działanie na zbiorze $X$. Element $e\in X$ nazywamy elementem neutralnym (dla $\circ$), gdy:
$\forall_{x\in X} e\circ x = x\circ e = x$
\end{de}

\begin{pk}
\begin{itemize}
.
\item $0$ jest elementem neutralnym dla $+$ na $\mathbb{N}$
\item $1$ jest elementem neutralnym dla $\cdot$ na $\mathbb{R}$
\end{itemize}
\end{pk}

\begin{fakt}
Niech $\circ$ - działanie na zbiorze $X$. Jeżeli $\circ$ ma element neutralny, to jest on jedyny.
D-d. Niech $a,b$ oznaczają elementy neutralne. Działanie $\circ$ na $X$:
\begin{itemize}
\item $a\circ b = b$
\item $a\circ b = a$
\end{itemize}
Zatem: $a=b\qed$
\end{fakt}

\begin{de}
Niech $\circ$ - działanie na zbiorze $X$. Element $a\in X$ nazywamy elementem odwrotnym (dla $\circ$), gdy:
$\forall_{x\in X} a\circ x = x\circ a = e$
\end{de}

\begin{pk}
\begin{itemize}
.
\item $-x$ jest elementem odwrotnym dla $+$ na $\mathbb{R}$
\item $\frac{1}{x}$ jest elementem odwrotnym dla $\cdot$ na $\mathbb{R}$
\item $x^2$ nie ma elementu odwrotnego dla $\cdot$ na $\mathbb{R}$
\item $x^2$ ma element odwrotny dla $\cdot$ na $\mathbb{R}^+$
\end{itemize}
\end{pk}

\end{document}