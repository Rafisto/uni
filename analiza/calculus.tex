\documentclass{article}

\usepackage[polish]{babel}
\usepackage[utf8]{inputenc}
\usepackage{polski}
\usepackage[T1]{fontenc}
 
\usepackage[margin=1.5in]{geometry} 

\usepackage{color} 
\usepackage{amsmath}                                                                    
\usepackage{amsfonts}                                                                   
\usepackage{graphicx}                                                             
\usepackage{booktabs}
\usepackage{amsthm}
\usepackage{pdfpages}
\usepackage{wrapfig}
\usepackage{hyperref}

\theoremstyle{definition}
\newtheorem{de}{Definicja}[subsection]

\theoremstyle{definition}
\newtheorem{tw}{Twierdzenie}[subsection]

\theoremstyle{definition}
\newtheorem{pk}{Przykład}[subsection]

\theoremstyle{definition}
\newtheorem*{fakt}{FAKT}

\author{Rafal Wlodarczyk}
\title{Analiza Matematyczna}  
\date{INA 1 Sem. 2023}

\begin{document}

\maketitle

\section{Wykład pierwszy}
tbd

\section{Wykład drugi}
tbd

\section{Wykład trzeci}

\begin{tw}
Twierdzenie (o ciągu monotonicznym i ograniczonym)\\
a) Ciąg rosnący i ograniczony z góry jest zbieżny.\\
$\forall_{n>n_0} a_n\leq a_{n+1}$ i $\forall_{n\in \mathbb{N} a_n< M}$ $\implies \exists \lim_{n\rightarrow \infty} a_n$\\
b) Ciąg malejący i ograniczony z dołu jest zbieżny.\\
$\forall_{n>n_0} a_n\geq a_{n+1}$ i $\forall_{n\in \mathbb{N} a_n> m}$ $\implies \exists \lim_{n\rightarrow \infty} a_n$\\
Idea dowodu:\\
$A=\{a_{n_0+1},a_{n_0+2},\dots,a_n,\dots\} \in \mathbb{R}$\\
A - ograniczony, istnieje kres górny zbioru A\\
Każdy ograniczony podzbiór liczb rzeczywistych ma kres\\
czyli $sup(A)$ (??) $sup(A)=lim_{n\rightarrow \infty a_n}$
\end{tw}

\begin{pk}
Rozważmy następujący ciąg rekurencyjny:
$a_1=\sqrt{2}$ $a_{n+1}=\sqrt{2+a_n}$\\
Idea dowodu indukcyjnego:\\
1. $a_n\leq 2$, indukcja po $n$\\
2. $a_n\leq a_{n+1}$, indukcja po $n$. $a_n\leq a_{n+1}\implies a_{n+1}\leq a_{n+2}$\\
3. $\sqrt{2+a_n}\leq \sqrt{2+a_{n+1}}$ kwadrat stronami rozwiązuje krok indukcyjny\\

$\forall_{n\geq 1} a_n \leq 2 \implies a_{n+1}\leq 2$\\
$a_{n+1}=\sqrt{2+a_n}\leq_{z. ind} \sqrt{2+2}=2$\\

Na mocy twierdzenia o ciągu monotonicznym i ograniczonym istnieje:\\
$lim_{n\rightarrow \infty} a_n = g$\\

\begin{center}
$a_{n+1}=\sqrt{2+a_n}$, $lim_{n\rightarrow \infty} a_n = g = lim_{n\rightarrow \infty} a_{n+1} = g$\\
$g=\sqrt{2+g}$\\
$g^2-g-2=0$\\
$\Delta=9=3^2$\\
$g_1=\frac{1+3}{2}=2$ lub $g_2=\frac{1-3}{2}=-1$, które nie zachodzi, zatem $lim a_n=g_1$
\end{center}
\end{pk}

\begin{de}
Podciąg ciągu\\
Niech $a_n$ będzie dowolnym ciągiem. Niech $n_1, n_2, ... n_k$ będzie pewnym rosnącym ciągiem liczb naturalnych.
Wówczas ciąg $a_{nk}=(a_{n1}, a_{n2}, a_{n3}, ...)$ Nazywamy podciągiem ciągu.
\end{de}

\begin{pk}
Rozważmy następujące przykłady ($\mathbb{N}=\{1,2,3,4,\dots\}$):\\
a) $a_n=(-1)^{n}$, $n\in\mathbb{N}$\\
$a_{2k}=(-1)^{n}=1$, $k\in\mathbb{N}$\\
$(a_2, a_4, a_6, ...)$ - podciąg o wyrazach parzystych.\\
b) $a_{2k-1}=(-1)^{2k-1}=-1, n\in\mathbb{N}$\\
$(a_1, a_3, a_5, ...)$ - podciąg o wyrazach nieparzystych.\\
$S=\{1,-1\}$\\
c) $(1, \frac{1}{2}, 3, \frac{1}{4}, 5, \frac{1}{6},\dots)$\\
$a_{2k-1}=2k-1$ - podciąg o wyr. nieparzystych.\\
$a_{2k}=\frac{1}{2k}$ - podciąg o wyr. parzystych.\\
$S=\{0, \infty\}$
d) $sin(\frac{n\pi}{3})$ - $plot(sin(\frac{n\pi}{3}),(n,1,17)) \leftarrow$ wolframalpha
\end{pk}

\begin{de}
Liczba $s$ jest punktem skupienia ciągu $a_n\iff s$ jest granicą właściwą lub niewłaściwą pewnego podciągu.
Oznaczenie $S$ - zbiór punktów skupienia.\\

Jeśli $lim_{n\rightarrow \infty} a_n = \infty \implies a_n$ ma granicę niewłaściwą $+\infty$
\end{de}

\begin{itemize}
\item $sup()$ - superior - kres górny
\item $inf()$ - inferior - kres dolny
\end{itemize}

\begin{de}
Granica górna ciągu $a_n$ to kres górny granic podciągu $a_n$.\\
$\lim_{n\rightarrow \infty} sup(a_n) = \lim_{n\rightarrow\infty} a_n$
\end{de}

\begin{de}
Granica dolna ciągu $a_n$ to kres dolny granic podciągu $a_n$.\\
$\lim_{n\rightarrow \infty} inf(a_n) = \lim_{n\rightarrow\infty} a_n$
\end{de}

$\lim inf(a_n)\leq \lim sup(a_n)$, równość dla granicy ciągu.

\begin{tw}
Twierdzenie (Bolzano - Weierstrassa). Każdy ciąg ograniczony ma podciąg zbieżny.
\href{https://en.wikipedia.org/wiki/Bolzano%E2%80%93Weierstrass_theorem}{(English Wikipedia)}\\
D-d. $\forall_{n\in\mathbb{N}} m \leq a_n \leq M$ Dzielimy przedział $[m_1,M_1]$ na dwa podprzedziały:
$[m_1, \frac{m_1+M_1}{2}]$, $[\frac{m_1+M_1}{2},M_1]$. Przynajmniej w jednym z przedziałów jest nieskończenie wiele wyrazów ciągu.
Oznaczmy tę połówkę przez $[m_2, M_2]$. Postępujemy tak dalej i mamy:\\
$\forall_{k\in\mathbb{N}} m_1\leq m_k\leq a_{nk} \leq M_k \leq M_1$\\
$M_k$ malejący i ograniczony $\implies$ zbieżny $g_1$\\
$m_k$ rosnący i ograniczony $\implies$ zbieżny $g_2$\\
$g_1=g_2=g$\\
$M_k-m_k=\frac{M_1-m_1}{2}$\\
$M_k\rightarrow g_1; m_k\rightarrow g_2$, ponieważ $\frac{M_1-m_1}{2^k}\rightarrow 0$

\end{tw}

\begin{de}
Ciąg $a_n$ nazywamy ciągiem Cauchy'ego, wtedy i tylko wtedy, gdy:\\
$\forall_{\varepsilon > 0}\exists_{n_0}\forall_{n,m>n_0} |a_n-a_m|<\varepsilon$.
\end{de}

\begin{tw}
Ciąg liczb rzeczywistych jest zbieżny $\iff$ jest ciągiem Cauchy'ego.
\end{tw}

\begin{pk}
$x_n = \frac{1}{0!} + \frac{1}{1!} + \frac{1}{2!} + \frac{1}{3!} + \frac{1}{4!} + ...$\\
$x_1 = 1, x_2 = 2, x_3 = 2 + 1/2$.
\begin{enumerate}
    \item $x_n$ jest rosnący $x_{n+1}-x_n=\frac{1}{(n+1)!}>0 \iff x_{n_1}>x_n$
    \item $x_n$ jest ograniczony (pamiętając, że $\forall_{n>3} 2^n\leq n!$
    czyli $\frac{1}{4!} < \frac{1}{2^4}$, $\frac{1}{5!} < \frac{1}{2^5}$)...\\
    Dla $n>3$ $x_n=\frac{1}{0!} + \frac{1}{1!} + \frac{1}{2!} + \frac{1}{3!} + \frac{1}{4!} + ...\leq$\\
    $2+\frac{1}{2}+\frac{1}{6}+\frac{1}{2^4}+\frac{1}{2^5}+...+\frac{1}{2^n}$\\
    $\frac{1}{2^4}\cdot\frac{1}{1-\frac{1}{2}}=\frac{1}{2^3}$\\
    Istnieje $lim_{n\rightarrow \infty} x_n = e = 2.7182...$\\
    $sum(1/k!, (k,0,300))\leftarrow$ wolframalpha
\end{enumerate}
\end{pk}

\begin{tw}
    Liczba eulera wyraża się wzorem:
    \begin{center}
    $\lim_{n\rightarrow \infty} (1+\frac{1}{n})^n = e$
    \end{center}
\end{tw}

\begin{tw}
    Niech $a_n$ będzie dowolnym ciągiem takim, że:
    $lim_{n\implies \infty} a_n = \infty$. Wówczas:\\
    \begin{center}
    $lim_{n\implies \infty} (1+\frac{1}{a_n})^{a_n} = e, (1-\frac{1}{a_n})^{a_n} = \frac{1}{e}$
    \end{center}
\end{tw}

\begin{pk}
    $\lim ((1+\frac{1}{2n})^{2n})^{\frac{1}{2}}=e^{\frac{1}{2}}=\sqrt{e}$
\end{pk}

Własność: $lim_{n\rightarrow \infty} a_n = g_1 \land lim_{n\rightarrow \infty} b_n = g_2 \implies lim_{n\rightarrow \infty} (a_n^{b_n})=g_1^{g_2}$

\begin{pk}
    $lim (1-\frac{1}{n})^{n/2}=lim \left((1-\frac{1}{n})^n\right)^{\frac{\frac{n}{2}}{n}}=(\frac{1}{e})^{\frac{1}{2}}=\frac{1}{\sqrt{e}}$
\end{pk}

Wskazówka: $limit\left((1+\frac{1}{2^n})^{n+1},n\rightarrow infty\right)$

\begin{de}
Szereg o wyrazach nieujemnych. Dla dowolnego ciągu $a_1, a_2, \dots, a_n$ o wyrazach nieujemnych,
tworzymy ciąg sum częściowych:
\begin{center}
$S_1=a_1, S_2=a_1+a_2, S_3=a_1+a_2+a_3, \dots, S_N=a_1+a+2+...+a_N$
\end{center}

Przykładowo dla $e$ $S_0=\frac{1}{0!}, S_1=\frac{1}{0!}+\frac{1}{1!}\dots$.\\
Jeżeli ciąg $S_n$ jest zbieżny to piszemy, że:
\begin{center}
$\sum_{n=1}^{\infty} a_n = \lim_{n\rightarrow \infty} S_N$
\end{center}
(granica to suma szeregu)
\begin{center}
    $S_1\leq S_2\leq S_3\leq S_N < M$
\end{center}
\end{de}

\begin{pk}
    $apart(1/(n\cdot(n+1)),n)\leftarrow$ wolframalpha
    \begin{center}
    $\frac{1}{1\cdot 2} + \frac{1}{2\cdot 3}+ \dots + \frac{1}{n(n+1)}=S_N$\\
    $S_1=\frac{1}{2}, S_2=\frac{1}{1\cdot 2}+\frac{1}{2\cdot 3}$, zatem:\\
    $\frac{1}{1\cdot 2} + \frac{1}{2\cdot 3}+ \dots + \frac{1}{n(n+1)}=$\\
    $=\frac{1}{1} - \frac{1}{2} + \frac{1}{2} - \frac{1}{3} + ... + \frac{1}{n} - \frac{1}{n+1}=$\\
    $=1-\frac{1}{n+1}$, finalnie:\\
    $\sum_{n=1}^{\infty} \frac{1}{n(n+1)} = lim_{n\rightarrow \infty} S_N = lim_{n\rightarrow \infty} (1 - \frac{1}{n+1}) = 1$
    \end{center}
\end{pk}

\begin{pk}
    $a+aq+...+aq^n=a\cdot \frac{1-q^{n+1}}{1-q}$, dla $|q|<1$:
    \begin{center}
    $\sum_{n=0}^{\infty} aq^n = lim_{n\rightarrow \infty} a\cdot \frac{1-q^{n+1}}{1-q} = \frac{a}{1-q}$
    \end{center}
\end{pk}

\begin{pk}
    $\sum_{n=0}^{\infty} \frac{1}{n!} = e$
\end{pk}

\begin{pk}
Szereg harmoniczny.
$H_N = \sum_{n=1}^{N} \frac{1}{n}$, $\lim_{N\rightarrow \infty}=\infty$, wolny wzrost do $\infty$\\
$H_{2^{n+1}}=\frac{1}{1} + \frac{1}{2} + \frac{1}{2+1} + \frac{1}{2+2} + \frac{1}{2^2 + 1} + \frac{1}{2^2+2} + \frac{1}{2^2+3} + \frac{1}{2^3} + \frac{1}{2^3+1} + \frac{1}{2^3 + 2} + \frac{1}{2^3 + 3} + \dots + \frac{1}{2^3 + 2^3} + \frac{1}{2^n + 1} + \frac{1}{2^n + 2} + \dots + \frac{1}{2^n+2^n}$\\\\
$\frac{1}{1}+\frac{1}{2}=\frac{3}{2}$\\
$\frac{1}{2+1}+\frac{1}{2+2}\geq 2\cdot \frac{1}{2+2}=\frac{1}{2}$\\
$\frac{1}{2^2+1}+\frac{1}{2^2+2} + ... \geq 4\cdot \frac{1}{2^2 + 2^2}=\frac{1}{2}$\\
$\frac{1}{2^3+1}+\frac{1}{2^3+2} + ... \geq 8\cdot \frac{1}{2^3 + 2^3}=\frac{1}{2}$\\
$\frac{1}{2^n+1}+\frac{1}{2^n+2} + ... \geq 2^{n}\cdot \frac{1}{2^n + 2^n}=\frac{1}{2}$\\
$H_{2^{n+1}}\geq \frac{3}{2} + \frac{1}{2} \cdot n = 1 + \frac{1}{2} (n+1)$\\
$H_{2^{n+1}}\geq 1 + \frac{n+1}{2}$\\
$H_{2^n}\geq 1 + \frac{n}{2}$\\\\
Założmy, że $2^N=k \implies N=log_2()$\\
$H_{k}\geq 1 + \frac{\log_2(k)}{2} \rightarrow \infty$\\
Na mocy twierdenia o dwóch ciągach $H_k \rightarrow \infty$
\end{pk}

Następny wykład - kryteria zbieżności szeregów: kryterium kondensacyjne.

\begin{de}
    Warunek konieczny zbieżności szeregów. Jeżeli $\sum_{n=1}^{\infty} a_n$ jest zbieżny, to $lim_{n\rightarrow \infty} a_n = 0$. (dla $\sum_{n=1}^{\infty} a_n < \infty$).
\end{de}

Szereg $\sum_{n=1}^{\infty} \frac{n}{n+1}$ jest rozbieżny, bo nie jest spełniony warunek konieczny\\ $lim_{n\rightarrow \infty} \frac{n}{n+1}=1$\\
Warunek konieczny nie jest wystarczający.

\end{document}
