\documentclass{article}

\usepackage[english]{babel}
\usepackage[utf8]{inputenc}
\usepackage{polski}
\usepackage[T1]{fontenc}
 
\usepackage[margin=1.5in]{geometry} 

\usepackage{color} 
\usepackage{amsmath}
\usepackage{amsfonts}                                                                   
\usepackage{graphicx}                                                             
\usepackage{booktabs}
\usepackage{amsthm}
\usepackage{pdfpages}
\usepackage{wrapfig}
\usepackage{hyperref}
\usepackage{etoolbox}
\usepackage{tikz}

\makeatletter
\newenvironment{definition}[1]{%
    \trivlist
    \item[\hskip\labelsep\textbf{Definition. #1.}]
    \ignorespaces
}{%
    \endtrivlist
}
\newenvironment{fact}[1]{%
    \trivlist
    \item[\hskip\labelsep\textbf{Fact. #1.}]
    \ignorespaces
}{%
    \endtrivlist
}
\newenvironment{theorem}[1]{%
    \trivlist
    \item[\hskip\labelsep\textbf{Theorem. #1.}]
    \ignorespaces
}{%
    \endtrivlist
}
\newenvironment{information}[1]{%
    \trivlist
    \item[\hskip\labelsep\textbf{Information. #1.}]
    \ignorespaces
}{%
    \endtrivlist
}
\newenvironment{identities}[1]{%
    \trivlist
    \item[\hskip\labelsep\textbf{Identities. #1.}]
    \ignorespaces
}{%
    \endtrivlist
}
\makeatother

\title{Programowanie Funkcyjne}  
\author{Rafał Włodarczyk}
\date{INA 4, 2025}

\begin{document}

\maketitle

\tableofcontents

\section{Haskell 03.06}

\subsection{Monada Writer}

\begin{align}
    \mathcal{M} &= (M,\cdot, e)\\
    W_{\mathcal{M}} (X) &= X \times M\\
    f &: X \rightarrow Y \\
    W_{\mathcal{M}} (f) &: W_{\mathcal{M}} (X) \rightarrow W_{\mathcal{M}} (Y)\\
    W_{\mathcal{M}} (f) &: X \times M \rightarrow Y \times M\\
    W_{\mathcal{M}} (f) (x, m) &= (f(x), m)
    W_{\mathcal{M}} (Y^X) &\rightarrow W_{\mathcal{M}} (X) \rightarrow W_{\mathcal{M}} (Y)\\
    (Y^X \times M) &\rightarrow (X \times M) \rightarrow (Y \times M)\\
    (f,m) <*> (x,n) &= (f(x), m + n)\\
    <|> \cdot W_{\mathcal{M}} (X) &\rightarrow W_{\mathcal{M}} (Y) \rightarrow W_{\mathcal{M}} (X \times Y)\\
    (x,m) <|> (y,n) &= ((x,y), m + n)
\end{align}

\begin{verbatim}
    (>>=) :: m a -> (a -> m b) -> mb
    f :: X -> Y x M    (x,m) \in X x M
    f (x) = (f1 (x), f2 (x))
    f = (f1, f2)

    ((x,m) >>= f) = let (y, n) = f(x)
        in (y, m * n)

    ((x,m) >>= (f1, f2)) = (f1(x), m * f2(x))

    pure x = return x = (x,e)
    prue : a -> ma

    Własności (aksjomaty) monadyczne w starych książkach.
    (A1) (mx >>= return) = mx
    (A2) ((return x) >>= f) = f x
    (A3) ((mx >>= f) >>= g) = (mx >>= (\x -> f(x) >>= g))
    (A4) mf <*> mx = do {f <- mf; x <- mx; return (fx)} 
                   = mf >>= (\f -> (mx >>= (\x -> return (fx))))
\end{verbatim}

W parserach do gramatyk bezkontekstowych możemy wykorzystać jedynie \text{<*>}

\begin{verbatim}
    Writer
    return x = (x,e)
    return = (id, const e)

    (A1) : ((x,m) >>= return) =
           ((x,m) >>= (id, const e)) = (x, m * e) = (x, m)

    (A2) : ((return x) >>= f) = (f1,f2) = (x,e) >>= (f1, f2) =
           (f1(x), e * f2(x)) = (f1(x),f2(x)) = f(x)

    (A3) : (((x,m) >>= f) >> g) = ...
\end{verbatim}

Pozostałe własności można sprawdzić automatycznie tą samą techniką.

\subsection{Operator Kleisli'ego}

Rybka. W języku Kleisli'ego własności monadyczne są jasno widoczne.

\begin{verbatim}
    (>=>) :: (a -> mb) -> (b -> mc) -> (a -> mc)
    (f >=> g) =def= (\x -> f(x) >>= g)
    (f >=> return)(x) = (f(x) >>= return) =A1= f(x)
    (A1') f >=> return = f

    (return >=> g)(x) = (return x) >>= g =A2= g(x)
    (A2') return >=> f = f
    (A3') (f >=> g) >=> h = f >=> (g >=> h)

    id >=> id // dla listy to konkatenacja listy list (operator spłaszczania)
\end{verbatim}

\subsection{Monada State}

\begin{verbatim}
    Fix S 
    ST (X) = (X x S)^S
    f : X -> Y
    ST (f) : ST(X) -> ST(Y)
           : (X x S)^S -> (Y x S)^S
    ST(f) (phi) = (\s : S -> let (x,t) = phi(s) in
                             (f(x),t))
    <*> = ??

    (>>=) :: ma -> (a -> mb) -> mb
    (>>=) : (X x S)^S -> (X -> (Y x S)^S) -> (Y x S)^S
    (phi >>= f) = \s -> let(x,t) = phi(s),
                        (y,r) = f(x)(t) in
                        (y,r)
                = \s -> let(x,t) = phi(s)
                        (y,r) = u(f(x,t)) in
                        (y,r)
                = \s -> let(x,t) = phi(s) = u(f) . phi
                    
    u(f)(x,t) = (fx)(t)

    (Zadanie) (f >=> g) = c (u(g) . u(f))

    return x = \s -> (x,s)
    u(return) = id
\end{verbatim}





\end{document}