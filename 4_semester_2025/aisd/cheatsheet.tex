\documentclass[landscape,a4paper]{article}
\usepackage[fontsize=2px]{fontsize}
\usepackage{amssymb,amsmath,amsthm,amsfonts}
\usepackage{multicol,multirow}
\usepackage{calc}
\usepackage{ifthen}
\usepackage[landscape]{geometry}
\usepackage[colorlinks=true,citecolor=blue,linkcolor=blue]{hyperref}

\ifthenelse{\lengthtest { \paperwidth = 11in}}
    {\geometry{top=0.3cm,left=0.3cm,right=0.3cm,bottom=0.3cm} }
	{\ifthenelse{ \lengthtest{ \paperwidth = 297mm}}
		{\geometry{top=0.3cm,left=0.3cm,right=0.3cm,bottom=0.3cm} }
		{\geometry{top=0.3cm,left=0.3cm,right=0.3cm,bottom=0.3cm} }
	}
\pagestyle{empty}
\makeatletter
\renewcommand{\section}{\@startsection{section}{1}{0mm}%
                                {-1ex plus -.5ex minus -.2ex}%
                                {0.5ex plus .2ex}%x
                                {\normalfont\large\bfseries}}
\renewcommand{\subsection}{\@startsection{subsection}{2}{0mm}%
                                {-1explus -.5ex minus -.2ex}%
                                {0.5ex plus .2ex}%
                                {\normalfont\normalsize\bfseries}}
\renewcommand{\subsubsection}{\@startsection{subsubsection}{3}{0mm}%
                                {-1ex plus -.5ex minus -.2ex}%
                                {1ex plus .2ex}%
                                {\normalfont\small\bfseries}}
\makeatother
\setcounter{secnumdepth}{0}
\setlength{\parindent}{0pt}
\setlength{\parskip}{0pt plus 0.5ex}
% -----------------------------------------------------------------------


\usepackage{listings}
\usepackage{xcolor}
\usepackage{hyperref}
\usepackage{parskip}
\usepackage{geometry}
\geometry{margin=0.5cm}

\begin{document}

\raggedright
\footnotesize

\begin{center}
    \textbf{Rafał Włodarczyk} (27.06.2025) https://github.com/Rafisto/uni/blob/master/4\_semester\_2025/aisd/cheatsheet.tex)
\end{center}

\begin{multicols}{5}
\setlength{\premulticols}{1pt}
\setlength{\postmulticols}{1pt}
\setlength{\multicolsep}{1pt}
\setlength{\columnsep}{2pt}

\section{Logarytmy}

$
\log_a b + \log_a c = \log_a(b \cdot c)
; \log_a b - \log_a c = \log_a\left(\frac{b}{c}\right);
;a^{\log_a b} = b
;n \cdot \log_a b = \log_a b^n
;\log_a b = \frac{\log_c b}{\log_c a}
;a^{\log_b n} = n^{\log_b a}
$

\section{Notacja Asymptotyczna}

$
    f(n) = O(g(n)) \equiv \limsup_{n\rightarrow \infty} \left|\frac{f(n)}{g(n)}\right| \leq \infty
$

$
    f(n) = \Omega(g(n)) \equiv \limsup_{n\rightarrow \infty} \left|\frac{g(n)}{f(n)}\right| \leq \infty
$

$
    f(n) = o(g(n)) \equiv \lim_{n\rightarrow \infty} \left|\frac{f(n)}{g(n)}\right| = 0
$

$
    f(n) = \omega(g(n)) \equiv \lim_{n\rightarrow \infty} \left|\frac{f(n)}{g(n)}\right| = \infty
$

\subsection{Metoda podstawiania - Metoda dowodu indukcyjnego}
Przykład 1. Rozwiążmy równanie rekurencyjne:
$
    T(n) = 4T\left(\frac{n}{2}\right) + n \quad T(1) = \Theta(1)
$
Wzmocnijmy zatem założenie indukcyjne:
\begin{enumerate}
    \item $T(n) \leq c_1 n^2 - c_2 n$ (zał. indukcyjne)
    \item $T(n) = 4T\left(\frac{n}{2}\right) + n \leq 4(c_1 \frac{n}{2}^2 - c_2 \frac{n}{2}) + n$
    \item $= c_1 n^2 - 2c_2 n + n = c_1n^2 - (2c_2 -1)n \leq$
    \item $\leq c_1n^2 - c_2n$
    \item Weźmy $c_1=1, c_2=2$, wtedy $T(n) \leq n^2 - 2n = O(n^2)$
\end{enumerate}

Przykład 2. Weźmy paskudną rekursję $T(n) = 2T(\sqrt{n}) + \log n$.\\
Załóżmy, że $n$ jest potęgą $2$ oraz oznaczny $n=2^m, m=\log_2 n$.
$
    T(2^m) = 2T((2^m)^\frac{1}{2}) + m
$
Oznaczmy $T(2^m) = S(m)$. Wtedy:
$
    S(m) = 2S\left(\frac{m}{2}\right) + m
$
(dobrze znana rekurencja - $S(n) = O(m\log m))$ - patrz Lecture 2.
Przejdźmy z powrotem na $T,n$:
$
    T(2^m) = S(m)\\
    T(2^m) = O(m \log m)\\
    T(n) = O(\log n \log \log n)
$

\section{Master Theorem}

$
    T(n) = a \cdot T\left(\lceil\frac{n}{b}\rceil\right) + \Theta(n^d)
$

$
    T(n) = \begin{cases}
        \Theta\left(n^d\right) \quad \text{jeśli} \quad d > \log_b a\\
        \Theta\left(n^d \log n\right) \quad \text{jeśli} \quad d = \log_b a\\
        \Theta\left(n^{log_b a}\right) \quad \text{jeśli} \quad d < \log_b a
    \end{cases}
$

\subsection{Fibonacci}

Istnieje macierz, która mnożona pozwala na policzenie $n$-tej liczby Fibonacciego.

$
    \begin{pmatrix}
        1 & 1 \\
        1 & 0
    \end{pmatrix}^n =
    \begin{pmatrix}
        F_{n+1} & F_{n} \\
        F_{n} & F_{n-1}
    \end{pmatrix}
$
Algorytm używający tego wzoru - połączony z szybkim potęgowaniem, ma złożoność $\Theta(\log n)$.

\subsection{Mnożenie liczb}

$
    (a+ib)(c+id) = ac - bd + i(bc + ad)\\
    bc + ad = (a + b)(c + d) - ac - bd
$
Zobaczmy, że $ac, bd$ są już policzone wyżej - zamiast 4 mnożeń, mamy 3 mnożenia.
$
    x\cdot y = x_L y_L 2^n + \left((x_L + x_R)(y_L + y_R\\ - x_Ly_L - x_Ry_R\right) + x_Ry_R\quad T(n) = \Theta(n^{\log_2 3})
$

\begin{verbatim}
mutiply(x, y)
    n = max {|x|, |y|}
    if n == 1 return x * y
    x_L, x_R = leftmost(ceil(n/2),x), rightmost(floor(n/2),x)
    y_L, y_R = leftmost(ceil(n/2),y), rightmost(floor(n/2),y)

    p1 = multiply(x_L, y_L)
    p2 = multiply(x_R, y_R)
    p3 = multiply(x_L + x_R, y_L + y_R)

    return p1 << n + (p3 - p1 - p2) << ceil(n/2) + p2
\end{verbatim}

\begin{verbatim}
COUNTING-SORT(A, B, k)
let C[0..k] be a new array
for i = 0..k
    C[i] = 0
for j = 1..length[A]
    C[A[j]] = C[A[j]] + 1
for i = 1..k
    C[i] = C[i] + C[i - 1]
for j = length[A]..1
    B[C[A[j]]] = A[j]
    C[A[j]] = C[A[j]] - 1
\end{verbatim}

\begin{verbatim}
BS(A[1..n], key) // BinarySearch O(log(n))
    if (key == BS[mid]) ret mid
    if (key >= BS[mid]) ret BS(A[mid...n], key)
    else ret BS(A[1...mid-1], key)
    ret -1
\end{verbatim}

\begin{verbatim}
LomutoPartition(A[1..n]) // O(n)
    pivot = A[1], i = 1 // any
    for j = 1 to n
        if A[j] <= pivot
            i = i + 1
            swap (A[i], A[j])
    swap (A[i+1], A[n])
    ret i + 1
\end{verbatim}

\begin{verbatim}
InsertionSort(A[1..n])
for j = 2...n
    key = A[j]
    i=j-1
    while(i>0 && A[i]>key)
        A[i+1] = A[i]
        i = i - 1
    A[i+1] = key
\end{verbatim}

\begin{verbatim}
MergeSort(A[1..n]) // O(nlogn)
    if (n==1) return A[1]
    return Merge(
        MergeSort(A[1, mid]),
        MergeSort(A[mid+1,n])
    )

Merge(A[1..k], B[1..l]) // O(k+l)
    i = 1, j = 1
    RES = []
    while (i <= k) AND (j <= l)
        if (A[i] <= B[j]) RES+=A[i], i++
        else RES+=B[j], j++
    RES+=A[i..k] OR RES+=B[j..l]
    ret RES
\end{verbatim}

\begin{verbatim}
RandomSelect(A, p, q, i)
    if p == q return A[p]
    r = Rand_Partition(A,p,q) # Hoare
    k = r - p + 1
    if i == k return A[r]
    if i < k return RandomSelect(A, p, r-1, i)
    else return RandomSelect(A, r+1, q, i-k)
\end{verbatim}

\begin{verbatim}
Select5(A, p, q, i)
    divide A[p..q] into groups of 5 elements
    for each group sort it and find its median
    store these medians in a new array M
    mms = Select5(M, 0, len(M) - 1, len(M)/2)
    r = Partition A[p..q] around mms
    k = r - p + 1
    if (i == k) return A[r]
    if (i < k) return Select5(A, p, r - 1, i)
    else return Select5(A, r + 1, q, i - k)
\end{verbatim}

\subsection{Struktury Danych}

\subsection{Drzewa}

Zbalansowane drzewo $h=O(\log_2 n)$

Drzewo statystyk pozycyjnych - RBT z rozmiarem poddrzewa.

\begin{verbatim}
OS_Select(x, i) // O(log n), wykonaj x=root
    r = size(x.left) + 1
    if i == r return x
    if i < r return OS_Select(x.left, i)
    else return OS_Select(x.right, i - r)
\end{verbatim}

\begin{verbatim}
OS_Rank(x) // O(log n), statystyka wezla x
    r = x.left.size + 1
    y = x
    while (y != root) 
        if (y == y.p.right) // y jest prawym synem
            r += size(y.parent.left) + 1
        y = y.p
    return r
    
\end{verbatim}

\subsection{Kopce}

\begin{verbatim}
left(i) = 2i (LSH)
right(i) = 2i + 1 (LSH + 1)
parent(i) = i // 2 (RSH)
size(A) = rozmiar listy
\end{verbatim}

\begin{verbatim}
HEAPIFY(A, i) // O(h)
    l = left(i), r = right(i)
    if (l <= size(A) AND A[l] > A[i]) 
        largest = l
    else largest = i
    if (r <= size(A) AND A[r] > A[largest])
        largest = r
    if (largest != i) swap(A[i], A[largest])
    HEAPIFY(A, largest)
\end{verbatim}

\subsection{PQ}

\begin{verbatim}
Maximum(Q) return Q[1]

Union(Q1,Q2) // O(|Q1|+|Q2|)
    BuiltHeap([Q1,Q2])

Insert(Q,key) // O(log n)
    size(Q)++, i = size(Q)
    while(i>1 AND Q[parent(i)<key]
        Q[i] = Q[oarent(i)]
        i = parent(i)
    Q[i]=key

ExtractMax(Q) // O(log n)
    max = Q[1]
    Q[1] = Q[size(Q)]
    size(Q)--
    HEAPIFY(Q,1)
    return max

Delete(Q, i) // O(log n)
Q[i] = Q[size(Q)]
size(Q)--
if (Q[i]<Q[parent(i)]) HEAPIFY(Q,i)
else
    while(i<1 AND Q[parent(i)]<Q[i])
        swap (Q[parent(i)],Q[i])
        i = parent(i)

Increase/DecreaseKey(A, i, newK)
if A[i] > newK
    A[i] = newK
    HEAPIFY(A,i)
else
    while i>1 AND A[parent(i)] < newK
        A[i] = A[parent(i)]
        i = parent(i)
    A[i] = newK
\end{verbatim}

\subsection{Grafy}

\subsection{Cut Property}

Niech $X$ będzie podzbiorem krawędzi minimalnego drzewa rozpinającego grafu $G=(V,E)$. Wybierzmy podzbiór wierzchołków $S\subset V$,
takich, że żadna krawędź z $X$ nie przechodzi pomiędzy wierzchołkami z $S$ i $V\setminus S$. Niech $e\in E$ będzie krawędzią o najmniejszej
wadze, która przechodzi pomiędzy $S$ i $V\setminus S$. Wtedy $X\cup \{e\}$ należy do minimalnego drzewa rozpinającego grafu $G$.

\subsection{Min-Cut Problem}

Możemy zbudować minimalne drzewo rozpinające, w $S$ $V-S$:
\begin{align}
    \text{Pr}(A\in\text{MinCut}) \geq \frac{1}{n(n-1)}
\end{align}
Możemy stworzyć algorytm Kruskala (nieskierowany, więc nie musimy sortować krawędzi) do ostatniego jego kroku, w którym miałby on znaleźć ostatnią
krawędź, która przechodzi pomiędzy $S$ i $V-S$. Wtedy ta krawędź rozspójniłaby graf.

Powtórzmy $\Theta(n^2)$ razy algorytm Kruskala, aby znaleźć minimalne przecięcie - wraz z $n$ dążącym do nieskończoności porafimy wyznaczyć najmniejszy zbiór rozcinający.

Sortowanie Topologiczne
Wykonujemy DFS, zapisując czasy pre i post.
Sortujemy wierzchołki według czasu post w porządku malejącym.

Mamy krawędź między składowymi silnie spójnymi (SCC), jeśli istnieje krawędź między wierzchołkami tych składowych w oryginalnym grafie.

Odwracamy kierunek wszystkich krawędzi, otrzymując graf $G^T$.
Wykonujemy DFS na grafie $G^T$.

Ujście $G$ ma największy post w $G^T$.
DFS na $G$ zaczyna się od ujścia $G$.

\textbf{Komponenty:} \\
\textit{Silnie Spójne Składowe} — w obrębie jednej komponenty można dojść do każdego węzła.

Źródło - Wierzchołek grafu skierowanego, nie będący końcem żadnej krawędzi.\\
Ujście - Wierzchołek grafu skierowanego, nie będący początkiem żadnej krawędzi.

\begin{verbatim}
EXPLORE(G,v) # G - Graph, v - start vertex
    visited(v) = true
    previsit(v)
    for each edge (v,u) in E
        if not visited(u) EXPLORE(G,u)
    postvisit(v)
\end{verbatim}

\begin{verbatim}
DFS(G) // O(|V|+|E|)
    for each vertex v in G
        visited(v) = false
    for each vertex v in G
        if not visited(v) EXPLORE(G,v)
\end{verbatim}

\begin{verbatim}
BFS(G, s) // O(|V|+|E|)
    for each vertex v in G
        visited(v) = false
    visited(s) = true
    enqueue(Q, s)
    while Q not empty
        v = dequeue(Q)
        for each neighbor u of v
            if not visited(u)
                visited(u) = true
                enqueue(Q, u)
\end{verbatim}

\begin{verbatim}
Bellman-Ford(G, s) // O(|V|*|E|)
for all v in V
    dist(v) = infinity
    prev(v) = null
dist(s) = 0
repeat |V|-1 times
    for all e in E
        if dist(u) + w(u,v) < dist(v):
            dist(v) = dist(u) + w(u,v)
            prev(v) = u
\end{verbatim}

\begin{verbatim}
Dijkstra(G, s) // O((|V|+|E|) log |V|)
for each vertex v in G
    dist(v) = infinity
    visited(v) = false
dist(s) = 0
Q = makePQ(vertices) // by dist
while Q not empty
    u = extract-min(Q)
    if visited(u) continue
    visited(u) = true
    for each neighbor v of u
        if dist(v) > dist(u) + w(u,v)
            dist(v) = dist(u) + w(u,v)
            decrease-key(Q, v, dist(v))
\end{verbatim}

\begin{verbatim}
Prim(G=(V,E), (w_i)i=1,...|E|) -> MST dla grafu G
for v in V 
    cost(v) = infinity
    prev(v) = null
cost(u) = 0
H = MakePQ(V) // priorytetem jest cost(v)
while H is not empty
    v = ExtractMin(H)
    for each {v,z} in E
        if cost(z) > w(v,z)
            cost(z) = w(v,z)
            prev(z) = v
            decreaseKey(H,v,cost(z))
\end{verbatim}

\begin{verbatim}
Kruskal(G=(V,E), w) // O(|E| log |E|), O(|E| log |V|)
for all v in V
    makeSet(v)
x = {}
for all {u,v} in sorted E
    if find(u) != find(v)
        x = x U ({u,v})
        union(u,v)
ret x
\end{verbatim}

\section{House Robber}

\begin{verbatim}
rob(values: list):
    prev, curr = 0, 0
    for value in values:
        prev, curr = curr, 
            max(curr, prev + value)
    return curr
\end{verbatim}

\section{Edit Distance}

$|W_1|=i, |W_2|=j$. Miminum z możliwości do $(i,j)$:

\begin{itemize}
    \item insert $(i,j-1)\rightarrow(i,j)$ (+1)
    \item delete $(i-1,j)\rightarrow(i,j)$ (+1)
    \item replace $(i-1,j-1)\rightarrow(i,j)$ (+1)
    \item keep $(i-1,j-1)\rightarrow(i,j)$ (+0)
\end{itemize}

\begin{verbatim}
for (size_t n = 1; n <= s1; ++n) {
    for(size_t m = 1; m <= s2; ++m) {
        dp[n][m] = std::min({
        dp[n-1][m] + 1,
        dp[n][m-1] + 1,
        dp[n-1][m-1] +
        diff(word1[n-1], word2[m-1])});
    }
}
\end{verbatim}

\section{0/1 Knapsack}

$
    max(v+arr[i-1][j-w], arr[i-1][j])
$

\section{Coin change 1}

Minimalna liczba monet potrzebna do wydania reszty

\begin{verbatim}
for(int i = 1; i <= amount; ++i) {
    for(int j = 0; j < coins.size(); ++j) {
        if (coins[j] <= i) {
            L[i] = std::min(L[i], 1 + L[i - coins[j]]);
        }
    }
}
\end{verbatim}

\section{Coin change 2}

Unbounded Knapsack. Na ile sposobów możemy wydać resztę amount.

\begin{verbatim}
def change(amount: int, coins: List[int]) -> int:
    dp = [0] * (amount + 1)
    dp[0] = 1

    for c in coins:
        for a in range(c, amount + 1): 
            dp[a] += dp[a-c]

    return dp[amount]
\end{verbatim}

\section{Longest Common Subsequence}

Największy wspólny podciąg. "ace" is a subsequence of "abcde".

\begin{verbatim}
int longestCommonSubsequence(string &a, string &b) {
    short m[1001][1001] = {0};
    for (auto i = 0; i < a.size(); ++i)
        for (auto j = 0; j < b.size(); ++j)
            m[i + 1][j + 1] = a[i] == b[j] 
            ? m[i][j] + 1
            : max(m[i + 1][j], m[i][j + 1]);
    return m[a.size()][b.size()];
}
\end{verbatim}

\section{Longest Increasing Subsequence}

\begin{verbatim}
def lengthOfLIS(self, nums: List[int]) -> int:
    if not nums:
        return 0
        
    n = len(nums)
    dp = [1] * n

    for i in range(1, n):
        for j in range(i):
            if nums[i] > nums[j]:
                dp[i] = max(dp[i], dp[j] + 1)

    return max(dp)
\end{verbatim}

\section{Perły}

Potrzeba $(a_i,p_i)$ ale możemy zastępować gorsze perły lepszymi.

\begin{verbatim}
REP(i, MAX)
    REP(j, MAX)
        t[i][j] = INF;
t[n - 1][n - 1] = (a[n - 1] + 10) * p[n - 1];
FORD(i, n - 2, 0)
{
    t[i][i] = (a[i] + 10) * p[i] + 
        *min_element(t[i + 1], t[i + 1] + MAX);
    FOR(j, i + 1, n - 1)
        t[i][j] = a[i] * p[j] + t[i + 1][j];
}
return *min_element(t[0], t[0] + MAX)
\end{verbatim}

\subsection{Red Black Trees}

'78 Guibas, Sedgewick - Red Black (RB) Trees 
\begin{itemize}
    \item Własność 0 - Drzewa RB są drzewami BST - mają BST Property - po lewej stronie węzła występują wartości mniejsze, a po prawej większe
    \item Własność 1 - Każdy węzeł ma kolor czerwony albo czarny (to może być bit)
    \item Własność 2 - Korzeń oraz \textit{liście} są czarne
    \item Własność 3 - Jeśli węzeł jest czerwony, to jego bezpośrednie dzieci są czarne
    \item Własność 4 - $\forall X$ Każda prosta ścieżka od węzła X do liści ma tyle samo czarnych węzłów. $(\text{black\_height}(x)$, inaczej $ \text{bh}(x))$. Prosta ścieżka oznacza, że nie zawracamy, zawsze idziemy w dół.
\end{itemize}

\end{multicols}

\begin{table}[h!]
\centering
\begin{tabular}{|l|l|l|l|l|l|l|}
\hline
\textbf{Struktura} & \textbf{Build} & \textbf{Find} & \textbf{Insert/Delete} & \textbf{Find mM} & \textbf{Find pn} & \textbf{List\_ordered} \\ \hline
Unsorted Array & $\Theta(n)$ & $\Theta(n)$ & $\Theta(n)$ & $\Theta(n)$ & $\Theta(n)$ & $\Theta(n\log n) $ \\ \hline
Sorted Array & $\Theta(n\log n)$ & $\Theta(\log n)$ & $\Theta(n)$ & $\Theta(1)$ & $\Theta(\log n)$ & $\Theta(n)$ \\ \hline
Linked List & $\Theta(n)$ & $\Theta(n)$ & $\Theta(n)$ & $\Theta(n)$ & $\Theta(n)$ & $\Theta(n\log n)$ \\ \hline
BST & $\Theta(n^2)$ & $\Theta(n)$ & $\Theta(n)$ & $\Theta(n)$ & $\Theta(n)$ & $\Theta(n)$ \\ \hline
RBT & $\Theta(n \log n)$ & $\Theta(\log n)$ & $\Theta(\log n)$ & $\Theta(\log n)$ & $\Theta(\log n)$ & $\Theta(n)$ \\ \hline
MinHeap & $\Theta(n\log n)$ & $\Theta(n)$ & $\Theta(\log n)$ & $\Theta(1)$ (min) / $\Theta(n)$ (max) & $\Theta(n)$ & $\Theta(n \log n)$ \\ \hline
\end{tabular}
\caption{Porównanie różnych struktur danych (złożoność w najgorszym przypadku)}
\label{tab:structure_comparison_worst_case}
\end{table}

\end{document}