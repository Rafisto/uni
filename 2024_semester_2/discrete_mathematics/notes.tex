\documentclass{article}

\usepackage[english]{babel}
\usepackage[utf8]{inputenc}
\usepackage{polski}
\usepackage[T1]{fontenc}
 
\usepackage[margin=1.5in]{geometry} 

\usepackage{color} 
\usepackage{amsmath}
\usepackage{amsfonts}                                                                   
\usepackage{graphicx}                                                             
\usepackage{booktabs}
\usepackage{amsthm}
\usepackage{pdfpages}
\usepackage{wrapfig}
\usepackage{hyperref}
\usepackage{etoolbox}

\makeatletter
\newenvironment{definition}[1]{%
    \trivlist
    \item[\hskip\labelsep\textbf{Definition. #1.}]
    \ignorespaces
}{%
    \endtrivlist
}
\newenvironment{fact}[1]{%
    \trivlist
    \item[\hskip\labelsep\textbf{Fact. #1.}]
    \ignorespaces
}{%
    \endtrivlist
}
\newenvironment{theorem}[1]{%
    \trivlist
    \item[\hskip\labelsep\textbf{Theorem. #1.}]
    \ignorespaces
}{%
    \endtrivlist
}
\newenvironment{information}[1]{%
    \trivlist
    \item[\hskip\labelsep\textbf{Information. #1.}]
    \ignorespaces
}{%
    \endtrivlist
}
\newenvironment{identities}[1]{%
    \trivlist
    \item[\hskip\labelsep\textbf{Identities. #1.}]
    \ignorespaces
}{%
    \endtrivlist
}
\makeatother

\title{Discrete Mathematics}  
\author{Rafał Włodarczyk}
\date{INA 2, 2024}

\begin{document}

\maketitle

\tableofcontents

\section{Basic formulas and operators}

\subsection{Factorials}

\begin{definition}{Factorial}
    Factorial of a non-negative integer $n$, denoted by $n!$, is the product of all positive integers less than or equal to $n$. 
\end{definition}

\begin{definition}{Falling Factorial}
     Falling factorial (sometimes called the descending factorial) is defined as the polynomial:
     $$
     \begin{aligned}
        (x)_{n}=x^{\underline {n}}&=\overbrace {x(x-1)(x-2)\cdots (x-n+1)} ^{n{\text{ factors}}}\\
        &=\prod _{k=1}^{n}(x-k+1)=\prod _{k=0}^{n-1}(x-k).
     \end{aligned}
     $$
\end{definition}

\begin{definition}{Rising Factorial}
    Rising factorial (sometimes called the descending factorial) is defined as the polynomial:
    $$
    \begin{aligned}
        x^{(n)}=x^{\overline {n}}&=\overbrace {x(x+1)(x+2)\cdots (x+n-1)} ^{n{\text{ factors}}}\\
        &=\prod _{k=1}^{n}(x+k-1)=\prod _{k=0}^{n-1}(x+k).
    \end{aligned}
    $$
\end{definition}

\subsection{Binomial Coefficient}

\begin{definition}{Binomial Coefficient}
    Let $n,k \in \mathbb{N}$ and $n\geq k$. The binomial coefficient is the number of $k$-element subsets of an $n$-element set, and it is defined as:
    \[\binom{n}{k}=\frac{n^{\underline{k}}}{k!}=\frac{n!}{k!(n-k)!}\]
    Furthermore let $x\in \mathbb{R}$, and again $k\in\mathbb{N}$. Then we define the binomial coefficient as:
    \[\binom{x}{k}=\frac{x^{\underline{k}}}{k!}\]
\end{definition}

\subsection{Binomial Coefficient Identities}

\begin{identities}{Binomial Coefficient}
    The binomial coefficient carries within itself a lot of identities, most of which can be easily observed in the Pascal's Triangle:
    \begin{enumerate}
        \item First identity.
        \[\binom{n}{k}=\binom{n}{n-k}\]
        \item Recursion for binomial coefficients.
        \[\binom{n+1}{k+1}=\binom{n}{k}+\binom{n}{k+1}\]
        alternatively re-indexed as $\binom{n+1}{k}=\binom{n}{k-1}+\binom{n}{k-1}$.
        \item Another recursion.
        \[\binom{n+1}{k+1} = \frac{n+1}{k+1} \binom{n}{k}\]
        \item Another identity.
        \[\binom{n}{k}\binom{k}{j} = \binom{n}{j}\binom{n-k}{k-j}\]
        \item Bookkeeper sum.
        \[\sum_{k=2}^{n} \binom{k}{2} = \binom{n}{3}\]
        \item Sum of coefficients.
        \[\sum_{k=0}^{n} \binom{n}{k} = 2^n\]
        \[\sum_{k=0}^{n} \binom{n}{k} k^p = n^p 2^{n-1}\]
        \item Pascal diagonal sums.
        \[\sum_{j=k}^{n} \binom{j}{k}=\binom{k}{k} + \binom{k+1}{k} + \binom{k+2}{k} + \dots + \binom{n}{k} = \binom{n+1}{k+1}\]
        \item Alternating Sums.
        \[\sum_{\substack{k=0 \\ k \text{ even}}}^{n} \binom{n}{k} =\sum_{\substack{k=0 \\ k \text{ odd}}}^{n} \binom{n}{k}\]
        \item Strong sum.
        \[\sum_{k=0}^{n} \binom{n}{k} kx^{k-1} = n(1+x)^{n-1}\]
    \end{enumerate}
\end{identities}

\subsection{Binomial Theorem}

\begin{theorem}{Binomial Theorem}
    The expansion of any non-negative integer power $n\in\mathbb{Z}^{+}$ of the binomial $(x + y): x,y\in\mathbb{R}$ is a sum of the form:
    \[ (x+y)^{n}=\sum _{k=0}^{n} \binom{n}{k} x^{n-k}y^{k}=\sum _{k=0}^{n} \binom{n}{k} x^{k}y^{n-k}. \]
    Notable example for when $y=1$:
    $$
    \begin{aligned}
        (1+x)^{n}&=\binom{n}{0}x^{0}+\binom{n}{1}x^{1}+\binom{n}{2}x^{2}+\cdots +\binom{n}{n-1}x^{n-1}+\binom{n}{n}x^{n}\\
        &=\sum _{k=0}^{n}\binom{n}{k}x^{k}.{\vphantom {\Bigg )}}
    \end{aligned}
    $$
\end{theorem}

\subsection{Vandermonde Convolution Identity}

\begin{theorem}{Vandermonde's Convolution Identity}
    Let $m,n, k \in \mathbb{N}$. The identity states:
    \[\binom{m+n}{r}=\sum _{k=0}^{r}\binom{n}{k}\binom{n}{r-k}\]
\end{theorem}

\subsection{Binomial Coefficient Combinatorics}

\begin{information}{Choosing $k$ elements from $n$}
    Let $n, k \in\mathbb{N}, k\leq n$. Combinatorial formulas for choosing \( k \) elements from \( n \):
    \begin{table}[htbp]
        \centering
        \bgroup
        \def\arraystretch{1.5}%
        \begin{tabular}{|c|c|c|}
            \hline
            \textbf{Selection Method} & \textbf{Order} & \textbf{No Order} \\
            \hline
            No Repetition & \( n^{\underline{k}} \) & \( \binom{n}{k} \) \\
            \hline
            Repetition & \( n^{k} \) & \( \binom{n+k-1}{k} \) \\
            \hline
        \end{tabular}
        \egroup
    \end{table}
\end{information}

\section{Combinatorical Principles}

\subsection{Inclusion–exclusion principle}

\begin{definition}{Inclusion–exclusion principle}
    Inclusion–exclusion principle is a counting technique which generalizes the familiar method of obtaining the number of elements in the union of two finite sets. Symbolically expressed as: 
    \[|A\cup B|=|A|+|B|-|A\cap B|\]
    \[|A\cup B\cup C|=|A|+|B|+|C|-|A\cap B|-|A\cap C|-|B\cap C|+|A\cap B\cap C|\]
    For $n={2,3}$. Or further in general $n\in\mathbb{N}$ by the formula:
    \[|A_1\cup A_2\cup \dots \cup A_n| = 
    \sum_{i=1}^{n} A_i
    - \sum_{1\leq i\leq j\leq n}^{n} |A_i \cap A_j |
    + \sum_{1\leq i\leq j\leq k\leq n} |A_i \cap A_j \cap A_k|
    + \dots 
    + (-1)^{n+1} |A_1\cap A_2\cap \dots \cap A_n|\]
\end{definition}

\subsection{Pigeonhole principle}

\begin{definition}{Pigeonhole principle}
    Let $S$ be a finite set. Let $s_1, s_2, \dots, s_k$ be the subsets, which satisfy $\left(\forall i\neq j\right) i,j\in[k] s_i \cap s_j = \emptyset$ and $s_1 \dot\cup s_2 \dot\cup s_3 \dot\cup \dots \dot\cup s_k = S$. Then:
    \[\left(\exists i\in[k]\right) |s_i| \geq \frac{|S|}{k} \]
\end{definition}

\section{Asymptotic Notation}

\begin{enumerate}
    \item \(H_n \approx \ln(n)\)
    \item \(\sum_{k=1}^{n} k^s \approx \frac{k^{s+1}}{s+1}\) \(\in O(k^{s+1})\)
\end{enumerate}

\subsection{Big O}

\begin{definition}{Big O Asymptotic Notation}
    Let $g: \mathbb{N}\rightarrow \mathbb{R}^{+}$ We define:
    \[O\left(g(n)\right) = \{f:\mathbb{N}\rightarrow \mathbb{R}^{+} : \left(\exists c\in\mathbb{R}^{+}\right) \left(\exists n_0\in\mathbb{N}\right) \left(\forall n>n_0\right) f(n) \leq c\cdot g(n)\}\]
    For when $g: \mathbb{N}\rightarrow \mathbb{R}$ one can write $|f(n)| \leq |c\cdot g(n)|$.\\
    Even though $O(g(n))$ is clearly a set we often write $f=O(g(n))$, instead of $f\in O(g(n))$.
\end{definition}

\begin{fact}{Big O Limit}
    Let $f,g \in\mathbb{N}\rightarrow\mathbb{R}^{+}$. As a fact:
    \[f(n)=O(g) \iff \limsup_{n\rightarrow\infty} \frac{f(n)}{g(n)} < \infty\]
\end{fact}

\subsection{Big Theta}

\begin{definition}{Big Theta Asymptotic Notation}
    Let $g: \mathbb{N}\rightarrow \mathbb{R}^{+}$ We define:
    \[\Theta\left(g(n)\right) = \{f:\mathbb{N}\rightarrow \mathbb{R}^{+} : \left(\exists c_1, c_2\in\mathbb{R}^{+}\right) \left(\exists n_0\in\mathbb{N}\right) \left(\forall n>n_0\right) c_1\cdot g(n) \leq f(n) \leq c_2\cdot g(n)\}\]
    Furthermore:
    \[f(n)=\Theta(g(n)) \iff 
    \begin{cases} 
        f(n) = O(g(n))\\
        g(n) = O(f(n))
    \end{cases}
    \]
\end{definition}

\begin{fact}{Big Theta Limit}
    Let $f,g \in\mathbb{N}\rightarrow\mathbb{R}^{+}$. As a fact:
    \[f(n)=\Theta(g) \iff \left(\limsup_{n\rightarrow\infty} \frac{f(n)}{g(n)} < \infty \right) \land \left(\limsup_{n\rightarrow\infty} \frac{f(n)}{g(n)} > 0 \right)\]
\end{fact}

\subsection{Approximate Notation}

\begin{definition}{$\approx$ Notation}
    Let $f,g \in\mathbb{N}\rightarrow\mathbb{R}^{+}$. We define:
    \[f(n) \approx g(n) \iff \lim_{n\rightarrow \infty} \frac{f(n)}{g(n)} = c \in \mathbb{R}^{+}\]
\end{definition}

\section{Integral Sum Approximation}

\begin{theorem}{Sum Approximation}
Let \(a,b \in \mathbb{N}, f:[a,b]\rightarrow\mathbb{R}\) \textbf{non-decreasing}, differentiable. Then:

\[
f(a) + \int_{a}^{b} f(x) dx \leq \sum_{k=a}^{b} f(k) \leq \int_{a}^{b} f(x) dx + f(b)
\]
Analogically. Let \(a,b \in \mathbb{N}, f:[a,b]\rightarrow\mathbb{R}\) \textbf{non-increasing}, differentiable. Then:

\[
f(a) + \int_{a}^{b} f(x) dx \geq \sum_{k=a}^{b} f(k) \geq \int_{a}^{b} f(x) dx + f(b)
\]  
\end{theorem}

\subsection{Stirling formula}

\[n! \approx \sqrt{2\pi n} \cdot \left(\frac{n}{e}\right)^n\]

\section{Stirling numbers of the second kind}

We define \( \genfrac\{\}{0pt}{1}{n}{k} \) as the number of ways to partition a set of $n$ objects into $k$ non-empty subsets.

\subsection{Basic values}

\begin{enumerate}
    \item \( \genfrac\{\}{0pt}{1}{0}{0} = 1, \genfrac\{\}{0pt}{1}{n}{0} = 0\)
    \item \( \genfrac\{\}{0pt}{1}{n}{n} = \genfrac\{\}{0pt}{1}{n}{1} = 1\)
    \item \( \genfrac\{\}{0pt}{1}{n}{n-1} = \binom{n}{2}\)
    \item \( \genfrac\{\}{0pt}{1}{n}{2} = \frac{2^n-2}{2} = 2^{n-1}-1\)
\end{enumerate}

\subsection{Properties}

\begin{enumerate}
    \item Explicit formula
    \[
    \genfrac\{\}{0pt}{0}{n}{k} = \frac{1}{k!} \sum_{j=0}^{k} \binom{k}{j} (k-j)^n (-1)^j 
    \]
    \item Pascal identity:
    \[
    \genfrac\{\}{0pt}{0}{n}{k} = \genfrac\{\}{0pt}{0}{n-1}{k-1} + k \genfrac\{\}{0pt}{0}{n-1}{k}
    \]
    \item Expansion
    \[
    x^n = \sum_{k=0}^{n} \genfrac\{\}{0pt}{0}{n}{k} x^{\underline{k}}
    \]
    \item Boundary for triangle row inequality at \(k_n ~ \frac{n}{\ln(n)}\)
    \[
    \genfrac\{\}{0pt}{0}{n}{1} \leq \dots \leq \genfrac\{\}{0pt}{0}{n}{k_n} \geq ... \geq \genfrac\{\}{0pt}{0}{n}{n}
    \]
\end{enumerate}

\subsection{Bell Numbers}

Bell number $B_n$ is the number of all partitions of an $n$-element set:
\[
B_n = \sum_{k=0}^{n} \genfrac\{\}{0pt}{0}{n}{k}
\]
Bell numbers satisfiy the following recurrence relation:
\[
\begin{cases}
B_{n+1} = \sum_{k=0}^{n} \binom{n}{k} B_k\\
B_0 = 1
\end{cases}
\]

\subsection{Stirling Number Combinatorics}

\begin{information}{Choosing $k$ elements from $n$}
    Let $n, k \in\mathbb{N}, k\leq n$. Combinatorial formulas for choosing \( k \) non empty subsets from a set of size \( n \):
    \begin{enumerate}
        \item TOP - Elements
        \item SIDE - Subsets
    \end{enumerate}
    \begin{table}[htbp]
        \centering
        \bgroup
        \def\arraystretch{1.5}%
        \begin{tabular}{|c|c|c|}
            \hline
            \textbf{Selection Method} & \textbf{Distinguishable} & \textbf{Non-distinguishable} \\
            \hline
            Distinguishable & \( \genfrac\{\}{0pt}{1}{n}{k}\cdot k! (\text{surj.}) \) & \( \binom{n-1}{k-1} \) \\
            \hline
            Non-distinguishable & \( \genfrac\{\}{0pt}{1}{n}{k} \) & \( \binom{n+k-1}{k} \) \\
            \hline
        \end{tabular}
        \egroup
    \end{table}
\end{information}

\section{Permutations}

\subsection{Permutation}
A \textbf{permutation} of a set \(A\) is a bijection from the set \(A\) to itself. A permutation \(\sigma\) can be written as:
\[\sigma : A \to A\]
where \(\sigma\) reorders the elements of \(A\).\\
If \(|A|=n\), without loss of generality we can assume: \(A = \{1, 2, \ldots, n\}\).

\subsection{Set of permutations}

\(S_n = \{f: [n]  \xrightarrow[\text{bijection }]{} [n]\} \quad \text{and} \quad |S_n| = n!\)

\subsection{Cycle}
A \textbf{cycle} in a permutation \(\sigma\) is a subset of elements in \(S\) that are permuted among themselves, with each element mapping to the next element in the subset, and the last element mapping back to the first. A cycle of length \(k\) is written as:
\[
\sigma = (a_1 \, a_2 \, \ldots \, a_k)
\]
indicating that \(\sigma(a_i) = a_{i+1}\) for \(i = 1, 2, \ldots, k-1\) and \(\sigma(a_k) = a_1\).

\subsection{Two-Line Notation for Permutations}
In \textbf{two-line notation}, a permutation \(\sigma\) is written as:
\[
\sigma = \begin{pmatrix}
1 & 2 & \cdots & n \\
\sigma(1) & \sigma(2) & \cdots & \sigma(n)
\end{pmatrix}
\]
where the top row lists the elements of the set \(S\), and the bottom row lists their images under \(\sigma\).\\
For example:

\[
\sigma = \begin{pmatrix}
1 & 2 & 3 & 4 & 5 \\
2 & 3 & 1 & 5 & 4
\end{pmatrix}
\]

\subsection{One-Line Notation for Permutations}
In \textbf{one-line notation}, a permutation \(\sigma\) is written as a partition into disjoint cycles:
\[
\sigma = (1\, 2\, 3)(4\, 5)
\]

\subsection{Fixed point}

Let \(\sigma\) be a permutation of a set \(S\). A \textit{fixed point} of \(\sigma\) is an element \(x \in S\) such that \(\sigma(x) = x\).\\
For example \textit{Id.} (identity) has $n$ fixed points.

\subsection{Derangement}

A \textbf{derangement} is a permutation of a set where no element appears in its original position. More formally, for a set of \( n \) elements, a derangement is a permutation \( \sigma \) such that \( \sigma(i) \neq i \) for all \( i \) in the set.

\[
D_n = n! \sum_{k=0}^{n} \frac{(-1)^k}{k!}
\]

\subsection{Transposition}
A \textbf{transposition} is a cycle of length 2, i.e., it swaps two elements and leaves the others unchanged. It is written as:
\[
\sigma = (a \, b)
\]
indicating that \(\sigma(a) = b\) and \(\sigma(b) = a\), with \(\sigma(x) = x\) for all \(x \ne a, b\).

\subsection{Inversion}

Let \(\sigma\in S_n\). An \textit{inversion} is a pair \((\sigma(i),\sigma(j))\), which satisfies:
\[i < j \text{ and } \sigma(i) > \sigma(j) \]
One may think these two are "not in order".

\subsection{Sign of a permutation (sgn)}
The \textbf{sign} (or \textbf{parity}) of a permutation \(\sigma\), denoted \(\text{sgn}(\sigma)\), is defined as number of inversions in a permutation. It satisfies the following property:
\[
\text{sgn}(\sigma) = (-1)^{N(\sigma)}
\]
Where \(N(\sigma)\) is number of transpositions in the decomposition of \(\sigma\).\\
A permutation is called even if \(\text{sgn}(\sigma) = +1\) and odd if \(\text{sgn}(\sigma) = -1\).\\
For example:\\
Consider the permutation \(\sigma = (1\, 3\, 2)\). This can be decomposed into transpositions as:
\[
\sigma = (1\, 3)(3\, 2)
\]
Since there are 2 transpositions, \(\text{sgn}(\sigma) = (-1)^2 = 1\). Therefore, \(\sigma\) is an even permutation.

\subsection{Order of a permutation (ord)}
The \textbf{order} of a permutation \(\sigma\), denoted \(\text{ord}(\sigma)\), is the smallest positive integer \(k\) such that \(\sigma^k\) is the identity permutation. Formally,
\[
\text{ord}(\sigma) = \min \{ k \in \mathbb{N} \mid \sigma^k = \text{id} \}
\]
For \(\sigma\) built of disjoint cycles of length \(c_1,c_2,\dots,c_k\), its order satisfies:
\[\text{ord}(\sigma)=\text{lcm}(c_1,c_2,\dots,c_k)\]
For example:\\
Consider the permutation \(\sigma = (1\, 2\, 3)\). Applying \(\sigma\) three times returns to the identity permutation:
\[
\sigma = (1\, 2\, 3) \quad \sigma^2 = (1\, 3\, 2) \quad \sigma^3 = \text{id}
\]
Thus, \(\text{ord}(\sigma) = 3\).

\section{Stirling numbers of the first kind}

The Stirling numbers \(\genfrac[]{0pt}{1}{n}{k}\) is the number of permutations in \(S_n\), which have exactly $k$-disjoint cycles.
\[
\genfrac[]{0pt}{0}{n}{k}= \genfrac[]{0pt}{0}{n-1}{k-1} + (n-1)\cdot \genfrac[]{0pt}{0}{n-1}{k}.
\]
with the initial conditions:
\[
\genfrac[]{0pt}{0}{0}{0} = 1 \quad \text{and} \quad \genfrac[]{0pt}{0}{n}{0} = 0 \quad \text{for} \quad n > 0.
\]
and some interesting features:
\[
\genfrac[]{0pt}{0}{n}{1} = (n-1)! \quad \text{ and } \quad \genfrac[]{0pt}{0}{n}{n} = 1 \quad \text{ and } \quad \genfrac[]{0pt}{0}{n}{n-1} = \binom{n}{2}
\]
the following is also true:
\[
\genfrac[]{0pt}{0}{n}{2} = (n-1)!\cdot H_{n-1}
\]

\subsection{Properties}

\begin{enumerate}
    \item Factorial correlation
    \[
    \sum_{k=0}^n \genfrac[]{0pt}{0}{n}{k} = n!.
    \]
    \item Stirling relation
    \[
    \genfrac[]{0pt}{0}{n}{k} \geq \genfrac\{\}{0pt}{0}{n}{k}
    \]
    \item Relation for $x^{\underline{n}}$:
    \[
    x^{\underline{n}} = \sum_{k=0}^{n} (-1)^{k+n} \genfrac[]{0pt}{0}{n}{k} x^k
    \]
    \item Harmonic relation
    \[
    n! H_{n} = \sum_{k=0}^{n} \genfrac[]{0pt}{0}{n}{k} k
    \]
    \item Weird Pascal recurrence
    \[
    \genfrac[]{0pt}{0}{n+m+1}{n} = \sum_{k=0}^{m} (n+k) \genfrac[]{0pt}{0}{n+k}{k}
    \]
    \item Another sum
    \[
    \genfrac[]{0pt}{0}{n+1}{m+1} = \sum_{k=m}^{n} \binom{n}{k}\genfrac[]{0pt}{0}{k}{m} (n-k)!
    \]
\end{enumerate}

\section{Fibonacci Numbers}

\subsection{Definition}

The Fibonacci sequence $(F_n)$ is defined as follows:
\begin{equation}
F_0 = 0, \quad F_1 = 1
\end{equation}
\begin{equation}
F_n = F_{n-1} + F_{n-2} \quad \text{for} \quad n \geq 2
\end{equation}

\subsection{Closed Form (Binet's Formula)}

The $n$-th Fibonacci number can be expressed in closed form using Binet's formula:
\begin{equation}
F_n = \frac{\varphi^n - \psi^n}{\sqrt{5}}
\end{equation}
where $\varphi = \frac{1 + \sqrt{5}}{2}$ (the golden ratio) and $\psi = \frac{1 - \sqrt{5}}{2}$.

\subsection{Matrix Representation}

Fibonacci numbers can also be represented using matrices:
\[
Q=\begin{bmatrix}
F_2 & F_1 \\
F_1 & F_0
\end{bmatrix}
= \begin{bmatrix}
1 & 1 \\
1 & 0
\end{bmatrix}
\]
Then:
\[
Q^n=\begin{bmatrix}
F_{n+1} & F_n \\
F_n & F_{n-1}
\end{bmatrix}
\]

\section{Catalan Numbers}

The $n$-th Catalan number $C_n$ is the number of ways to triangulate a convex polygon with $n+2$ sides. $C_n$ can be defined using the binomial coefficients:
\begin{equation}
C_n = \frac{1}{n+1} \binom{2n}{n} = \binom{2n}{n} - \binom{2n}{n+1}
\end{equation}
It can also be defined recursively as:
\begin{equation}
C_0 = 1
\end{equation}
\begin{equation}
C_{n+1} = \sum_{k=0}^{n} C_k C_{n-k} \quad \text{for} \quad n \geq 0
\end{equation}
\subsection{Asymptotic growth}

\[
c_n = \frac{1}{n+1} \cdot \binom{2n}{n} \approx \frac{1}{n} \frac{4^n}{\pi n} \text{ (Stirling approx.)}
\]


\subsection{Alternate definitions}

\begin{enumerate}
    \item The number of ways to correctly parenthesize a product of $n+1$ factors is the $n$-th Catalan number.
    \item The number of distinct binary trees with $n+1$ leaves (or $n$ internal nodes) is the $n$-th Catalan number.
    \item The number of mountain up-right, down-right paths of length $2n$ (paths from $(0,0)$ to $(2n,0)$ that do not dip below the $x$-axis) is given by the $n$-th Catalan number.
\end{enumerate}

\section{Generating Functions}

A generating function for a sequence \( \{a_n\}_{n=0}^{\infty} \) is a formal power series of the form:
\[ A(x) = \sum_{n=0}^{\infty} a_n x^n \]
The coefficients \( a_n \) represent terms of the sequence.

\subsection{Geometric series}

The geometric series for $a_n=a_0\cdot q^n$ is defined as:
\[ A(x) = a_0 \cdot \sum_{n=1}^{\infty} (qx)^{n} = \frac{a_0}{1-qx}\]

\subsection{Exponential Generating Functions}

The Taylor series for $e^x$ is defined as:
\[ e^{x} = \sum_{n=0}^{\infty} \frac{x^n}{n!} \]

\subsection{Generating function for the Fibonacci sequence}

Let \( \{F_n\} \) denote the Fibonacci sequence defined by \( F_0 = 0, F_1 = 1 \), and \( F_{n} = F_{n-1} + F_{n-2} \) for \( n \geq 2 \). The generating function for the Fibonacci sequence is:
\[ F(x) = \sum_{n=0}^{\infty} F_n x^n = \frac{x}{1 - x - x^2} \]

\subsection{Generating function for binomial coefficient}

The generating function for the binomial coefficient \( \binom{n}{k} \) is:
\[ (1+x)^n = \sum_{k=0}^{n} \binom{n}{k} x^k \]

\subsection{Generating function for n}

Use derivation to find the generating function for the coefficient \( n \) is.
\[ \sum_{n=0}^{\infty} n x^n = \frac{x}{1-x^2} \]
\[ \sum_{n\geq 0} \binom{n+k}{k} x^n = \frac{1}{(1-x)^{k+1}}\]

\subsection{Generating function 1/(x+1)}

The generating function \( \frac{1}{1+x} \) is the sum:
\[ \frac{1}{1+x} = \sum_{k=0}^{n} (-1)^k x^k \]

\subsection{Identities}

\begin{enumerate}
    \item $A(x)+B(x)$ is the generating function for $c_n = a_n+b_n$
    \item $cA(x)$ is the generating function for $c_n = c\cdot a_n$
    \item $A(x)B(x)$ is the generating function for $c_n = \sum_{k=0}^{n} a_k b_{n-k}$ (convolution)
    \item $A'(x)$ is the generating function for $c_n = (n+1) a_{n+1}$
    \item $\frac{A(x)-a_0}{x}$ is the generating function for $c_n = a_{n+1}$
\end{enumerate}

\section{Counting functions}

\subsection{Number of functions}

\begin{enumerate}
    \item Number of functions \(|f:[k]\rightarrow[n]| = n^k\) 
    \item Number of 1-1 functions \(|f_{1-1}:[k]\rightarrow[n]| = n^{\underline{k}}\)
    \item Number of surjective functions \(|f_{surj.}:[k]\rightarrow[n]| = \sum_{i=0}^{n} \binom{n}{i} (n-i) (-1)^{i} = k! \cdot \genfrac\{\}{0pt}{1}{n}{k}\)
    \item Number of growing functions \(|f_{grow.}:[k]\rightarrow[n]| = \binom{n}{k}\)
\end{enumerate}

\subsection{Solutions to x_1 + x_2 + ... + x_k = n}

For when $x_i\geq 1$. We can write $x' = x_i -1$, then $x' \in \{0,1,2,..\}$. But now:
\[
x_1' + x_2' + \dots + x_k' = n-k
\]
There are:
\[
\binom{k+(n-k-1)}{k-1} = \binom{n-1}{k-1}
\]
unique solutions to this equation.

\subsection{Expansion coefficient}

Coefficient for $a^{k_1} b^{k_2} \dots $ in the expansion of $(a+b+c+\dots)^n$
\[
\binom{n}{k_1, k_2, \dots, k_m} = \frac{n!}{k_1! k_2! \dots k_m!}
\]

\section{Helpful integrals}

\begin{enumerate}
    \item \(ln(x)\) \[\int \ln(x) dx = x\ln(x) - x + C\]
\end{enumerate}

\end{document}

