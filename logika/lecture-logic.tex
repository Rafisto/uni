\documentclass{article}

\usepackage[polish]{babel}
\usepackage[utf8]{inputenc}
\usepackage{polski}
\usepackage[T1]{fontenc}
 
\usepackage[margin=1.5in]{geometry} 

\usepackage{color} 
\usepackage{amsmath}                                                                    
\usepackage{amsfonts}                                                                   
\usepackage{graphicx}                                                             
\usepackage{booktabs}
\usepackage{amsthm}
\usepackage{pdfpages}
\usepackage{wrapfig}

\theoremstyle{definition}
\newtheorem{de}{Definicja}[subsection]

\theoremstyle{definition}
\newtheorem{tw}{Twierdzenie}[subsection]

\theoremstyle{definition}
\newtheorem{pk}{Przykład}[subsection]

\theoremstyle{definition}
\newtheorem*{fakt}{FAKT}

\author{Rafal Wlodarczyk}
\title{Logika i struktury formalne}  
\date{INA 1 Sem. 2023}

\begin{document}

\maketitle

\section{Rachunek zdań - logika bez kwantyfikatorów}
\begin{center}
$(p\land q)\implies (r\lor \lnot q)$
\end{center}

\subsection{Konstrukcja języka rachunku zdań}
\begin{itemize}
    \item Mamy ustaloną rodzinę \underline{zmiennych zdaniowych}.\\
    $P=\{p,q,r,s\}$ lub $P=\{p_1,p_2,p_3,\dots\}$
    \item Mamy ustaloną rodzinę \underline{spójników logicznych}.\\
    $\neg,\land,\lor,\implies,\iff$
    \item Mamy $( , )$ - nawiasy
    \item Mamy symbole $\top, \bot$ - prawda, fałsz
    \item Konstrukcja języka $\mathcal{L}(\mathcal{P})$
\end{itemize}

\begin{enumerate}
    \item Zmienne zdaniowe oraz symbole $\top, \bot$ są zdaniami (języka predykatów $\mathcal{L}(\mathcal{P})$)
    \item Jeśli $\varphi, \psi$ są zdaniami, to również napisy $\neg \varphi, (\varphi \land \psi), (\varphi \lor \psi), (\varphi \implies \psi), (\varphi \iff \psi)$ są zdaniami.
    \item Wyrażenie $\varphi$ nazywamy zdaniem jeśli w skończonej liczbie kroków może być skonstruowane za pomocą reguł (1) i (2)
\end{enumerate}

\begin{pk}
    Niech $P={p,q,r}$. Przykłady zdań w $\mathcal{L}(\mathcal{P})$:
    \begin{itemize}
        \item $p;q;r; \top$; $\bot$
        \item $(p\land \top), (p\lor q), (p\implies \top)$
        \item $(r\land (p\lor q)), ((p\lor q)\lor (p\implies \top))$
    \end{itemize}
\end{pk}

\begin{pk}
Rozważmy następujące działanie: $x=(10\cdot 8)/(7\cdot 3)$. Skąpilowane C zwraca $3$.
\end{pk}

\begin{de}
Jeśli $\varphi$ jest z $\mathcal{L}(\mathcal{P})$, to wtedy $\varphi$ ma parzystą liczbę nawiasów.
\begin{proof}
Niech $X$ oznacza kolekcje napisów o parzystej liczbie nawiasów.
\begin{enumerate}
    \item zmienne zdaniowe - 0 nawiasów, $\top, \bot$
    \item załóżmy, że $\varphi, \psi$ są w $X$. Wtedy\\
    $(\varphi \land \psi), ... (\varphi \iff \psi)$ są w $X$.
\end{enumerate}
\end{proof}
\end{de}

\subsection{Zadanie}
Naucz się alfabetu greckiego.

SYNTAKTYKA - badanie wyrażeń.

SEMANTYKA - badanie wartości.

\subsection{Wartości logiczne}
\begin{itemize}
    \item Wartości logiczne: $0, 1$ - fałsz, prawda
    \item Funktory logiczne: $\neg, \land, \lor, \implies, \iff$
    \item Tablice prawdy:\\
\begin{tabular}{|c|c|c|c|c|c|}
    \hline
    X & Y & $\neg X$ & $X\land Y$ & $X\lor Y$ & $X\implies Y$\\
    \hline
    0 & 0 & 1 & 0 & 0 & 1\\
    0 & 1 & 1 & 0 & 1 & 1\\
    1 & 0 & 0 & 0 & 1 & 0\\
    1 & 1 & 0 & 1 & 1 & 1\\
    \hline
\end{tabular}
\end{itemize}

\begin{de}
    Waluacją nazywamy dowolne przyporządowanie $\pi$, które zmiennym zdamiowym przyporządkowuje wartości $0,1$.
\end{de}

\begin{pk}
    Rozważmy następujący przykład waluacji $\pi$:
    \begin{align*}
        P=\{&p_0,p_1,p_2,...\}&\\
        \pi=\{&0,0,0,...\}&\\
        p=\{&0,1,0,1,...\}&
    \end{align*}
    $val(\pi: \text{waluacja}, \varphi: \text{zdanie})$\\
    LOGICAL $0 \lor 1$
\end{pk}

\begin{pk}
Dla $\varphi \in \mathcal{L}(\mathcal{P})$:
\begin{itemize}
    \item $val(\pi, p_i)=\pi(p_i)$
    \item $val(\pi, \top)=1$
    \item $val(\pi, \bot)=0$
    \item $val(\pi, (\varphi \land \psi))=val(\pi, \varphi)\land val(\pi, \psi)$
    \item $val(\pi, \neg \varphi)=\neg val(\pi, \varphi)$
\end{itemize}
\end{pk}

\begin{de}
    $\varphi$ jest tautologią, jeśli dla dowolnej waluacji $\pi$ mamy $val(\pi, \varphi)=1$.
    \\\\($\models \varphi$) - Zapis
\end{de}

\end{document}