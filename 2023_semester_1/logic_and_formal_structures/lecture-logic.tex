\documentclass{article}

\usepackage[polish]{babel}
\usepackage[utf8]{inputenc}
\usepackage{polski}
\usepackage[T1]{fontenc}
 
\usepackage[margin=1.5in]{geometry} 

\usepackage{color} 
\usepackage{amsmath}                                                                    
\usepackage{amsfonts}                                                                   
\usepackage{graphicx}                                                             
\usepackage{booktabs}
\usepackage{amsthm}
\usepackage{pdfpages}
\usepackage{wrapfig}

\theoremstyle{definition}
\newtheorem{de}{Definicja}[subsection]

\theoremstyle{definition}
\newtheorem{tw}{Twierdzenie}[subsection]

\theoremstyle{definition}
\newtheorem{pk}{Przykład}[subsection]

\theoremstyle{definition}
\newtheorem*{fakt}{FAKT}

\author{Rafal Wlodarczyk}
\title{Logika i struktury formalne}  
\date{INA 1 Sem. 2023}

\begin{document}

\maketitle

\section{Rachunek zdań - logika bez kwantyfikatorów}
\begin{center}
$(p\land q)\implies (r\lor \lnot q)$
\end{center}

\subsection{Konstrukcja języka rachunku zdań}
\begin{itemize}
    \item Mamy ustaloną rodzinę \underline{zmiennych zdaniowych}.\\
    $P=\{p,q,r,s\}$ lub $P=\{p_1,p_2,p_3,\dots\}$
    \item Mamy ustaloną rodzinę \underline{spójników logicznych}.\\
    $\neg,\land,\lor,\implies,\iff$
    \item Mamy $( , )$ - nawiasy
    \item Mamy symbole $\top, \bot$ - prawda, fałsz
    \item Konstrukcja języka $\mathcal{L}(\mathcal{P})$
\end{itemize}

\begin{enumerate}
    \item Zmienne zdaniowe oraz symbole $\top, \bot$ są zdaniami (języka predykatów $\mathcal{L}(\mathcal{P})$)
    \item Jeśli $\varphi, \psi$ są zdaniami, to również napisy $\neg \varphi, (\varphi \land \psi), (\varphi \lor \psi), (\varphi \implies \psi), (\varphi \iff \psi)$ są zdaniami.
    \item Wyrażenie $\varphi$ nazywamy zdaniem jeśli w skończonej liczbie kroków może być skonstruowane za pomocą reguł (1) i (2)
\end{enumerate}

\begin{pk}
    Niech $P={p,q,r}$. Przykłady zdań w $\mathcal{L}(\mathcal{P})$:
    \begin{itemize}
        \item $p;q;r; \top$; $\bot$
        \item $(p\land \top), (p\lor q), (p\implies \top)$
        \item $(r\land (p\lor q)), ((p\lor q)\lor (p\implies \top))$
    \end{itemize}
\end{pk}

\begin{pk}
Rozważmy następujące działanie: $x=(10\cdot 8)/(7\cdot 3)$. Skąpilowane C zwraca $3$.
\end{pk}

\begin{de}
Jeśli $\varphi$ jest z $\mathcal{L}(\mathcal{P})$, to wtedy $\varphi$ ma parzystą liczbę nawiasów.
\begin{proof}
Niech $X$ oznacza kolekcje napisów o parzystej liczbie nawiasów.
\begin{enumerate}
    \item zmienne zdaniowe - 0 nawiasów, $\top, \bot$
    \item załóżmy, że $\varphi, \psi$ są w $X$. Wtedy\\
    $(\varphi \land \psi), ... (\varphi \iff \psi)$ są w $X$.
\end{enumerate}
\end{proof}
\end{de}

\subsection{Zadanie}
Naucz się alfabetu greckiego.

SYNTAKTYKA - badanie wyrażeń.

SEMANTYKA - badanie wartości.

\subsection{Wartości logiczne}
\begin{itemize}
    \item Wartości logiczne: $0, 1$ - fałsz, prawda
    \item Funktory logiczne: $\neg, \land, \lor, \implies, \iff$
    \item Tablice prawdy:\\
\begin{tabular}{|c|c|c|c|c|c|}
    \hline
    X & Y & $\neg X$ & $X\land Y$ & $X\lor Y$ & $X\implies Y$\\
    \hline
    0 & 0 & 1 & 0 & 0 & 1\\
    0 & 1 & 1 & 0 & 1 & 1\\
    1 & 0 & 0 & 0 & 1 & 0\\
    1 & 1 & 0 & 1 & 1 & 1\\
    \hline
\end{tabular}
\end{itemize}

\begin{de}
    Waluacją nazywamy dowolne przyporządowanie $\pi$, które zmiennym zdamiowym przyporządkowuje wartości $0,1$.
\end{de}

\begin{pk}
    Rozważmy następujący przykład waluacji $\pi$:
    \begin{align*}
        P=\{&p_0,p_1,p_2,...\}&\\
        \pi=\{&0,0,0,...\}&\\
        p=\{&0,1,0,1,...\}&
    \end{align*}
    $val(\pi: \text{waluacja}, \varphi: \text{zdanie})$\\
    LOGICAL $0 \lor 1$
\end{pk}

\begin{pk}
Dla $\varphi \in \mathcal{L}(\mathcal{P})$:
\begin{itemize}
    \item $val(\pi, p_i)=\pi(p_i)$
    \item $val(\pi, \top)=1$
    \item $val(\pi, \bot)=0$
    \item $val(\pi, (\varphi \land \psi))=val(\pi, \varphi)\land val(\pi, \psi)$
    \item $val(\pi, \neg \varphi)=\neg val(\pi, \varphi)$
\end{itemize}
\end{pk}

\begin{de}
    $\varphi$ jest tautologią, jeśli dla dowolnej waluacji $\pi$ mamy $val(\pi, \varphi)=1$.
    \\\\($\models \varphi$) - Zapis
\end{de}

\section{Wykład drugi}
... tbd

\section{Wykład trzeci}
... tbd

\section{Wykład czwarty}

\subsection{Własności implikacji}
$\models ((p\implies q)\land (q\implies r)) \implies (p\implies r)$ - przechodniość implikacji\\
Weźmy dowolną waluację $\pi$ oraz załóżmy, że $val(\pi,\varphi)=0$, wtedy:
\begin{align*}
val(&\pi,((p\implies q)\land (q\implies r)) \implies (p\implies r))=1\\
val(&\pi,p\implies r)=0\\
\end{align*}
Wtedy:
\begin{itemize}
    \item $\pi(p)=1, \pi(r)=0$
    \item $val(\pi, p\implies q)=1$ oraz $val(\pi, q\implies r)=1$
    \item $\pi(q)=1$
    \item $\pi(r)=1$
\end{itemize}
Otrzymaliśmy sprzeczność, zatem tautologia zachodzi.

\paragraph{Własności}
\begin{enumerate}
    \item $\models (p_1\implies p_2)\land(p_2\implies p_3)\land (p_3\implies p_4)\implies (p_1\implies p_4)$
    \item $\models (p\implies q) \iff (\neg q\implies \neg p)$ <- Rozumowanie nie wprost\\
    D-d. $(\neg q)\implies (\neg p) \equiv  (\neg(\neg q)\lor (\neg p) \equiv (q\lor \neg p))$\\
    $\equiv (\neg p\lor q)\equiv(p\implies q)$
    \item $\models ((p\implies q)\land (q\implies p) \iff (p\iff q))$, czyli\\
    $((p\implies q)\land (q\implies p) \equiv (p\iff q))$
\end{enumerate}

\subsection{Rozumowania matematyczne}
\begin{enumerate}
    \item Rozumowania wprost.\\
    P. Jeżeli $a,b\in \mathbb{Z}$ są parzyste to $a+b$ jest parzyste.\\
    D-d. Załóżmy, że $a=2k\land b=2l$ dla pewnych $k,l\in \mathbb{Z}$\\
    $a+b=2k+2l=2\cdot(k+l)$, więc $a+b$ jest parzyste, bo $k+l\in \mathbb{Z} \qed$
    \item Rozumowania nie wprost. $(p\implies q) \iff (\neg q\implies \neg p)$ \\
    P. Jeśli $\frac{x+y}{2}\geq a$, to $x\geq a \lor y\geq a$\\
    D-d.  $\frac{x+y}{2}\geq a \implies x\geq a \lor y\geq a$\\
    Mamy $\neg(x\geq a \lor y\geq a)\equiv (x\leq a\land y\leq a)$\\
    $x< a \land y< a$\\
    $x+y< 2a\equiv \frac{x+y}{2}< a \equiv \neg(\frac{x+y}{2}\geq a)\qed$\\
    (Korzystając z własności $\neg(x\geq a)\equiv(x< a)$).\\\\
    P. Pracujemy w $\mathbb{R}$. Jeśli $x\in \mathbb{R}^{+}, a,b\in \mathbb{R}^{+}$ oraz $a\cdot b=x$\\
    to $a\leq \sqrt{x}\lor b\leq \sqrt{x}$.\\
    D-d. $\neg(a\leq \sqrt{x}\lor b\leq \sqrt{x}) \equiv (a>\sqrt{x}\land b>\sqrt{x})$\\
    $a\cdot b > \sqrt{x}\cdot \sqrt{x} = x$\\
    $a\cdot b \neq x\qed$\\\\
    P. Jeśli liczba $p\in \mathbb{N}$ nie jest pierwsza, to istnieje $d\leq\sqrt{p}$, taka że $d|p$.
    \item Rozumowanie przez rozważenie przypadków. $\models ((p\implies q)\land(\neg p\implies q) \implies q)$\\
    P. Dla dowolnych liczb rzeczywistych zachodzi $|x+y|\leq |x| + |y|$ $(\equiv q)$\\
    \begin{enumerate}
        \item $p="x+y\geq 0" \implies |x+y|=x+y\leq |x|+|y|$
        \item $p="x+y<0" \implies |x+y|=-x-y=|-x|+|-y|\leq |x|+|y|$ 
    \end{enumerate}
    Zachodzą wszystkie przypadki zatem teza jest prawdziwa. $\qed$
\end{enumerate}

\subsection{Rozważania cykliczne}
\begin{tw}
    Następujące zdania $z_1,z_2,z_3,z_4,z_5$ są równoważne:\\
    $\models (p_1\implies p_2)\land(p_2\implies p_3)\land(p_3\implies p_1)\implies(p_1\iff p_2)\land(p_2\iff p_3)\land(p_3\iff p_1)$\\
    Narysuj kółeczko strzałek z $(p_1,\dots,p_3)$...\\
    Narysuj kółeczko strzałek z $(p_1,\dots,p_4)$...\\
    Korzystamy następnie z przechodniości implikacji.\\
    Oszczędność $n$ - implikacji, zamiast $\frac{n(n-1)}{2}$ implikacji.
\end{tw}

\subsection{Pojęcie dedukcji}
CEL. "Ze zdań $\varphi_1,\dots,\varphi_n$ mogę wywnioskować $\psi$\\
P. Ze zdań $p, p\implies q, q\implies r$ mogę wywnioskować $r$

\begin{de}
    Niech $\varphi_0, \dots, \varphi_n$ będą zdaniami $R.Z$.\\
    \begin{center}
    $\{\varphi_1,\dots, \varphi_n\}\models \psi$
    \end{center}
    Jeśli dla dowolnej waluacji $\pi$, że:
    \begin{center}
    $val(\pi, \varphi_1)=...=val(\pi,\varphi_n)=1$
    \end{center}
    Mamy również:
    \begin{center}
    $val(\pi,\psi)=1$
    \end{center}
\end{de}

\paragraph{Przykłady}
P. Rozważmy następujący przykład:\\
$\{p\}\models p$\\
$\{p\}\models p\lor q$\\
Weźmy waluację $\pi$ taką, że $val(\pi,p)=\pi(p)=1$, wtedy:\\
$val(\pi,p\lor q)=val(\pi,p)\lor val(\pi,q)=1\lor val(\pi,q)=1$\\
P. $\{p\land q\}\models p$\\
Bierzemy $\pi$ taką, że $val(\pi, p\land q)=1$, wtedy:\\
$1=val(\pi,p)\land val(\pi, q)$, więc $val(\pi, p)=1$

\begin{tw}
$\{\varphi_1,\dots,\varphi_n\}\models \psi$, czyli
$\models (\land_{l=1}^{n} \varphi_i) \implies \psi$\\
D-d. Załóżmy, że $\{\varphi_1,\dots,\varphi_n\}\models \psi$. Weźmy dowolną waluację $\pi$
\begin{enumerate}
    \item $val(\pi, (\land_{l=1}^{n} \varphi_i) )=0$, wtedy:\\
    $val(\pi, (\land_{l=1}^{n} \varphi_i \implies \psi)) = val(\pi, (\land_{l=1}^{n} \varphi_i)) \implies val(\pi, \psi) = 0 \implies val(\pi, \psi) = 1$
    \item Załóżmy, że $val(\pi, (\land_{l=1}^{n} \varphi_i) )=1$, wtedy:\\
    $val(\pi, \varphi_1)=...=val(\pi,\varphi_n)=1$, ale:\\
    $\{\varphi_1,\dots,\varphi_n\}\models \psi$, więc $val(\pi,\psi)=1$
\end{enumerate}
2 do 1
Załóżmy, że $(\land_{l=1}^{n} \varphi_i)\implies \psi$\\
Rozważmy dowolną $\pi$ taką, że $val(\pi,\varphi_1)=...=val(\pi,\varphi_n)=1$\\
Wtedy $(\land_{l=1}^{n} \varphi_i)=1$\\
Ale $1=val((\land_{l=1}^{n} \varphi_i)\implies \psi)=val(\pi, \land_{l=1}^{n} \varphi_i) \implies val(\pi,\psi)$\\
Zatem $val(\pi,\psi)=1\qed$
\end{tw}

\paragraph{Semantyczna dedukcja} $\models$

\begin{de}
    $\{\varphi_1,\dots,\varphi_n\}$ jest sprzeczny jak:\\
    $\{\varphi_1,\dots,\varphi_n\}\models \bot$
\end{de}
Wnioski:\\
P. $\{p, \neg p\}\models \bot (\equiv \models(p\land \neg p\implies \bot))$\\
W. $\{\varphi_1,\dots,\varphi_n\}$ - sprzeczny. Wtedy dla dowolnego zdania $\psi$ mamy:\\
$\{\varphi_1,\dots,\varphi_n\}\models \psi$

\paragraph{Uwaga}
Założmy, że $\{\varphi_1, \dots, \varphi_n\}\models \alpha$\\
Wtedy dla dowolnego $\psi$ następujące wzory są równoważne:
\begin{enumerate}
    \item $\{\varphi_1,\dots,\varphi_n\}\models \psi$
    \item $\{\varphi_1,\dots,\varphi_n,\alpha\}\models \psi$
    \item $(1)\implies(2)$ - trywialne
    \item $(2)\implies(1)$ 
\end{enumerate}

\begin{pk}
    Reguła rezolucji:\\
    $\{\psi\lor\alpha,\neg\psi\lor\beta\} \models \alpha\lor\beta$
    równoważnie: $\frac{\{\psi\lor\alpha,\neg\psi\lor\beta\}}{\alpha\lor\beta}$\\
    D-d: Załóżmy waluację $\pi$ taką, że $val(\pi, \varphi\lor\alpha)=val(\pi,\neg\varphi\lor\beta)=1$
    \begin{enumerate}
        \item $val(\pi,\varphi)=1$, wtedy: \\
        $1=val(\pi,\neg\varphi\lor\beta)=val(\pi,\neg\varphi)\lor val(\pi,\beta)=0\lor val(\pi,\beta)=val(\pi,\beta)$
        \item $val(\pi, \varphi)=0$, wtedy: \\
        $1=val(\pi,\varphi\lor\alpha)=val(\pi,\varphi)\lor val(\pi,\alpha)=0\lor val(\pi,\alpha)=val(\pi, \alpha)$\\
        $val(\alpha\lor\beta)=1\qed$
    \end{enumerate}
\end{pk}

\begin{pk}
    Dowodzenie za pomocą metody rezolucji:\\
    $\{p\implies q, q\implies r, r\implies s\}\models p\implies s$\\
    Przekształćmy zdania $\{(\neg p \lor q),(\neg q \lor r),(\neg r \lor s)\}$
    \begin{enumerate}
        \item $\neg p \lor q, \neg q \lor r \implies \neg p\lor r$
        \item $\neg p \lor r, \neg r \lor s \implies \neg p\lor s \equiv p\implies s$
    \end{enumerate}
\end{pk}

\end{document}