\documentclass[12pt]{article}
\usepackage[margin=1in]{geometry}

\usepackage[T1]{fontenc}
\usepackage[polish]{babel}

\usepackage{amsmath}
\usepackage{amsthm}
\usepackage{amsfonts}
\usepackage{hyperref}
\usepackage{graphicx}

\usepackage{algorithm}
\usepackage{algpseudocode}

\makeatletter
\@addtoreset{equation}{subsubsection} 
\makeatother

\title{Alogrytmy Optymalizacji Dyskretnej\\
    \large Lista 2}
\author{Rafał Włodarczyk 279762}
\date{2025-11-05}

\begin{document}

\maketitle

\tableofcontents

\section{Zadanie 1}

\subsection{Opis problemu}

Definiujemy $S$ - zbiór dostawców, z którego każdy dostawca może sprzedać maksymalnie $s_i$ paliwa,
oraz $A$ - zbiór odbiorców, do którego należy dostarczyć co najmniej $a_j$ paliwa. Definiujemy
macierz kosztu, gdzie $c_{ij}$ oznacza koszt galonu paliwa transportowanego od dostawcy $i$ do lotniska $j$.

\subsection{Definicje zmiennych decyzyjnych}

Zdefiniujmy następującą zmienną decyzyjną:

\begin{enumerate}
    \item $x_{i,j}$ - paliwo dostarczane z $i$-tego dostawcy do $j$-tego lotniska.
\end{enumerate}icje zmiennych decyzyjnych (opis, jednostki),
(b) ograniczenia (nie umieszczaj ´zródeł modelu),
(c) funkcja celu,
2. krótki opis rozwi ˛az

\subsection{Ograniczenia}

Zdefiniujmy następujące ograniczenia:

\begin{enumerate}
    \item $s_i \geq 0$ - każdy dostawca dostarcza paliwo
    \item $a_j \geq 0$ - każde lotnisko potrzebuje paliwa
    \item $c_{ij} \geq 0$ - koszt paliwa i transportu jest zawsze większy od zera.
    \item $\sum_j x_{ij} \leq s_i$ - od każdego dostawcy dostarczyć możemy maksymalnie $s_i$ paliwa.
    \item $\sum_i x_{ij} \geq a_j$ - do każdego lotniska dostarczyć musimy co najmniej $a_j$ paliwa.
\end{enumerate}

\subsection{Funkcja Celu}

Zdefiniujmy następującą funkcje celu:

\begin{align*}
    \text{minimize } \sum_{i\in S, j\in A} c_{ij} \cdot x_{ij}
\end{align*}

\subsection{Egzemplarz problemu}

Egzemplarz problemu definiują następujące parametry - trzech dostawców i cztery lotniska:
\begin{align*}
    S &= \{s_1, s_2, s_3\} \\
    A &= \{a_1, a_2, a_3, a_4\}
\end{align*}
Następujące dostępności $s_i$:
\begin{align*}
    s_1 = 275{,}000 \quad
    s_2 = 550{,}000 \quad
    s_3 = 660{,}000
\end{align*}
Następujące zapotrzebowania $a_j$:
\begin{align*}
    a_1 = 110{,}000 \quad
    a_2 = 220{,}000 \quad
    a_3 = 330{,}000 \quad
    a_4 = 440{,}000
\end{align*}
Oraz następującą macierz kosztu $c_{ij}$:
\begin{align*}
    \text{cost} &: 
    \begin{pmatrix}
        & s_1 & s_2 & s_3 \\
        a_1 & 10 & 7 & 8 \\
        a_2 & 10 & 11 & 14 \\
        a_3 & 9 & 12 & 4 \\
        a_4 & 11 & 13 & 9
    \end{pmatrix}
\end{align*}

\subsection{Uzyskane Wyniki}

Model dla zadanego egzemplarza prezentuje następujące wyniki:

\begin{verbatim}
$ glpsol --model fuel.mod --data fuel.data  
Time used:   0.0 secs
Memory used: 0.1 Mb (128314 bytes)

Optimal shipment quantities:
x[s1,a1] =     0.00
x[s1,a2] = 165000.00
x[s1,a3] =     0.00
x[s1,a4] = 110000.00
x[s2,a1] = 110000.00
x[s2,a2] = 55000.00
x[s2,a3] =     0.00
x[s2,a4] =     0.00
x[s3,a1] =     0.00
x[s3,a2] =     0.00
x[s3,a3] = 330000.00
x[s3,a4] = 330000.00

Total shipped per supplier:
s1: 275000.00
s2: 165000.00
s3: 660000.00

Total cost: 8525000
Model has been successfully processed
\end{verbatim}

\subsection{Interpretacja Wyników}

Łączny koszt dostaw wyniósł $8{,}525{,}000$ dolarów. Dostawcy dostarczyli następującą ilość galonów paliwa:
$x_{s_1}=275{,}000; x_{s_2} = 165{,}000; x_{s_3} = 660{,}000$. Wszystkie lotniska uzyskały wystarczającą ilość paliwa do swojej działalności.
Na tej podstawie można odpowiedzieć na pytania postawione w poleceniu:

\begin{enumerate}
    \item Minimalny łączny koszt dostaw wynosi $8{,}525{,}000$ dolarów.
    \item Tak, wszystkie firmy dostarczają paliwo.
    \item Możliwości dostaw paliwa są wyczerpane u pierwszego i trzeciego dostawcy. 
        Drugi dostawca może jeszcze dostarczyć $385{,}000$ galonów.
\end{enumerate}

\section{Zadanie 2}

\subsection{Opis problemu}

Mamy maszyny $M$ i produkty $P$. Każdy produkt musi przejść przez określoną liczbę minut na każdej maszynie.
Maszyny nie mogą pracować dłużej niż $T$ godzin każda. Każda maszyna ma koszt stały $c_m$ na godzinę pracy.
Każdy produkt ma cene materiałową $C_p$ i cenę sprzedaży $R_p$. Maksymalny popyt na produkt $p$ to $D_p$.
Definiujemy macierz produkcji $\text{prod}_{p,m}$ jako liczbę minut, które produkt $p$ musi spędzić na maszynie $m$.

\subsection{Definicje zmiennych decyzyjnych}

Zdefiniujmy następujące zmienne decyzyjne:
\begin{enumerate}
    \item $x_p$ - ilość wyprodukowanych jednostek produktu $p$ w kilogramach.
\end{enumerate}

\subsection{Ograniczenia}

Zdefiniujmy następujące ograniczenia:
\begin{enumerate}
    \item $\left(\forall m\in M\right)\left(\sum_{p \in P} \text{prod}_{p,m} \cdot x_p \leq T \cdot 60\right)$ - maksymalny czas pracy każdej maszyny.
    \item $x_p \leq D_p$ - maksymalna produkcja każdego produktu.
\end{enumerate}

\subsection{Funkcja Celu}

Zdefiniujmy następującą funkcję celu:
\begin{align*}
    \text{maximize } \sum_{p \in P} (R_p - C_p) \cdot x_p - \sum_{m \in M} c_m \cdot \frac{1}{60} \left( \sum_{p \in P} \text{prod}_{p,m} \cdot x_p \right)
\end{align*}
Po stronie zysków mamy sumę $R_p\cdot x_p$, czyli zysk jednostkowy razy liczba wyprodukowanych produktów. 
Po stronie kosztów mamy sumę $C_p\cdot x_p$, czyli koszt jednostkowy razy liczba wyprodukowanych produktów oraz
koszt zmienny pracy każdej z maszyn, biorąc pod uwagę produkcje, które na nich wykonaliśmy. 

\subsection{Egzemplarz problemu}

Egzemplarz problemu definiują następujące parametry:
\begin{align*}
M=\{m_1, m_2, m_3\} \quad P=\{p_1, p_2, p_3, p_4\}
\end{align*}

\begin{align*}
D_p &= \{ 400, 100, 150, 500 \} \quad
\text{T} = 60 \text{ godzin} \\
c_m &= 2, 2, 3 \quad
C_p = 4, 1, 1, 1 \quad
R_p = 9, 7, 6, 5\\
\text{prod}_{p,m}&=
\begin{pmatrix}
    & m_1 & m_2 & m_3 \\
    p_1 & 5 & 10 & 6 \\
    p_2 & 3 & 6 & 4 \\
    p_3 & 4 & 5 & 3 \\
    p_4 & 4 & 2 & 1
\end{pmatrix}
\end{align*}

\subsection{Uzyskane Wyniki}

Model dla zadanego egzemplarza prezentuje następujące wyniki:
\begin{verbatim}
glpsol --model prod.mod --data prod.data  
Time used:   0.0 secs
Memory used: 0.1 Mb (134445 bytes)

Optimal production quantities:
Product p1:   125.00
Product p2:   100.00
Product p3:   150.00
Product p4:   500.00

Production cost per product:
Product p1:   500.00
Product p2:   100.00
Product p3:   150.00
Product p4:   500.00

Machine usage and cost:
Machine m1:  3525.00 minutes, cost:   117.50
Machine m2:  3600.00 minutes, cost:   120.00
Machine m3:  2100.00 minutes, cost:   105.00

Total profit: 3632.5
Model has been successfully processed
\end{verbatim}

\subsection{Interpretacja Wyników}

Wyniki interpretuję następująco:
Maksymalny popyt został wypełniony dla drugiego, trzeciego i czwartego produktu. Pierwszy produkt,
ze względu na wysoki koszt produkcji i ograniczony czas pracy maszyn nie został wyprodukowany w maksymalnej ilości.

\noindent
Łączny zysk z produkcji wyniósł $3{,}632.50$ dolarów.


\section{Zadanie 3}

\subsection{Opis Problemu}

Wytwarzamy maksymalnie $C$ jednostek w każdym z $K$ następujących po sobie okresów. W każdym okresie
koszt produkcji wynosi $c_j, j\in\{1,\dots,K\}$. Żeby sprostać wymaganiom rynku dodajemy dodatkową
produkcje $a_j$, przy koszcie jednostkowym $o_j$. Wymaganie rynku (demand) określamy jako $d_j$. Możemy
przechować do $S$ jednostek w magazynie na kolejny okres, płacąc jednak $T$ za każdą jednostkę. 
$S_0\geq 0$ to początkowy stan magazynowy.

\subsection{Definicje zmiennych decyzyjnych}

Zdefiniujmy następujące zmienne decyzyjne:
\begin{enumerate}
    \item $I_j$ - stan magazynowy w okresie $j$.
    \item $P_j$ - produkcja w okresie $j$.
    \item $A_j$ - dodatkowa produkcja w okresie $j$.
\end{enumerate}

\subsection{Ograniczenia}

Zdefiniujmy następujące ograniczenia dla magazynu, definiuje go poprzedni stan magazynu, produkcja i zapotrzebowanie.
\begin{enumerate}
    \item $I_j \leq S$. Nie przekraczamy pojemności magazynu.
    \item $I_1 = S_0 + P_1 + A_1 - d_1$. W pierwszym okresie uwzględniamy stan początkowy magazynu.
    \item $I_j = I_{j-1} + P_{j} + A_{j} - d_j$, $j>1$. Następnie uwzględniamy poprzedni stan magazynu.
\end{enumerate}
Zdefiniujmy następujące ograniczenia dla produkcji:
\begin{enumerate}
    \item $P_j \leq C$, zawsze produkujemy bazowo maks $C$ produktów.
    \item $A_j \leq a_j$, produkujemy dodatkowo maks $a_j$ produktów.
\end{enumerate}

\subsection{Funkcja Celu}

Zdefiniujmy następującą funkcję celu:
\begin{align*}
    \text{minimize } \sum_{j\in 1..K} c_j\cdot P_j + o_j \cdot A_j + \cdot T \cdot I_j 
\end{align*}
Minimalizujemy koszt produkcji standardowej, dodatkowej oraz cene przechowywania w magazynie, po każdym z okresów $K$.

\subsection{Egzemplarz problemu}

Egzemplarz problemu definiują następujące parametry:
\begin{align*}
    K&=1,2,3,4, C=100, S=70, S_0=15, T=1500 \\
    c_j&=(6000,4000,8000,9000)\\
    a_j&=(60,65,70,60)\\
    o_j&=(8000,6000,10000,11000)\\
    d_j&=(130,80,125,195)\\
\end{align*}

\subsection{Uzyskane Wyniki}

Model dla zadanego egzemplarza prezentuje następujące wyniki:
\begin{verbatim}
glpsol --model prod.mod --data prod.data
Time used:   0.0 secs
Memory used: 0.1 Mb (126561 bytes)
Display statement at line 73
production[1].val = 100
production[2].val = 100
production[3].val = 100
production[4].val = 100
additional_production[1].val = 15
additional_production[2].val = 50
additional_production[3].val = 0
additional_production[4].val = 50
inventory[1].val = 0
inventory[2].val = 70
inventory[3].val = 45
inventory[4].val = 0
total_cost.val = 3842500
Model has been successfully processed
\end{verbatim}

\subsection{Interpretacja Wyników}

Najniższy koszt wyniósł $3{,}842{,}500$ dolarów. Każdorazowo produkcja standardowa wynosiła
$100$ - czyli ustalony przez nas limit produkcji standardowej. Następnie aby sprostać wymaganiu
$d_3,d_4$ przechowaliśmy w okresach $2,3$ odpowiednio $70$ i $45$ produktów. Zgada się to z 
intuicją, ponieważ w okresie $2$ obserwujemy najniższy koszt produkcji jednostkowej. Możemy
udzielić odpowiedzi na pytania:
\begin{enumerate}
    \item Minimalny łączny koszt produkci i magazynowania wynosi $3{,}842{,}500$ dolarów.
    \item W okresach $1,2,4$ firma musi zaplanować produkcje ponadwymiarową w ilościach $15,50,50$.
    \item W okresie $2$ możliwości magazynowania towaru są wyczerpane.
\end{enumerate}

\section{Zadanie 4}

\subsection{Opis Problemu}

Weźmy sieć połączeń za pomocą skierowanego grafu $G=(N,A)$, $N$-miasta, $A$-połaczenia między miastami.
Dane są $c_{ij}$ - koszt przejazdu z $i$ do $j$, $i,j\in N$ oraz czas przejazdu $t_{ij}$.
Mamy również dwa miasta $s$, $e$ - miasto startowe oraz miasto końcowe. Celem jest znalezienie
najtańszej ściezki z $s$ do $e$, której sumaryczny czas przejścia nie przekracza $T$.

\subsection{Definicje zmiennych decyzyjnych}

Zdefiniujmy następujące zmienne decyzyjne:
\begin{enumerate}
    \item $R_{ij}$ - macierz krawędzi ($1$-bierzemy krawędź do ścieżki, $0$-w.p.p.)
\end{enumerate}

\subsection{Ograniczenia}

Zdefiniujmy następujące ograniczenia:
\begin{enumerate}
    \item $\sum_{j} R_{sj} = 1$ - wychodzimy raz z miasta startowego.
    \item $\sum_{j} R_{je} = 1$ - wchodzimy raz do miasta końcowego.
    \item $\left(\forall k\in N-\{s,e\}\right)\left(\sum_{i} R_{ik} = \sum_{j} R_{kj}\right)$ - dla każdego pozostałego miasta suma wejść jest równa sumie wyjść.
    \item $\sum_{ij} t_{ij} \cdot R_{ij} \leq T$ - wybrana ścieżka nie przekracza $T$.
\end{enumerate}

\subsection{Funkcja Celu}

Zdefiniujmy następujące funkcje celu:
\begin{align*}
    \text{minimize } \sum_{i,j \in N} c_{ij} \cdot R_{ij}
\end{align*}

\subsection{Podpunkt a}

\subsubsection{Egzemplarz problemu}

Egzemplarz problemu definiują następujące parametry:
\begin{align*}
N &= \{1, 2, 3, 4, 5, 6, 7, 8, 9, 10\}, s = 1, e = 10, T = 15\\
(i,j, c_{ij}, t_{ij}) = \{
&(1, 2, 3, 4),  (1, 3, 4, 9),  (1, 4, 7, 10), (1, 5, 8, 12),\\
&(2, 3, 2, 3),  (3, 4, 4, 6),  (3, 5, 2, 2), (3, 10, 6, 11),\\
&(4, 5, 1, 1),  (4, 7, 3, 5),  (5, 6, 5, 6), (5, 7, 3, 3),\\
&(5, 10, 5, 8), (6, 1, 5, 8),  (6, 7, 2, 2), (6, 10, 7, 11),\\
&(7, 3, 4, 6),  (7, 8, 3, 5),  (7, 9, 1, 1), (8, 9, 1, 2),\\
&(9, 10, 2, 2) \}
\end{align*}

\subsubsection{Uzyskane Wyniki}

Model dla zadanego egzemplarza prezentuje następujące wyniki:
\begin{verbatim}
glpsol --data graph.data --model graph.mod
Time used:   0.0 secs
Memory used: 0.2 Mb (244155 bytes)
1 -> 2
2 -> 3
3 -> 5
5 -> 7
7 -> 9
9 -> 10
Total cost: 13
Total time: 15
Model has been successfully processed
\end{verbatim}

\subsubsection{Interpretacja Wyników}

W zadanym grafie znaleźliśmy ścieżkę o koszcie $c=13$ i czasie $t=15$: $1\rightarrow 2\rightarrow 3\rightarrow 5\rightarrow 7\rightarrow 9\rightarrow 10$.
Jest długa pod względem liczby krawędzi (ignorując koszty można z $1\rightarrow 10$ dojść w dwóch krawędziach), 
ale jest najlepsza z możliwych opcji, biorąc również pod uwagę ograniczenie czasowe.

\subsection{Podpunkt b}

\subsubsection{Egzemplarz problemu}

Wymyśliłem następujący graf - opis krawędzi $(\text{cost},\text{time})$, $s=1, e=10$

\begin{figure}[H]
    \centering
    \begin{minipage}{0.5\textwidth}
        \centering
        \includegraphics[width=\linewidth]{graph4b.png}
        \caption{Graf dla podpunktu b}
        \label{fig:graph4b}
    \end{minipage}
\end{figure}

\subsubsection{Uzyskane Wyniki}

\begin{verbatim}
max_time = 25

Time used:   0.0 secs
Memory used: 0.2 Mb (240874 bytes)
1 -> 3
3 -> 5
5 -> 9
9 -> 10
Total cost: 60
Total time: 20
Model has been successfully processed

max_time = 30

Time used:   0.0 secs
Memory used: 0.2 Mb (240874 bytes)
1 -> 2
2 -> 6
6 -> 10
Total cost: 30
Total time: 30
Model has been successfully processed
\end{verbatim}

\subsubsection{Interpretacja Wyników}

Ścieżka o najniższym koszcie dla czasu maksymalnego $T=25$ i zadanego grafu to
$1\rightarrow3\rightarrow5\rightarrow9\rightarrow10$
  Podnosząc $T=30$ widzimy nową, krótszą ścieżkę
$1\rightarrow 2\rightarrow 6\rightarrow 10$, której koszt jest jednak większy od $25$, co
oznacza że dla mniejszego $T$ jest ona zbyt długa czasowo.

\subsection{Podpunkt c}

Zakładamy niecałkowitoliczbowość w wartościach macierzy decyzyjnej $R_{ij}$. Spróbujmy rozwiązać
następujący przykład $T=3$, $s=1,e=3$.

\begin{figure}[H]
    \centering
    \includegraphics{graph4c.png}
    \caption{Graf dla podpunktu c}
    \label{fig:graph4c}
\end{figure}

\begin{verbatim}
glpsol --model float.mod --data float.data
Time used:   0.0 secs
Memory used: 0.1 Mb (126442 bytes)
    0.50 1 -> 2
    0.50 1 -> 3
    0.50 2 -> 3
Total cost: 4
Total time: 3

glpsol --model graph.mod --data float.data
Time used:   0.0 secs
Memory used: 0.1 Mb (147075 bytes)
1 -> 3
Total cost: 6
Total time: 2
\end{verbatim}

Model niecałkowitoliczbowy zwraca ścieżkę o niższym koszcie $4$, biorąc z dwóch możliwych ścieżek obie, w równych wagach.
Model całkowitoliczbowy zwraca ścieżkę o koszcie $6$. Skoro minimalizujemy koszt i rozwiązanie optymalnie nie posiada
wartości całkowitych, to pokazaliśmy, że ograniczenie na całkowitoliczbowość jest potrzebne. W przeciwnym wypadku otrzymujemy
inny problem z potencjalnie wieloma ścieżkami. 

\subsection{Podpunkt d}

Usuńmy ograniczenie $T$ oraz rozważmy poniższy przykład, $s=1,e=4$. 

\begin{figure}[H]
    \centering
    \includegraphics{graph4d.png}
    \caption{Graf dla podpunktu d}
    \label{fig:graph4d}
\end{figure}

Rozwiązanie tego egzemplarza może obejmować przejście
 $R_1\left(1\rightarrow2\rightarrow4\right), R_2\left(1\rightarrow3\rightarrow4\right)$ dla $R_1+R_2=1$
Ale taka sytuacja nastąpi jedynie, gdy podścieżki mają sumarycznie ten sam koszt, co w zasadzie pozwala
zredukować dwie ścieżki, do jednej - pomimo że technicznie poprawny jest wybór dowolnego $R_1\in[0,1]$ solver
preferuje zwrócić liczbę całkowitą.

\begin{verbatim}
glpsol --model float.mod --data d.data
Time used:   0.0 secs
Memory used: 0.1 Mb (127711 bytes)
    1.00 1 -> 2
    1.00 2 -> 4
Total cost: 3
Total time: 6
\end{verbatim}
Zatem otrzymane połączenie, po ewentualnej korekcie wag, będzie akceptowalnym rozwiązaniem.

\section{Zadanie 5}

\subsection{Opis Problemu}

Mamy komendę policji która chce rozdysponować radiowozy po zmianach i dzielnicach.
Szukamy takiej macierzy $R_{zd}$ gdzie $z\in Z$-zmiana $d\in D$-dzielnica, przy czym nie możemy przekroczyć
limitów radiowozów na zmianę w dzielnicy $max_{zd}$, oraz musimy utrzymać minimalne ilości radiowozów
na zmianę w dzielnicy $min_{zd}$, na zmianę sumarycznie $S_z$ oraz na dzielnice sumarycznie $S_d$

\subsection{Definicje zmiennych decyzyjnych}

Zdefiniujmy następujące zmienne decyzyjne:
\begin{enumerate}
    \item $R_{zd}$ - radiowozy na zmianie $z$ w dzielnicy $d$.
\end{enumerate}

\subsection{Ograniczenia}

Zdefiniujmy następujące ograniczenia:
\begin{enumerate}
    \item $\left(\forall z\in Z \right)\left(\forall d\in D \right) R_{zd} \geq min_{zd}$ - minimalna liczba radiowozów w komórce
    \item $\left(\forall z\in Z \right)\left(\forall d\in D \right) R_{zd} \leq max_{zd}$ - maksymalna liczba radiowozów w komórce
    \item $\left(\forall z\in Z \right) \sum_{z\in Z} R_{zd} \geq S_z$ - sumarycznie więcej niż $S_z$ na zmianę
    \item $\left(\forall d\in D \right) \sum_{d\in D} R_{zd} \geq S_d$ - sumarycznie więcej niż $S_d$ na dzielnicę
\end{enumerate}

\subsection{Funkcja Celu}

Zdefiniujmy następujące funkcje celu:
\begin{align*}
    \text{minimize } & \sum_{z\in Z, d\in D} R_{zd}
\end{align*}
Efektywnie minimalizujemy liczbę pojazdów na drodze.

\subsection{Egzemplarz problemu}

Egzemplarz problemu definiują następujące parametry:
\begin{align*}
    \text{Z} &= \{1, 2, 3\}, \quad \text{D} = \{1, 2, 3\} \\
    S_z &= \{10, 20, 18\}, \quad S_d = \{10, 14, 13\}\\
    min_{zd} &=
    \begin{pmatrix}
        & 1 & 2 & 3 \\
        1 & 2 & 4 & 3 \\
        2 & 3 & 6 & 5 \\
        3 & 5 & 7 & 6
    \end{pmatrix}, \quad
    max_{zd} =
    \begin{pmatrix}
        & 1 & 2 & 3 \\
        1 & 3 & 7 & 5 \\
        2 & 5 & 7 & 10 \\
        3 & 8 & 12 & 10
    \end{pmatrix}
\end{align*}

\subsection{Uzyskane Wyniki}

Model dla zadanego egzemplarza prezentuje następujące wyniki:
\begin{verbatim}
$ glpsol --model police.mod --data police.data 
Time used:   0.0 secs
Memory used: 0.1 Mb (126583 bytes)
Occupancy Matrix:
3 4 3 
5 7 8 
5 7 6 
Display statement at line 58
48
Model has been successfully processed
\end{verbatim}

\subsection{Interpretacja Wyników}

W zasadzie robimy absolutne minimum, z warunku $S_z$ widzimy, że nie możemy zejść poniżej $48$ radiowozów 
i szczęśliwie takie rozstawienie udaje się znaleźć. Jesteśmy niewiele ponad minimalne wymagania $min_{zd}$ w każdej
komórce, a jednak dla każdej zmiany i dzielnicy pokrywamy zadane wymagania.

\section{Zadanie 6}

\subsection{Opis Problemu}

Mamy działkę podzieloną na siatkę $m\times n$, w której to trzymamy kontenery, wyrażone
za pomocą macierzy $K_{ij}$, $i\in M, j\in N$ - zero gdy nie ma kontenera, jeden gdy jest.
Chcemy aby chociaż jedna kamera ze minimalnego zbioru kamer $C_{ij}$ patrzyła na każdy kontener,
przy czym kamera widzi $k$ pól do góry, w dól, w prawo i w lewo, i nie może stać w polu, na
którym jest kontener.

\subsection{Definicje zmiennych decyzyjnych}

Zdefiniujmy następujące zmienne decyzyjne:
\begin{enumerate}
    \item $C_{ij}$ - ustawienie kamer; $C_{ij} = 0$ - nie ma kamery, $C_{ij} = 1$ - jest kamera.
\end{enumerate}

\subsection{Ograniczenia}

Zdefiniujmy następujące ograniczenia:
\begin{enumerate}
    \item Kamery nie mogą być ustawione na polach z kontenerami.
        \begin{align*}
            \forall i \in \{1, \dots, m\}, \forall j \in \{1, \dots, n\}, \text{jeśli } K_{ij} = 1, \text{ to } C_{ij} = 0
        \end{align*}
    \item Co najmniej jedna kamera patrzy na kontener.
        \begin{align*}
            (\forall i,j : K_{ij}=1, d_i,d_j\in\{-k,...-1,1,...,k\}) \sum_{i:K_{i+di,j}=0} C_{i+di,j} +  \sum_{j:K_{i,j+dj}=0} C_{i,j+dj}\ \geq 1
        \end{align*} 
\end{enumerate}

\subsection{Funkcja Celu}

Zdefiniujmy następujące funkcje celu:
\begin{align*}
    \text{minimize } \sum_{(i,j)\in M\times N} C_{ij}
\end{align*}

\subsection{Egzemplarz problemu}

Egzemplarz problemu definiują następujące parametry:
\begin{align*}
m &= 5, \quad n = 5, \quad k_1 = 1 \\
K_{ij} &=
\begin{pmatrix}
    0 & 1 & 0 & 0 & 1 \\
    0 & 0 & 0 & 0 & 0 \\
    0 & 0 & 0 & 0 & 0 \\
    0 & 0 & 0 & 0 & 1 \\
    1 & 0 & 0 & 0 & 0
\end{pmatrix}
\end{align*}
Oraz drugi egzemplarz, w którym $k_2 = 5$, $m,n,K_{ij}$ bez zmian.

\subsection{Uzyskane Wyniki}

Model dla zadanych egzemplarzy prezentuje następujące wyniki:
\begin{verbatim}
k = 1

Time used:   0.0 secs
Memory used: 0.1 Mb (148720 bytes)
Display statement at line 37
total_cameras.val = 4

* x . * x 
. . . . . 
. . . . . 
. . . * x 
x * . . . 

Legend:
x - container
* - camera
. - empty space
Model has been successfully processed

k = 5

Time used:   0.0 secs
Memory used: 0.2 Mb (161455 bytes)
Display statement at line 37
total_cameras.val = 2

. x * . x 
. . . . . 
. . . . . 
. . . . x 
x . . . * 

Legend:
x - container
* - camera
. - empty space
Model has been successfully processed
\end{verbatim}

\subsection{Interpretacja Wyników}

W przypadku $k=1$ potrzebujemy 4 kamery na 4 kontenery, 
a w przypadku $k=5$ na te same kontenery potrzebujemy już tylko dwie kamery. 
Widzimy, że wraz ze wzrostem $k$ pokrycie naszego pola staje się łatwiejsze,
ponieważ pojedyńcza kamera widzi więcej pól. 



\end{document}