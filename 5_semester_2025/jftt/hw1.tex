\documentclass[12pt]{article}
\usepackage[margin=1in]{geometry}

\usepackage[T1]{fontenc}
\usepackage[polish]{babel}

\usepackage{amsmath}
\usepackage{amsthm}
\usepackage{amsfonts}

\makeatletter
\@addtoreset{equation}{subsubsection} 
\makeatother

\title{Ćwiczenia z JFTT}
\author{Rafał Włodarczyk}
\date{2025-10-16}

\begin{document}

\maketitle

\section{Lista 1}

\setcounter{subsection}{4}
\subsection{Zadanie 5}

Udowodnić następujące tożsamości dla wyrażeń regularnych $r$, $s$ i $t$, 
przy czym $r$ = $s$ oznacza identyczność języków opisywanych przez $r$ i $s$.

\subsubsection{$(r+s)+t=r+(s+t)$}

\begin{align}
    &x\in (r+s)+t \equiv&\\
    &\equiv x\in (r+s) \lor x \in t \equiv\\
    &\equiv x\in r \lor x \in s \lor x \in t \equiv\\
    &\equiv x\in r \lor x \in (s+t) \equiv\\
    &\equiv x\in r+(s+t)_{\qed}
\end{align}

\subsubsection{$(rs)t=r(st)$}

\begin{align}
    &x\in (rs)t \equiv&\\
    &\equiv \left(\exists a,b \right)\left(x=ab \land a\in rs \land b\in t\right) \equiv\\
    &\equiv \left(\exists b,c,d \right)\left(x=cdb \land c\in r \land d\in s \land b\in t\right) \equiv\\
    &\equiv \left(\exists c,e \right)\left(x=ce \land c\in r \land e\in st\right) \equiv\\
    &\equiv x\in r(st)_{\qed}
\end{align}

\subsubsection{$r(s+t)=rs+rt$}

\begin{align}
    &x\in r(s+t) \equiv&\\
    &\equiv \left(\exists y,z \right)\left(x=yz \land y\in r \land z\in (s+t)\right) \equiv\\
    &\equiv \left(\exists y,z \right)\left(x=yz \land y\in r \land \left(z\in s \lor z\in t\right)\right) \equiv\\
    &\equiv \left(\exists y,z \right)\left(x=yz \land \left(\left(y\in r \land z\in s\right) \lor \left(y\in r \land z\in t\right)\right)\right) \equiv\\
    &\equiv rs + rt_{\qed}
\end{align}

\subsubsection{$(r+s)t = rt + st$}

\begin{align}
    &x\in (r+s)t \equiv&\\
    &\equiv \left(\exists y,z \right)\left(x=yz \land y\in (r+s) \land z\in t\right) \equiv\\
    &\equiv \left(\exists y,z \right)\left(x=yz \land \left(y\in r \lor y\in s\right) \land z\in t\right) \equiv\\
    &\equiv \left(\exists y,z \right)\left(x=yz \land \left(\left(y\in r \land z\in t\right) \lor \left(y\in s \land z\in t\right)\right)\right) \equiv\\
    &\equiv x\in rt + st_{\qed}
\end{align}

\subsubsection{$\emptyset^{*} = \varepsilon$}

Wiemy, że $\emptyset a = \emptyset$ oraz $a\emptyset = \emptyset$.
Wiemy również, że dla dowolnego języka $L$, w szczególności $L=\emptyset$ zachodzi $L^{0} = \{\varepsilon\}$. Zatem:

\begin{align}
    &x\in \emptyset^{*} \equiv
    x\in \bigcup_{i=0}^{\infty} \emptyset^{i} \equiv
    x\in (\emptyset^{0} + \emptyset^{1} + \emptyset^{2} + \dots) \equiv\\
    &\equiv x \in (\varepsilon + \varepsilon\emptyset + \varepsilon\emptyset\emptyset + \dots) \equiv\\
    &\equiv x \in \varepsilon_{\qed}
\end{align}

\subsubsection{$(r^{*})^{*} = r^{*}$}

Pokażmy, że domknięcie Kleene'ego jest idempotentne. Wyraźmy definicję za pomocą kwantyfikatorów:

\begin{align}
    x\in r^{*} \equiv x\in \bigcup_{i=0}^{\infty} r^i \equiv \left(\exists_n\right) 
    \left(x=x_1\dots x_n \land \left(\forall_{i\in \{1\dots n\}}\right) x_i \in r\right)
\end{align}

\noindent
Skorzystajmy następnie dwukrotnie z tej definicji, aby rozwinąć lewą stronę:

\begin{align}
    &x\in (r^{*})^{*} \equiv \left(\exists_n\right)\left(x=x_1\dots x_n \land \left(\forall_{i\in \{1\dots n\}}\right) x_i \in r^{*}\right) \equiv\\
    &\equiv \left(\exists_n\right)\left(x=x_1\dots x_n \land \left(\forall_{i\in \{1\dots n\}}\right) \left(\exists_{m_i}\right) \left(x_i = x_{i_1}\dots x_{im_i} \land \left(\forall_{j\in \{1\dots m_i\}}\right) x_{ij} \in r\right)\right) \equiv\\
    &\equiv \left(\exists_n\right)\left(\forall_{i\in \{1\dots n\}}\right)\left(\exists_{m_i}\right)\left(\forall_{j\in \{1\dots m_i\}}\right)\left(x=x_{11}\dots x_{1m_1}\dots x_{n1}\dots x_{nm_n} \land x_{ij} \in r\right) \equiv\\
    &\equiv \left(\exists_k\right)\left(x=x_1\dots x_k \land \left(\forall_{i\in \{1\dots k\}}\right)\left(x_i \in r\right)\right)\equiv \\
    &\equiv x\in r^{*}_{\qed}
\end{align}

\subsubsection{$(r^{*}s^{*})^{*} = (r+s)^{*}$}

Zawieranie ($\subseteq$):\\
\noindent
Jeśli $y\in r^{*}$, to na pewno $y\in (r+s)^{*}$, ponieważ $r \subseteq r+s$.\\ Analogicznie $y\in s^{*} \implies y\in (r+s)^{*}$.
\begin{align}
    &x\in (r^{*}s^{*})^{*} \implies x\in ((r+s)^{*}(r+s)^{*})^{*} = ((r+s)^{*})^{*} = (r+s)^{*}
\end{align}
Widzimy zatem, że $(r^{*}s^{*})^{*} \subseteq (r+s)^{*}$.\\

\noindent
Zawieranie ($\supseteq$):\\
\noindent
Jeśli $y\in r$, to na pewno $y\in r^{*}$, ponieważ $r \subseteq r^{*}$. Analogicznie $y\in s \implies y\in s^{*}$.
\begin{align}
    &x\in (r+s)^{*} \implies x\in (r^{*}+s^{*})^{*}
\end{align}
Następnie jeśli $y\in r^{*}$ lub $y\in s^{*}$, to na pewno $y\in r^{*}s^{*}$, ponieważ $r^{*} \subseteq r^{*}s^{*}$ oraz $s^{*} \subseteq r^{*}s^{*}$.
\begin{align}
    &x\in (r^{*}+s^{*})^{*} \implies x\in (r^{*}s^{*})^{*}
\end{align}
Widzimy zatem, że $(r+s)^{*} \subseteq (r^{*}s^{*})^{*}$.\\

\noindent
Na mocy AE z powyższych zawierań wynika, że $(r^{*}s^{*})^{*} = (r+s)^{*}_{\qed}$.


\end{document}
