\documentclass{article}

\usepackage[polish]{babel}
\usepackage[utf8]{inputenc}
\usepackage{polski}
\usepackage[T1]{fontenc}
 
\usepackage[margin=1.5in]{geometry} 

\usepackage{color} 
\usepackage{amsmath}                                                                    
\usepackage{amsfonts}                                                                   
\usepackage{graphicx}                                                             
\usepackage{booktabs}
\usepackage{amsthm}
\usepackage{pdfpages}
\usepackage{hyperref}

\theoremstyle{definition}
\newtheorem{de}{Definicja}[subsection]

\theoremstyle{definition}
\newtheorem{tw}{Twierdzenie}[subsection]

\theoremstyle{definition}
\newtheorem{pk}{Przykład}[subsection]

\theoremstyle{definition}
\newtheorem*{fakt}{FAKT}

\author{Rafał Włodarczyk}
\title{Fizyka I}  
\date{Informatyka Algorytmiczna 2 Semestr 2024}

\begin{document}

\maketitle

\section{Wszechświat}

\begin{itemize}
\item Wiek Wszechświata określany jest na $13.787\pm0.020$ miliardów lat [według Lambda-CDM concordance model, 2021] (inaczej ok. $4.3 \cdot 10^{17}$s)
\item Odwrotność wieku Wszechświata stanowi stała Hubble'a: 
\[\frac{1}{T_{\text{universe}}} = 74.2\pm 3.6 \frac{\frac{km}{s}}{Mpc}\]
Gdzie $pc$ - parsek, czyli odległość, dla której kąt paralaksy (równy połowie kąta całkowitej rocznej zmiany) położenia Ziemi widzianej prostopadle do płaszczyzny orbity wynosi 1 sekundę łuku.
\item Rozmiary charakterystycznych obiektów:
    \begin{enumerate}
    \item Ziemia: $R_{\text{Earth}} = 6.5 \cdot 10^{6} m$, $D_{\text{Earth - Sun}} = 1\text{AU} \approx 150$ mln km
    \item Słońce: $R_{\text{Sun}} = 4\cdot 10^{12} m$
    \item Układ Słoneczny: $R_{\text{SolarSystem}} = 1921.56$ AU [Sedna]
    \item Droga Mleczna: $R_{\text{MilkyWay}} \approx 4 \cdot 10^4 \cdot 10^{16} = 4\cdot 10^{20} m$
    \item Klaster Galaktyk: $R_{\text{ClusterOfGalaxies}} \approx 10^{23} m$
    \item Superklaster Galaktyk $R_{\text{SuperclusterOfGalaxies}} \approx 300$ mln ly ($1ly \approx 10^{16} m$ - rok świetlny)
    \end{enumerate}
\end{itemize}
\subsection{Skala jednorodności}
> doprecyzować
\subsection{Zasada kosmologiczna}
Wszechświat jest \underline{jednorodny} i \underline{izotropowy}. Żaden punkt nie jest wyróżniony, rozkład materii jest izotropowy - podobny w każdym kierunku, w którym spogląda obserwator.
\section{Pojęcia podstawowe}
\subsection{Absolutna przestrzeń oraz czas}
(Postulat Mechaniki Klasycznej)
Istnieje przestrzeń - arena wszystkich zdarzeń oraz niezależny, absolutny czas. Każdy obserwator mierzy ten sam czas, niezależnie od prędkości, z którą się porusza.
\subsection{Stan układu}
(Postulat Mechaniki Klasycznej)
Stan układu to zbiór wewnętrznych zmiennych niezależnych opisujących układ. W mechanice klasycznej jest to położenie $\vec{x}$ oraz pęd $\vec{p}$. 
\section{Zasady Dynamiki Newtona}
\subsection*{Pierwsza zasada dynamiki Newtona}
(Istnieje układ inercjalny) Jeżeli wypadkowa sił działających na ciało jest równa zeru, to ciało pozostaje w spoczynku lub porusza się ruchem jednostajnym prostoliniowym.
\begin{equation}
\sum \vec{F} = 0 \implies \frac{d\vec{v}}{dt} = 0
\end{equation}
\subsection*{Druga zasada dynamiki Newtona}
Jeżeli wypadkowa sił działających na ciało nie jest równa zeru, to przyspieszenie ciała jest wprost proporcjonalne do tej siły i odwrotnie proporcjonalne do masy ciała.
\begin{equation}
\sum \vec{F} = m \vec{a}
\end{equation}
gdzie \(\vec{F}\) to wypadkowa siła, \(m\) to masa ciała, a \(\vec{a}\) to przyspieszenie.

\subsection*{Trzecia zasada dynamiki Newtona}
Jeżeli ciało A działa na ciało B siłą \(\vec{F}_{AB}\), to ciało B działa na ciało A siłą \(\vec{F}_{BA}\) równą co do wartości i przeciwnie skierowaną.
\begin{equation}
\vec{F}_{AB} = -\vec{F}_{BA}
\end{equation}
\section{Równanie ruchu}
\subsection{W przypadku stałej siły}
Zacznijmy od drugiej zasady dynamiki Newtona:

\[ \vec{F} = m \vec{a} \]

gdzie \( \vec{a} \) to przyspieszenie, a \( m \) to masa ciała. Przyspieszenie jest drugą pochodną wektora położenia \( \vec{r}(t) \) względem czasu, czyli:
\[ \vec{a} = \frac{d^2 \vec{r}(t)}{dt^2} \]
Podstawiając to do równania Newtona, otrzymujemy:
\[ \vec{F} = m \frac{d^2 \vec{r}(t)}{dt^2} \]
Rozwiązując to równanie dla \( \vec{r}(t) \), możemy skorzystać z metody rozdzielania zmiennych. Ponieważ \( \vec{F} \) jest stałe, możemy rozpatrywać ruch wzdłuż każdej osi współrzędnych osobno.
Załóżmy, że \( \vec{F} = (F_x, F_y, F_z) \). Wtedy równania różniczkowe dla każdej składowej będą wyglądały następująco:
\[ F_x = m \frac{d^2 x(t)}{dt^2} \]
\[ F_y = m \frac{d^2 y(t)}{dt^2} \]
\[ F_z = m \frac{d^2 z(t)}{dt^2} \]
Rozwiążemy każde z tych równań po kolei. Na przykład dla składowej \( x \):
\[ \frac{d^2 x(t)}{dt^2} = \frac{F_x}{m} \]
Jest to równanie różniczkowe drugiego rzędu o stałym współczynniku. Rozwiązaniem tego równania jest:
\[ x(t) = \frac{F_x}{2m} t^2 + v_{0x} t + x_0 \]
gdzie \( v_{0x} \) to początkowa prędkość wzdłuż osi \( x \), a \( x_0 \) to początkowe położenie.
Podobnie, rozwiązania dla \( y(t) \) i \( z(t) \) będą miały postać:
\[ y(t) = \frac{F_y}{2m} t^2 + v_{0y} t + y_0 \]
\[ z(t) = \frac{F_z}{2m} t^2 + v_{0z} t + z_0 \]
Łącząc te wyniki, wektor położenia \( \vec{r}(t) \) w trzech wymiarach wynosi:
\[ \vec{r}(t) = \left( \frac{F_x}{2m} t^2 + v_{0x} t + x_0, \frac{F_y}{2m} t^2 + v_{0y} t + y_0, \frac{F_z}{2m} t^2 + v_{0z} t + z_0 \right) \]
Zatem wektor położenia \( \vec{r}(t) \) jako funkcja czasu dla ciała poruszającego się pod wpływem stałej siły \( \vec{F} \) wynosi:
\[
\boxed{
\vec{r}(t) = \left( a_x t^2 + v_{0x} t + x_0, a_y t^2 + v_{0y} t + y_0, a_z t^2 + v_{0z} t + z_0 \right)
}
\]
\subsection{W przypadku siły zależnej od prędkości}
Aby sformułować oraz rozwiązać równanie ruchu w 3 wymiarach dla siły \( \vec{F} \) proporcjonalnej do prędkości \( \vec{v} \), zaczniemy od drugiej zasady dynamiki Newtona:
\[ \vec{F} = m \vec{a} \]
Zakładamy, że siła jest proporcjonalna do prędkości, czyli:
\[ \vec{F} = -k \vec{v} \]
gdzie \( k \) jest stałą proporcjonalności (współczynnikiem tłumienia), a \( \vec{v} = \frac{d\vec{r}}{dt} \) to prędkość. Równanie ruchu staje się więc:
\[ m \frac{d\vec{v}}{dt} = -k \vec{v} \]
Możemy to zapisać jako:
\[ \frac{d\vec{v}}{dt} = -\frac{k}{m} \vec{v} \]
To równanie różniczkowe jest liniowe i jednorodne. Rozwiążemy je metodą rozdzielania zmiennych.
\[ \frac{d\vec{v}}{\vec{v}} = -\frac{k}{m} dt \]
\[ \int \frac{d\vec{v}}{\vec{v}} = -\frac{k}{m} \int dt \]
\[ \ln|\vec{v}| = -\frac{k}{m} t + C \]
\[ \vec{v} = \vec{v}_0 e^{-\frac{k}{m} t} \]
gdzie \( \vec{v}_0 \) to wektor początkowej prędkości.
Teraz, aby znaleźć wektor położenia \( \vec{r}(t) \), musimy zintegrować prędkość:
\[ \vec{v} = \frac{d\vec{r}}{dt} \]
\[ \frac{d\vec{r}}{dt} = \vec{v}_0 e^{-\frac{k}{m} t} \]
\[ \vec{r}(t) = \int \vec{v}_0 e^{-\frac{k}{m} t} dt \]
Ponieważ \(\vec{v}_0\) jest stałe, możemy całkować każdy składnik osobno:
\[ \vec{r}(t) = \vec{v}_0 \int e^{-\frac{k}{m} t} dt \]
\[ \vec{r}(t) = \vec{v}_0 \left( -\frac{m}{k} e^{-\frac{k}{m} t} \right) + \vec{r}_0 \]
gdzie \( \vec{r}_0 \) to wektor początkowego położenia.
\[
\boxed{
\vec{r}(t) = \vec{r}_0 - \frac{m}{k} \vec{v}_0 e^{-\frac{k}{m} t}
}
\]

\section{Ruch drgający}
\subsection{Oscylator Harmoniczny}
Oscylator harmoniczny – układ drgający wykonujący ruch harmoniczny.
W układzie takim występuje siła sprężysta $F(x)$ proporcjonalna do przemieszczenia $x$ tego układu od jego położenia równowagi: 
\[ F(x) = -kx \]

\subsection{Położenie od czasu w ruchu harmonicznym}
Zapiszmy powyższe równanie jako:
\[ m\frac{d^2 x}{dt^2} = - kx \]
Podstawmy:
\[
x(t) = e^{\alpha t}
\]
\[
m\frac{dx}{dt} = \alpha e^{\alpha t}
\]
Obliczmy pierwszą i drugą pochodną tego wyrażenia:
\[
m\frac{d^2 x}{dt^2} = \alpha^2 e^{\alpha t}
\]
Podstawiamy teraz te pochodne do równania różniczkowego:
\[
\alpha^2 e^{\alpha t} = - \frac{k}{m} e^{\alpha t}
\]
Ponieważ \( e^{\alpha t} \neq 0 \) dla dowolnej wartości \( \alpha \), możemy podzielić obie strony równania przez \( e^{\alpha t} \):
\[
\alpha^2 = -\frac{k}{m}
\]
Stąd otrzymujemy:
\[
\alpha^2 = -\frac{k}{m}
\]
A więc:
\[
\alpha = \pm i\sqrt{\frac{k}{m}}
\]
To oznacza, że mamy dwa rozwiązania dla \( \alpha \):
\[
\alpha_1 = i\sqrt{\frac{k}{m}} \quad \text{oraz} \quad \alpha_2 = -i\sqrt{\frac{k}{m}}
\]
Ogólne rozwiązanie równania różniczkowego jest kombinacją liniową dwóch podstawowych rozwiązań:
\[
x(t) = C_1 e^{i\sqrt{\frac{k}{m}} t} + C_2 e^{-i\sqrt{\frac{k}{m}} t}
\]
Aby uzyskać rzeczywiste rozwiązanie, stosujemy tożsamości Eulera:
\[
e^{i\theta} = \cos(\theta) + i\sin(\theta) \quad \text{oraz} \quad e^{-i\theta} = \cos(\theta) - i\sin(\theta)
\]
Podstawiając \( \theta = \sqrt{\frac{k}{m}} t \):
\[
e^{i\sqrt{\frac{k}{m}} t} = \cos\left(\sqrt{\frac{k}{m}} t\right) + i\sin\left(\sqrt{\frac{k}{m}} t\right) \quad \text{oraz} \quad e^{-i\sqrt{\frac{k}{m}} t} = \cos\left(\sqrt{\frac{k}{m}} t\right) - i\sin\left(\sqrt{\frac{k}{m}} t\right)
\]
Zatem ogólne rozwiązanie można zapisać jako:
\[
x(t) = C_1 \left(\cos(\sqrt{\frac{k}{m}} t\right) + i\sin\left(\sqrt{\frac{k}{m}} t)\right) + C_2 (\cos\left(\sqrt{\frac{k}{m}} t\right) - i\sin\left(\sqrt{k} t)\right)
\]
Porządkując:
\[
x(t) = (C_1 + C_2) \cos\left(\sqrt{\frac{k}{m}} t\right) + i(C_1 - C_2) \sin\left(\sqrt{\frac{k}{m}} t\right)
\]
Aby rozwiązanie było rzeczywiste, współczynniki przy częściach urojonych muszą być zerowe. Zatem przyjmujemy nowe stałe:
\[
A = C_1 + C_2 \quad \text{oraz} \quad B = i(C_1 - C_2)
\]
W końcu, rzeczywiste ogólne rozwiązanie ruchu harmonicznego to:
\[
x(t) = A \cos\left(\sqrt{\frac{k}{m}} t\right) + B \sin\left(\sqrt{\frac{k}{m}} t\right)
\]
W przypadku bardziej ogólnego ruchu harmonicznego prostego, możemy zapisać:
\[
\boxed{
x(t) = A \cos\left(\sqrt{\frac{k}{m}} t + \varphi\right)
}
\]
gdzie \( C \) jest amplitudą, a \( \varphi \) fazą początkową, które można wyznaczyć z warunków początkowych.

\subsection{Drgania tłumione}
Drgania tłumione można opisać za pomocą następującego równania ruchu:
\[ m \frac{d^2 x}{dt^2} = - b \frac{dx}{dt} - kx \]
Gdzie $b$ jest współczynnikiem tłumienia. Metodą analogiczną do poprzedniego równania:
\[
\boxed {
x(t) = A_0 e^{-\frac{b}{2m}t}\cos\left(\sqrt{\frac{k}{m}}t + \varphi\right)
}\]

\subsection{Drgania wymuszone}
Drgania wymuszone można opisać za pomocą następującego równania ruchu:
\[ m \frac{d^2 x}{dt^2} = -kx -b \frac{dx}{dt} + F_0 \sin\left(\sqrt{\frac{k}{m}t} + \varphi\right)\]
(Nie rozważając stanu przejściowego) Po pewnym czasie układ ustabilizuje się z częstotliwością własną:
\[
\boxed {
x(t) = A \cos\left(\sqrt{\frac{k}{m}} t + \varphi\right)
}\]

\section{Zasada Zachowania Energii}

\boxed{\text{Całkowita energia w układzie izolowanego pozostaje stała}}

\subsection{W Mechanice Klasycznej}

Całkowita energia (suma energii potencjalnej i kinetycznej) w ukłądzie zamkniętym pozostaje stała, jeżeli działające w układzie siły są zachowawcze.

\subsection{Siły Zachowawcze}
Siły, których praca zależy tylko od początkowego i końcowego położenia ciała, bez względu na drogę, którą to ciało przebyło.
\begin{itemize}
    \item Siła grawitacji (Pole grawitacyjne jest polem zachowawczym)
    \item Siła sprężystości
    \item Siła Lorentza
\end{itemize}

\subsection{Siły Niezachowawcze}
Siły, których praca zależy od drogi, którą przebyło ciało.
\begin{itemize}
    \item Siła tarcia
    \item Siła oporu ośrodka
\end{itemize}

\section{Zasada zachowania pędu}

\boxed{\text{Pęd wszystkich elementów układu izolowanego pozostaje stały}}\

\subsection{Zderzenia idealnie sprężyste}

Zderzenie, w którym zostaje zachowany pęd oraz energia mechaniczna. Ilość obiektów w stanie początkowym i końcowym jest taka sama,

\subsection{Zderzenia idealnie niesprężyste}

Zderzenie, w którym zostaje zachowany pęd, ale nie zostaje zachowana energia mechaniczna.

\section{Zasada zachowania momentu pędu}
Moment pędu układu cząstek wokół punktu, w ustalonym inercjalnym układzie odniesienia, jest zachowany, jeśli wypadkowy moment sił zewnętrznych względem tego punktu jest równy zero.
\[
\boxed{
\frac{\vec{dL}}{dt} = 0
}\]

Przykłady wykorzystania zasady zachowania momentu pędu:
\begin{enumerate}
    \item (Obieg planet wokół słońca) Planety mają stały moment pędu względem Słońca, co wynika z braku zewnętrznych momentów sił działających w znaczący sposób na układ.
    \item (Łyżwy) Gdy łyżwiarz zmniejsza moment bezwładności (np. przyciągając ramiona do ciała), prędkość kątowa rośnie, aby moment pędu pozostał stały. Podobny efekt można osiągnąć kręcąc się na krześle, kierując ręce i nogi na zmianę, do siebie oraz od siebie.
    \item (Koła zamachowe) W niektórych maszynach koła zamachowe przechowują moment pędu, który pomaga utrzymać stabilność pracy maszyny.
\end{enumerate}

\section{Bryła Sztywna}
Ciało fizyczne złożone ze zbioru punktów, które nie przemieszczają się względem siebie. 

\subsection{Moment bezwładności}
Moment bezwładności bryły sztywnej jest wielkością, która określa moment obrotowy potrzebny do uzyskania pożądanego przyspieszenia kątowego wokół osi obrotu, podobnie jak masa określa siłę potrzebną do uzyskania pożądanego przyspieszenia.

\[ I = \sum_{i} m_i r_i^2 \]

\subsection{Moment pędu w ruchu obrotowym}

Moment pędu w ruchu obrotowym to wektorowa wielkość fizyczna opisująca stan ruchu obrotowego ciała względem określonej osi. Jest on analogiczny do pędu w ruchu prostoliniowym, ale dotyczy obrotu. Moment pędu jest oznaczany symbolem \(\vec{L}\) i definiowany jako iloczyn wektorowy wektora położenia \(\vec{r}\) i wektora pędu \(\vec{p}\):
\[
\vec{L} = \vec{r} \times \vec{p}
\]
Dla ciała o masie \(m\) poruszającego się z prędkością \(\vec{v}\), pęd \(\vec{p}\) jest równy \(m\vec{v}\), więc:

\[
\vec{L} = \vec{r} \times (m\vec{v}) = m (\vec{r} \times \vec{v})
\]

W przypadku ruchu obrotowego ciała sztywnego moment pędu można także wyrazić w zależności od momentu bezwładności \(I\) i prędkości kątowej \(\vec{\omega}\):

\[
\vec{L} = I \vec{\omega}
\]

\subsection{Równanie ruchu obrotowego - dla małych drgań}

Wychodząc z momentu pędu:
\[\vec{L} = \vec{r} \times \vec{p} = \vec{r} \times m \dot{\vec{r}}\]
Różniczkując moment pędu względem czasu, otrzymujemy:
\[
\dot{\vec{L}} = \frac{d}{dt} \left( \vec{r} \times m \dot{\vec{r}} \right)
\]
Używając reguły iloczynu, dostajemy:
\[
\dot{\vec{L}} = \dot{\vec{r}} \times m \dot{\vec{r}} + \vec{r} \times m \ddot{\vec{r}}
\]
Pierwszy człon po prawej stronie zanika, ponieważ wektor prędkości $\dot{\vec{r}}$ jest równoległy do samego siebie:
\[
\dot{\vec{r}} \times m \dot{\vec{r}} = 0
\]
Pozostaje więc:
\[
\dot{\vec{L}} = \vec{r} \times m \ddot{\vec{r}} = \vec{r} \times \vec{F}
\]
Załóżmy, że ciało porusza się w osi $z$, po okręgu o promieniu \(R\) i kąt \(\alpha\) mierzy się od pionu. Wtedy położenie \(\vec{r}\) w układzie biegunowym można zapisać jako:
\[
\vec{r} = R (\sin(\alpha) \hat{i} + \cos(\alpha) \hat{j})
\]
Przy czym \(\vec{g}\) skierowane jest w dół, więc:
\[
\vec{g} = -g \hat{j}
\]
Moment siły wtedy jest równy:
\[
\vec{M} = R (\sin(\alpha) \hat{i} + \cos(\alpha) \hat{j}) \times m (-g \hat{j})
\]
Ponieważ \(\hat{j} \times \hat{j} = 0\) i \(\hat{i} \times \hat{j} = \hat{k}\), mamy:
\[
\vec{M} = R \sin(\alpha) \hat{i} \times (-mg \hat{j}) = -mgR \sin(\alpha) \hat{k}
\]
Moment bezwładności dla ruchu obrotowego \(I_0\) wokół osi przez środek obrotu wynosi:
\[
I_0 \frac{d^2 \alpha}{dt^2} = M_z
\]
gdzie \(M_z\) to składowa momentu siły w kierunku osi obrotu \(z\).
\[
I_0 \frac{d^2 \alpha}{dt^2} = -mgR \sin(\alpha)
\]
Dla małych drgań ($\alpha<\frac{\pi}{6}$, $\sin(\alpha)\approx \alpha$), przybliżenie \(\sin(\alpha) \approx \alpha\) jest uzasadnione. Mamy:
\[
I_0 \frac{d^2 \alpha}{dt^2} = -mgR \alpha
\]
Rozwiązaniem tego równania różniczkowego jest:
\[
\boxed{
\alpha(t) = A\cos(\omega_0 t + \varphi)
}
\]

\section{Ruch w mechanice klasycznej}

\subsection{Zasada względności Galileusza}

Prawa mechaniki są takie same w \underline{inercjalnym} układzie odniesienia.

\subsection{Transfrmacje Galileusza}

Załóżmy, że mamy dwa układy odniesienia: $S$ i $S^\prime$. Układ $S^\prime$ porusza się względem układu $S$ z prędkością $v$ wzdłuż osi OX.
Transformacje Galileusza między $S(x,y,z,t), S^\prime(x^\prime,y^\prime,z^\prime,t^\prime)$ możemy opisać jako:
\begin{itemize}
    \item $x' = x-vt$
    \item $y' = y$
    \item $z' = z$
    \item $t' = t$
\end{itemize}
Założenie, że czas jest absolutny jest nieprawdziwe.
Założenie, że przestrzeń jest absolutna również jest nieprawdziwe.

\subsection{Zasada względności Einsteina}

\begin{enumerate}
\item Prawa mechaniki są takie same w \underline{inercjalnym} układzie odniesienia.
\item Prędkość światła jest niezmiennicza $c=inv.$ (invariant)
\end{enumerate}

\subsection{Transformacje Lorentza}

Załóżmy, że mamy dwa układy odniesienia: $K$ i $K^\prime$. Dla każdego z nich postaje sferyczna fala świetlna:
\begin{align}
K&: x^2 + y^2 + z^2 = R^2 = (ct)^2\\
K^\prime &: (x^\prime)^2 + (y^\prime)^2 + (z^\prime)^2 = (R^\prime)^2 = (ct^\prime)^2
\end{align}
Niech układ $K^\prime$ porusza się względem układu $K$ z prędkością $v$ wzdłuż osi OX, wtedy $y=y', z=z'$
Spróbujmy wyprowadzić zależności między układami:\
\[
\begin{cases}
x = a_{11} x^\prime + a_{12} t^\prime\\
t = a_{21} x^\prime + a_{22} t^\prime
\end{cases}
\]
Skoro promień świetlny jest linią prostą (idealnie prostą w próżni), to ta relacja musi być relacją liniową. Jej współczynniki wynoszą:
\[\gamma=a_{11}=a_{22}=\frac{1}{\sqrt{1-\frac{u^2}{c^2}}}\]
\[a_{12}=\frac{u}{\sqrt{1-\frac{u^2}{c^2}}}\]
\[a_{21}=\frac{\frac{u}{c^2}}{\sqrt{1-\frac{u^2}{c^2}}}\]
Zatem transformacje Lorentza możemy zapisać jako:
\[
\begin{cases}
x = \gamma(x^\prime + ut^\prime)\\
t = \gamma(t^\prime + \frac{ux^\prime}{c^2})
\end{cases}
\]

\section{Efekty Szczególnej Teorii Względności}
\subsection{Dylatacja czasu}

Rozważmy dwa wydarzenia, które zachodzą w tym samym miejscu w układzie spoczynkowym \( S \). Czas między nimi dla obserwatora w \( S \) to \( \Delta t \). Chcemy znaleźć czas między tymi samymi wydarzeniami dla obserwatora poruszającego się z prędkością \( v \) w układzie \( S' \).
Z transformacji Lorentza dla czasu, gdy \( \Delta x = 0 \):
\[ t' = \gamma t \]
Różnica czasów \( \Delta t' \) w układzie \( S' \) jest więc:
\[ \Delta t' = \gamma \Delta t \]
Podstawiając wartość \( \gamma \):
\[ \Delta t' = \frac{\Delta t}{\sqrt{1 - \frac{v^2}{c^2}}} \]

\subsection{Skrócenie długości}

Rozważmy pręt poruszający się wzdłuż osi \( x \) w układzie spoczynkowym \( S \), gdzie jego długość wynosi \( L \). Chcemy znaleźć długość \( L' \) tego samego pręta w układzie \( S' \), poruszającym się z prędkością \( v \) względem \( S \).
Z transformacji Lorentza dla przestrzeni:

\[ x' = \gamma (x - vt) \]
Długość pręta w układzie \( S' \) jest zdefiniowana jako odległość między dwoma końcami pręta, stąd:

\[ L' = x'_2 - x'_1 = \gamma (x_2 - vt_2) - \gamma (x_1 - vt_1) \]
\[ L' = \gamma [(x_2 - x_1) - v(t_2 - t_1)] \]
Załóżmy, że pręt porusza się wzdłuż osi \( x \) i jest w spoczynku w układzie \( S \), więc \( x_2 - x_1 = L \). Czas trwania pomiaru długości pręta w obu układach to \( \Delta t = t_2 - t_1 \).
Zatem:

\[ L' = \gamma [L - v \Delta t] \]
Podstawiając \( \gamma = \frac{1}{\sqrt{1 - \frac{v^2}{c^2}}} \):

\[ L' = \frac{L}{\sqrt{1 - \frac{v^2}{c^2}}} \]

\subsection{Relatywistyczne składanie prędkości}

Rozważmy układ \( S \) i poruszający się względem niego z prędkością $v$, względem osi OX, układ \( S' \).
Zgodnie z transformacją Lorentza:
\[
\begin{cases}
    x' = \frac{x-vt}{\sqrt{1-\frac{v^2}{c^2}}}\\
    y' = y\\
    z' = z\\
    t' = \frac{t-\frac{vx}{c^2}}{\sqrt{1-\frac{v^2}{c^2}}}
\end{cases}
\]
Zapiszmy:
\[ 
dx' = \frac{dx - vdt}{\sqrt{1-\frac{v^2}{c^2}}} \text{ oraz } dt' = \frac{dt-\frac{v}{c^2} dx}{\sqrt{1-\frac{v^2}{c^2}}}
\]
Następnie:
\[
u_{x'} = \frac{dx'}{dt'} = \frac{dx-vdt}{dt-\frac{v}{c^2} dx} = \frac{\frac{dx}{dt}-v}{1-\frac{v}{c^2}\frac{dx}{dt}}=\frac{u_x - v}{1-\frac{v\cdot u_x}{c^2}}
\]

\section{Czasoprzestrzeń}
\subsection{Interwał czasoprzestrzenny}

Interwał czasoprzestrzenny między dwoma punktami w czasoprzestrzeni można zdefiniować jako różnicę pomiędzy kwadratem czasoprzestrzennych współrzędnych tych punktów.

\[\tau^2 = (x)^2 - (ct)^2\]

\subsection{Stożek świetlny}
\subsection{Niezmienniczość interwału}

\section{Diagram czasoprzestrzenny}
\subsection{Ilustracja dylatacji czasu}
\subsection{Ilustracja skrócenia długości}

\section{Wprowadzenie do mechaniki kwantowej}
\subsection{Widmo liniowe atomu wodoru}
\subsection{Promieniowanie ciała doskonale czarnego}
\subsection{Efekt fotoleketryczny}

\section{Hipoteza Plancka}

\section{Model Bohra}

\section{Hipoteza de Broglie'a}

\section{Dualizm korpuskularno falowy promieniowania oraz materii}

\section{Funkcja falowa}
\subsection{Równanie Schrödingera}

\section{Pomoc matematyczna}
\subsection{Iloczyn skalarny}
Niech \(\vec{v}, \vec{w}\) - wektory. Iloczyn skalarny:
\[
\langle \vec{v}, \vec{w} \rangle = v \cdot w \cdot \cos(\alpha)
\]

\subsection{Iloczyn wektorowy}
Niech \(\vec{v}, \vec{w}\) - wektory. Iloczyn wektorowy:
\[
\vec{v} \times \vec{w} = v \cdot w \cdot \sin(\alpha)
\]

\end{document}