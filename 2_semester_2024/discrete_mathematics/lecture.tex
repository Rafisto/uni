le\documentclass{article}

\usepackage[polish]{babel}
\usepackage[utf8]{inputenc}
\usepackage{polski}
\usepackage[T1]{fontenc}
 
\usepackage[margin=1.5in]{geometry} 
\usepackage{color} 
\usepackage{amsmath}
\usepackage{amsfonts}
\usepackage{graphicx}
\usepackage{booktabs}
\usepackage{amsthm}
\usepackage{pdfpages}
\usepackage{wrapfig}
\usepackage{hyperref}
\usepackage{etoolbox}
\AtBeginEnvironment{align}{\setcounter{equation}{0}}

\theoremstyle{definition}
\newtheorem{de}{Definicja}[subsection]

\theoremstyle{definition}
\newtheorem{tw}{Twierdzenie}[subsection]

\theoremstyle{definition}
\newtheorem{pk}{Przykład}[subsection]

\theoremstyle{definition}
\newtheorem*{fakt}{FAKT}

\author{Rafal Wlodarczyk}
\title{Matematyka Dyskretna}  
\date{INA 2 Sem. 2023}

\begin{document}

\maketitle

\section{Wykład I}

\subsection{Współczynniki Dwumianowe}

\begin{center}
  $\mathbb{N}=\{0,1,2,3,\dots\}$\\
  $\mathbb{N}^{+}=\{1,2,3,\dots\}$
\end{center}

$n\in\mathbb{N}^{+}$: $[n]=\{1,2,3,\dots,n\}$

\begin{de}
  Silnia. Niech $n\in\mathbb{N}$. Definiujemy:
  \begin{center}
    $0!=1$\\
    $n!=1\cdot2\cdot \dots \cdot n$
  \end{center}
\end{de}

\begin{de}
  Silnia górna. Niech $x\in \mathbb{R}, n\in \mathbb{N}$. Definiujemy:
  \begin{center}
    $x^0 = 1$\\
    $x^n = x(x+1)(x+2)\dots (x+n-1), n\geq 1$.
  \end{center}
\end{de}

\begin{de}
  Silnia dolna. Analogicznie
\end{de}

\begin{de}
  Współczynnik dwumianowy (symbol Newtona). Niech $x\in\mathbb{R}, k\in\mathbb{N}$. Definiujemy:
  \begin{center}
    $\binom{x}{k} = \frac{x^{\underline{k}}}{k!}$
  \end{center}
\end{de}
Uwaga. Czasami wygodnie będzie rozszerzyć defonicję $\binom{x}{k}$ na $k\in\mathbb{Z}$ wtedy dla $k<0$ przyjmujemy $\binom{x}{k}=0$\\
Interpretacja kombinatoryczna: $k, n\in\mathbb{N}, n\geq k$:\\
\begin{center}
  $\binom{n}{k}$ - \# podzbiorów k-elementowych zbioru n-elementowego
\end{center}

\begin{pk}
  Rozważ następujące ćwiczenia: \\
  Ćwiczenie 1. Pokaż, że podzbiorów k-elementowych zbioru n-elementowego jest $\frac{n!}{k!(n-k)!}$.\\
  Ćwiczenie 2. Niech $k,n \in \mathbb{N}, n\geq k$. Wtedy $\binom{n}{k}=\binom{n}{n-k}$\\
  Ćwiczenie 3. Reguła Pochłaniania. Niech $n\in\mathbb{N} k\in\mathbb{N}^{+}, n\geq k$. Wtedy: $\binom{n}{k} = \frac{n}{k} \binom{n-1}{k-1}$
\end{pk}

\begin{tw}
  Dwumian Newtona. Niech $x,y\in \mathbb{R}, n \in \mathbb{N}$. Wtedy:
  \begin{center}
    $(x+y)^n = \sum_{k=0}^{n} \binom{n}{k} x^k y^{n-k}$
  \end{center}
  D-d. (indukcyjny) - ćwiczenie\\
  D-d. (kombinatoryczny)\\
  Dokonujemy mnożenia:
  \begin{center}
    $(x+y)(x+y)\dots(x+y)=x^n + \_ x^{n-1}\cdot y + \dots + x\cdot y^{n-1} + y^n$
  \end{center}
  Wystarczy zauważyć, że współczynnik przy $x^k\cdot y^{n-k}$ to liczba sposobów, na jakie spośród 
  $n$ czynników $(x+y)$ możemy wybrać $k$ nawiasów jako te, z których wybieramy składniki $x$.
\end{tw}

Wniosek 1:\\
\begin{center}
  $2^n = (1+1)^n = \sum_{k=0}^{n} \binom{n}{k} 1^k 1^{n-k} = \sum_{k=0}^{n} \binom{n}{k}$\\
  $2^n = \sum_{k=0}^{n} \binom{n}{k}$ - liczba wszystkich podzbiorów zbioru $n$ - elementowego
\end{center}

Wniosek 2:\\
\begin{center}
  $0=0^n=(1-1)^n=\sum_{k=0}^{n} \binom{n}{k} (-1)^{k} 1^{n-k} = \sum_{k=0}^{n} \binom{n}{k} (-1)^{k}$\\
  $0=\sum_{k=0}^{n} \binom{n}{k} (-1)^k$
\end{center}

Zatem widzimy że:
\begin{center}
  $\sum_{k=0}^{n} \binom{n}{k} :$(2 nie dzieli $k$) $ = \sum_{k=0}^{n} \binom{n}{k} :$(2 dzieli $k$)
\end{center}
\# podzbiorów o mocy parzystej zbioru $n$-elementowego = \# podzbiorów o mocy nieparzystej zbioru $n$-elementowego

\begin{tw}
  Tożsamość Pascala. Niech $n\in \mathbb{N}, k\in \mathbb{N}^{+}, n>k$. Wtedy:
  \begin{center}
    $\binom{n}{k} = \binom{n-1}{k} + \binom{n-1}{k-1}$
  \end{center}
  D-d. (analityczny). Niech $x\in\mathbb{R}$.\\
  $\sum_{k=0}^{n} \binom{n}{k} x^{k}=$\\
  $=\sum_{k=0}^{n} \binom{n}{k} x^{k} 1^{n-k}=$\\
  $=(x+1)^{n} = (x+1)^{n-1} \cdot (x+1)=$\\
  $=(\sum_{k=0}^{n-1} \binom{n-1}{k} x^k)(x+1)=$\\
  $=\sum_{k=0}^{n-1} \binom{n-1}{k} x^{k+1} + \sum_{k=0}^{n-1} \binom{n-1}{k} x^{k}=$\\
  $=\sum_{k=1}^{n} \binom{n-1}{k-1} x^{k} + \sum_{k=0}^{n} \binom{n-1}{k} x^k$\\
  Zatem:\\
  $\sum_{k=0}^{n} \binom{n}{k} x^k = \sum_{k=1}^{n} \binom{n-1}{k-1} x^k + \sum_{k=0}^{n-1} \binom{n-1}{k} x^k$\\
  Współczynniki przy odpowiadających sobie są równe, zatem dla $k\in\mathbb{N}^{+}, k<n$ mamy:
  
  \begin{center}
    $\binom{n}{k} = \binom{n-1}{k-1} + \binom{n-1}{k}$
  \end{center}
  D-d. (kombinatoryczny). Zliczmy na dwa sposoby \# podzbiorów $k$-elementowych zbioru $n$-elementowego.
  \begin{enumerate}
    \item $\binom{n}{k}$ - z definicji
    \item Wyróżniamy jeden element $*$ w zbiorze $n$-elementowym. Podzbiory $k$-elementowe dzielą się teraz na dwie klasy:\\
      Te, które nie zawierają $*$. Jest ich $\binom{n-1}{k}$ (gwiazdki nie ma)\\
      Te, które zawierają $*$. Jest ich $\binom{n-1}{k-1}$ (gwiazdkę wybieram)\\
      Zatem zachodzi twierdzenie.
  \end{enumerate}
\end{tw}


\end{document}
