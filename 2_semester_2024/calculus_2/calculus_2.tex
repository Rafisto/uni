\documentclass{article}

\usepackage[polish]{babel}
\usepackage[utf8]{inputenc}
\usepackage{polski}
\usepackage[T1]{fontenc}
 
\usepackage[margin=1.5in]{geometry} 
\usepackage{color} 
\usepackage{amsmath}
\usepackage{amsfonts}
\usepackage{graphicx}
\usepackage{booktabs}
\usepackage{amsthm}
\usepackage{pdfpages}
\usepackage{wrapfig}
\usepackage{hyperref}
\usepackage{etoolbox}
\usepackage{tikz}

\usetikzlibrary{calc}

\AtBeginEnvironment{align}{\setcounter{equation}{0}}

\theoremstyle{definition}
\newtheorem{de}{Definicja}[subsection]

\theoremstyle{definition}
\newtheorem{tw}{Twierdzenie}[subsection]

\theoremstyle{definition}
\newtheorem{pk}{Przykład}[subsection]

\theoremstyle{definition}
\newtheorem*{wn}{Wniosek}

\theoremstyle{definition}
\newtheorem*{dw}{Dowód}

\theoremstyle{definition}
\newtheorem*{uw}{Uwaga}

\theoremstyle{definition}
\newtheorem*{fakt}{FAKT}

\author{Rafal Wlodarczyk}
\title{Analiza Matematyczna II}  
\date{INA 2 Sem. 2023}

\begin{document}

\maketitle

\tableofcontents

\section{Wykład I}

\subsection{Iloczyn skalarny}

\begin{de}
    Przestrzeń \(\mathbb{R}^n\) jest zbiorem wszystkich \(n\)-wymiarowych wektorów o rzeczywistych współrzędnych.
    \[
    \mathbb{R}^n = \{ (x_1, x_2, \dots, x_n) \mid x_i \in \mathbb{R} \text{ dla } i = 1, 2, \dots, n \}
    \]
    Każdy wektor w \(\mathbb{R}^n\) można zapisać jako uporządkowany zbiór \(n\) rzeczywistych liczb, gdzie:
    \[
    x = (x_1, x_2, \dots, x_n)
    \]
    jest elementem przestrzeni \(\mathbb{R}^n\), a \(x_i\) są jego współrzędnymi.
\end{de}

\begin{de}
    Dla \(x, y \in \mathbb{R}^n\) definiujemy iloczyn skalarny jako:
    \begin{equation*}
        \langle x, y \rangle = \sum_{i=1}^{n} x_i y_i,
    \end{equation*}
    gdzie \( x = (x_1, x_2, \dots, x_n) \) i \( y = (y_1, y_2, \dots, y_n) \).
    Dla dowolnego skalaru \( a \in \mathbb{R} \), mnożenie skalarne wektora \( x \) przez \( a \) jest zdefiniowane jako:
    \[
        ax = (ax_1, ax_2, \dots, ax_n).
    \]
    Norma (długość) wektora \( x \) jest dana wzorem:
    \[
        \|x\| = \sqrt{\langle x, x \rangle} = \sqrt{x_1^2 + x_2^2 + \dots + x_n^2}.
    \]
    Iloczyn skalarny spełnia następujące własności (2. dla dowolnego \( a \in \mathbb{R} \)):
    \begin{enumerate}
        \item Przemienność: \(\langle x, y \rangle = \langle y, x \rangle\).
        \item Dysocjatywność względem mnożenia przez skalar: \(\langle ax, y \rangle = \langle x, ay \rangle = a \langle x, y \rangle\) 
        \item Rozdzielność względem dodawania wektorów: \(\langle x + y, z \rangle = \langle x, z \rangle + \langle y, z \rangle\).
    \end{enumerate}
\end{de}

\begin{pk}
    Stwórzmy kilka prostych przestzeni:
    \begin{itemize}
        \item \(\mathbb{R}\): Przestrzeń jednowymiarowa, zwana także osią liczbową. Każdy punkt \(x \in \mathbb{R}\) jest liczbą rzeczywistą. Długość (moduł) liczby rzeczywistej \(x\) jest dana wzorem:
        \[
        |x| = |x_1|
        \]
        gdzie \(x_1\) to współrzędna punktu na osi liczbowej.
        
        \item \(\mathbb{R}^2\): Przestrzeń dwuwymiarowa, znana jako płaszczyzna kartezjańska. Każdy punkt \(x \in \mathbb{R}^2\) jest parą liczb rzeczywistych. Długość (norma) wektora \(x = (x_1, x_2)\) w tej przestrzeni jest dana wzorem:
        \[
        |x| = \sqrt{x_1^2 + x_2^2}
        \]
        gdzie \(x_1\) i \(x_2\) to współrzędne punktu na płaszczyźnie.
    \end{itemize}
\end{pk}

\begin{tw}
    Niech \( x,y \in \mathbb{R}^n \). Wówczas:
    \[
        |\langle x,y \rangle| \leq \|x\| \cdot \|y\|
    \]
    D-d. \( x=(x_1,x_2,\dots,x_n) \), \( y=(y_1,y_2,\dots, y_n) \). Z definicji iloczynu skalarnego:
    \[
        |\langle x,y \rangle| = \left| \sum_{i=1}^{n} x_i y_i \right| \leq \sqrt{\sum_{i=1}^{n} x_i^2} \cdot \sqrt{\sum_{i=1}^{n} y_i^2}
    \]
    Otrzymaliśmy nierówność Cauchy'ego-Schwarza, a zatem dowód.
\end{tw}
\begin{wn}
Nierówność Trójkąta $x,y\in\mathbb{R}^n$.
\begin{center}
	$|x+y| \leq |x| + |y|$
\end{center}
    \begin{dw}
        $|x+y|^2=\langle x+y, x+y \rangle$
        $= \langle x, x+y \rangle + \langle y, x+y \rangle$
        $= \langle x, x \rangle + \langle x, y \rangle + \langle y, y \rangle + \langle y, x \rangle$
        $=|x|^2+|y|^2 + 2\langle x,y \rangle \leq |x|^2 |y|^2 + 2|x||y|$
        $=(|x|+|y|)^2$
        $=|x+y|^2 \leq (|x|+|y|)^2$
        $\iff$
        $|x+y| \leq |x| + |y|\qed$
    \end{dw}
\end{wn}

\subsection{Kąt między wektorami}

Niech \( \overrightarrow{x_1}, \overrightarrow{x_2} \in \mathbb{R}^2 \),
\( \overrightarrow{x_1} = (x_{11}, x_{12}) \),
\( \overrightarrow{x_2} = (x_{21}, x_{22}) \).

\begin{center}
    \( \cos(\overrightarrow{x_1}, \overrightarrow{x_2}) = \frac{\overrightarrow{x_1} \odot \overrightarrow{x_2}}{\|\overrightarrow{x_1}\| \|\overrightarrow{x_2}\|} \)
\end{center}

\subsection{Metryka}
Rozważmy funkcję:
\[ d_n: \mathbb{R}^n \times \mathbb{R}^n \rightarrow \mathbb{R} \]
\(d_n\) jest metryką w \(\mathbb{R}^n\) jeśli spełnia następujące aksjomaty:
\begin{enumerate}
    \item \( d_n(x,y) \geq 0 \)
    \item \( d_n(x,y) = 0 \iff x=y \)
    \item \( d_n(x,y) = d_n(y,x) \)
    \item \( d_n(x,z) \leq d_n(x,y) + d_n(y,z) \) -- nierówność trójkąta
\end{enumerate}

\begin{pk}
    \( d_n(x,y) = |x-y| \).
    Dla \( n=2 \):
    \[
        d_2(x,y) = |x-y| = \sqrt{(x_1-y_1)^2 + (x_2-y_2)^2}
    \]
\end{pk}

\subsection{Przestrzeń metryczna}

\begin{de}
Przestrzenią metryczną nazywamy dowolny zbiór $X$, pewną funkcję $X\times X \rightarrow \mathbb{R}$,
która spełnia następujące aksjomaty:
\begin{enumerate}
    \item $d(x,y)\geq 0$
    \item $d(x,y)=0 \iff x=y$
    \item $d(x,y)=d(y,x)$ dla $x,y\in X$
    \item $d(x,z)\leq d(x,y) + d(y,z)$ dla każdych $x,y,z \in X$
\end{enumerate}
Funkcję $d$ nazywamy metryką, a wartość $d(x,y)$ odległością punktów.

\end{de}
\begin{uw}
    Aksjomat $1$ wynika z pozostałych aksjomatów.
\end{uw}
\(
    d(x,y) = \frac{1}{2} \left(d(x,y)+d(y,x)\right) \geq \frac{1}{2} d(x,x) = 0$, zatem $d(x,y)\geq 0
\)

\begin{tw}
    Niech $(X,d)$ będzie przestrzenią metryczną oraz $x_1,x_2,\dots,x_n \in X$. Wówczas:
    \[
    d(x_1,x_n) \leq \sum_{j=1}^{n-1} d(x_j,x_{j+1}), \quad n \geq 2
    \]
    Dla $n=2$:
    \[
    d(x_1,x_2) \leq d(x_1,x_2) \quad \text{- oczywiste}
    \]
    Dla $n=3$:
    \[
    d(x_1,x_3) \leq d(x_1,x_2) + d(x_2,x_3) \quad \text{- nierówność trójkąta}
    \]
    Krok indukcyjny:
    \[
    d(x_1,x_{n+1}) \leq d(x_1,x_2) + d(x_2,x_3) + \dots + d(x_n,x_{n+1})
    \]
    \[
    d(x_1,x_{n+1}) \leq d(x_1,x_n) + d(x_n,x_{n+1}) \leq_{ind}
    d(x_1,x_2) + d(x_2,x_3) + \dots + d(x_{n-1}, x_n) + d(x_{n},x_{n+1}) \qed
    \]
\end{tw}   

\subsection{Przestrzeń metryczna dyskretna}

Niech $X$ będzie dowolnym zbiorem, a metryka $d$ jest określona wzorem:
\[
d(x,y) = \begin{cases}
    0 & \text{dla } x=y, \\
    1 & \text{dla } x \neq y.
\end{cases}
\]
Przestrzeń \((X,d)\) jest przestrzenią metryczną ponieważ spełnia jej aksjomaty:
\begin{enumerate}
    \item Dla wszystkich \( x, y \in X \), \( d(x, y) \geq 0 \):
    
    - Jeśli \( x = y \), to \( d(x, y) = 0 \geq 0 \).\\
    - Jeśli \( x \neq y \), to \( d(x, y) = 1 \geq 0 \).
    
    Zatem w obu przypadkach \( d(x, y) \geq 0 \).
    
    \item \( d(x, y) = 0 \) wtedy i tylko wtedy, gdy \( x = y \):
    
    Jeśli \( d(x, y) = 0 \), to z definicji wiemy, że \( x = y \).

    Z drugiej strony, jeśli \( x = y \), to metryka może mieć tylko jedną wartość \( d(x, y) = 0 \).

    \item Symetria: \( d(x, y) = d(y, x) \):
    
    Ponieważ metryka dyskretna przyjmuje wartości 0 lub 1, to zarówno \( d(x, y) \) jak i \( d(y, x) \) są równe:\\
    - Jeśli \( x = y \), to \( d(x, y) = 0 \) oraz \( d(y, x) = 0 \).\\
    - Jeśli \( x \neq y \), to \( d(x, y) = 1 \) oraz \( d(y, x) = 1 \).
    
    Zatem w obu przypadkach \( d(x, y) = d(y, x) \).
    
    \item Nierówność trójkąta: \( d(x, z) \leq d(x, y) + d(y, z) \) dla wszystkich \( x, y, z \in X \):
    
    Rozważmy dowolne \( x, y, z \in X \).\\
    - Jeśli \( x = z \), to \( d(x, z) = 0 \), a \( d(x, y) + d(y, z) \geq 0 \). Nierówność trójkąta jest spełniona.
    
    - Jeśli \( x \neq z \), to istnieją dwa przypadki:
    \begin{itemize}
        \item Jeśli \( x = y \) lub \( y = z \), ale \( x \neq z \), to \( d(x, y) = 0 \) lub \( d(y, z) = 0 \), ale \( d(x, z) = 1 \). Wtedy \( d(x, y) + d(y, z) = 0 + 1 = 1 \geq 1 = d(x, z) \).
        
        \item Jeśli \( x \neq y \) i \( y \neq z \), to \( d(x, y) = 1 \) i \( d(y, z) = 1 \). Wtedy \( d(x, y) + d(y, z) = 1 + 1 = 2 \geq 1 = d(x, z) \).
    \end{itemize}
    
    W każdym przypadku nierówność trójkąta jest zachowana, co kończy dowód.
\end{enumerate}

\subsection{Metryka Euklidesowa}

Metryka Euklidesowa \( d_n: \mathbb{R}^n \times \mathbb{R}^n \rightarrow \mathbb{R} \) jest określona przez:
\[
d_n(x,y) = \sqrt{\sum_{i=1}^{n} (x_i - y_i)^2},
\]
gdzie \( x = (x_1, x_2, \dots, x_n) \) oraz \( y = (y_1, y_2, \dots, y_n) \).

\subsection{Przestrzeń Hilberta}

Niech \( x = (x_1, x_2, \dots, x_n) \) oraz \( y = (y_1, y_2, \dots, y_n) \) będą wektorami w przestrzeni Hilberta, gdzie:
\[
\sum_{i=1}^{\infty} x_i^2 < \infty, \quad \sum_{i=1}^{\infty} y_i^2 < \infty.
\]
Przykłady wektorów:
\[
x = \left(\frac{1}{1}, \frac{1}{2}, \frac{1}{3}, \dots, \frac{1}{i}, \dots \right), \quad \sum_{i=1}^{\infty} \frac{1}{i^2} < \infty,
\]
co oznacza, że \( x \) należy do przestrzeni Hilberta.\\
Natomiast wektor
\[
y = \left(\frac{1}{\sqrt{1}}, \frac{1}{\sqrt{2}}, \dots \right),
\]
spełnia warunek
\[
\sum_{i=1}^{\infty} \left(\frac{1}{\sqrt{i}}\right)^2 = \sum_{i=1}^{\infty} \frac{1}{i} = \infty,
\]
co oznacza, że \( y \) nie należy do przestrzeni Hilberta.

\subsection{Metryka Manhattan}

Metryka Manhattan między dwoma punktami \( (x_1, x_2) \) i \( (y_1, y_2) \) w przestrzeni \( \mathbb{R}^2 \) jest określona jako:
\[
d((x_1, x_2), (y_1, y_2)) = |x_1 - y_1| + |x_2 - y_2|.
\]
Metryka ta mierzy sumę bezwzględnych różnic współrzędnych między punktami.
Swoją nazwę zawdzięcza podobieństwu do liczenia odległości w miastach, gdzie trzeba poruszać się wzdłuż prostopadłych ulic i alei.

\section{Wykład II}

\subsection{Kula otwarta}

\begin{de}
Kula otwarta w przestrzeni metrycznej \( Y \):
\[
K(y_0, r) = \{ y \in Y \mid d(y, y_0) < r \}
\]
gdzie:
\begin{itemize}
    \item \( y_0 \) - środek kuli,
    \item \( r \) - promień kuli,
    \item \( d \) - funkcja metryczna określająca odległość między punktami w przestrzeni \( Y \).
\end{itemize}
\end{de}

\begin{pk}
    Rozważmy następujący przykład: \\
    \( K((x_0,y_0), r) \) to zbiór punktów \( (x,y) \) w przestrzeni \( \mathbb{R}^2 \), dla których zachodzi warunek:
    \[
    \sqrt{(x-x_0)^2 + (y-y_0)^2} < r.
    \]
    \begin{center}
        \begin{tikzpicture}
            \coordinate (center) at (0,0);
            \def\radius{0.5};
            \draw[thick, dotted] (center) circle (\radius);
            \filldraw[black] (center) circle (0.5pt) node[anchor=west] at ($(center) + (0.8, 0)$) {\((x_0, y_0)\)};
        \end{tikzpicture}
    \end{center}
\end{pk}

\begin{de}
    Niech \( (X,d) \) będzie przestrzenią metryczną. Zbiór \( U \subseteq X \) jest otwarty, jeśli dla każdego \( x \in U \) istnieje \( \varepsilon > 0 \) takie, że
    \[
    K(x,\varepsilon) = \{ y \in X \mid d(y,x) < \varepsilon \} \subseteq U.
    \]
    gdzie \( K(x,\varepsilon) \) oznacza kulę otwartą o środku w punkcie \( x \) i promieniu \( \varepsilon > 0 \)
\end{de}

\begin{pk}
    Przykłady: \\
    \( (a,b) \) - jest otwarty\\
    \( [a,b) \) - nie jest otwarty
\end{pk}

\begin{de}
    Niech \( (X,d) \) będzie przestrzenią metryczną. 
    \begin{center}
        Zbiór \( D \subseteq X \) jest zbiorem domkniętym \(\iff\) \( X - D \) jest otwarty.
    \end{center}
    Przykład: \( [a,b] \subset \mathbb{R} \implies \mathbb{R} - [a,b] = (-\infty, a) \cup (b, \infty) \) - zbiór otwarty.
\end{de}

\begin{de}
    Niech $(X,d_1)$ i $(Y, d_2)$ będą przestrzeniami metrycznymi, a $F: X \rightarrow Y$ będzie funkcją. 
    Granicą funkcji $F(x)$ oznaczymy:
    \[ 
    \lim_{x \rightarrow a} F(x) = b.
    \]
    gdzie $a \in X$ i $b \in Y$.\\
    Warunki (1), (2) są równoważne:
    \begin{enumerate}
        \item $\lim_{x\rightarrow a} F(x) = b$
        \item dla dowolnego ciągu $(x_n)_{n\geq 0}$ 
        punktów przestrzeni metrycznej $X(x_n \neq a)$\\
        jeśli $\lim_{n\rightarrow a} x_n = a$ w metryce $d_1$ to
        $\lim_{n\rightarrow \infty} F(x_n) = b$
    \end{enumerate}
    \[ \lim_{x \rightarrow a} F(x) = b \iff (\forall \varepsilon > 0) (\exists \delta > 0) (\forall x \in X) (0 < d_1(x,a) < \delta) \implies d_2(F(x),b) < \varepsilon \]
\end{de}

\begin{uw}
    Analogicznie dla granicy ciągów:
    \[
    \lim_{n \to \infty} x_n = a \iff (\forall \varepsilon > 0) (\exists n_0 \in \mathbb{N}) (\forall n > n_0) \, d_1(x_n,a) < \varepsilon
    \]
\end{uw}

\begin{pk}
    $x_n\in\mathbb{R}, a\in X$\\
    $d_1(x_n,a) \in \mathbb{R}$\\
    $\lim_{n\rightarrow \infty} x_n = a \iff \lim_{n\rightarrow\infty} d_1(x_n,a)=0$ w metryce $d_1$
\end{pk}

\begin{pk}
    $(a_n,b_n,c_n) \in \mathbb{R}^3$\\
    $\lim_{n\rightarrow \infty} (a_n,b_n,c_n) = (g_1,g_2,g_3)$ w metryce Eulidesowej $\mathbb{R}^3 \iff$\\
    $\lim_{n\rightarrow \infty} a_n = g_1 \land \lim_{n\rightarrow \infty} b_n = g_2 \land \lim_{n\rightarrow \infty} c_n = g_3$\\
    Idea:\\
    $\sqrt{(a_n-g_1)^2 + (b_n-g_2)^2 + (c_n-g_3)^2}\rightarrow 0 \iff a_n\rightarrow g_1 \land b_n \rightarrow g_2 \land c_n \rightarrow g_3$\\
    Dla $\mathbb{R}^k$ podane włansości zachodzą analogicznie.
\end{pk}

\begin{de}
    Ciągłość funkcji. Niech $(X,d_1)$, $(Y, d_2)$ będą przestrzeniami metrycznymi, oraz niech $F: X \rightarrow Y$.
    Funkcja $F$ jest ciągła w punkcie $a \in X$ jeśli zachodzi:
    \[
    \lim_{x \to a} F(x) = F(a)
    \]
    Innymi słowy, jeśli $x_n \to a$ w metryce $d_1$, to $F(x_n) \to F(a)$ w metryce $d_2$.
\end{de}

\begin{pk}
    Jak pokazać, że funkcja nie jest ciągła. Weźmy funkcję $f: \mathbb{R}^2 \rightarrow \mathbb{R}$.
    \[
    f(x,y)=\begin{cases}
        \frac{xy}{x^2+y^2} & \text{dla } (x,y) \neq (0,0) \\
        0 & \text{dla } (x,y) = (0,0)
    \end{cases}
    \]
    Pokażmy, że $f$ nie jest ciągła w $(0,0)$.
    Rozważmy ciąg $(x_n, y_n) = \left( \frac{1}{n}, \frac{1}{n} \right) \rightarrow (0,0)$.
    Obliczmy granicę $\lim_{n\rightarrow \infty} f(x_n, y_n)$:
    \[
    f\left( \frac{1}{n}, \frac{1}{n} \right) = \frac{\frac{1}{n} \cdot \frac{1}{n}}{\left( \frac{1}{n} \right)^2 + \left( \frac{1}{n} \right)^2} = \frac{\frac{1}{n^2}}{\frac{2}{n^2}} = \frac{1}{2}
    \]
    Zatem $\lim_{n\rightarrow \infty} f(x_n, y_n) = \frac{1}{2}$, co nie jest równe $0$.
    Stąd funkcja $f$ nie jest ciągła w punkcie $(0,0)$.
\end{pk}

\begin{pk}
    Rozważmy funkcję $f: \mathbb{R}^3 \rightarrow \mathbb{R}^2$ określoną przez
    \[
    f(x,y,z) = (x^2, y^2 \cdot z)
    \]
    Zbadajmy ciąg $a=(x_0, y_0, z_0)$, gdzie
    \[
    f(x_0, y_0, z_0) = (x_0^2, y_0^2 \cdot z_0)
    \]
    Niech $(x_n, y_n, z_n) \rightarrow (x_0, y_0, z_0)$. Oznacza to, że $x_n \rightarrow x_0$, $y_n \rightarrow y_0$ oraz $z_n \rightarrow z_0$.
    \[
    \lim_{(x_n, y_n, z_n) \rightarrow (x_0, y_0, z_0)} f(x_n,y_n,z_n) = (x_0^2, y_0^2 \cdot z_0)
    \]
\end{pk}

\begin{pk}
    $$f(x,y)=\begin{cases}
        \frac{xy^2}{x^2+y^4} \text{ dla } x^2 + y^4 > 0\\
        0 \text { dla } (x,y) = (0,0)
    \end{cases}$$
    $f$ jest ciągła w $(0,0)$ $(\alpha t, \beta t) \rightarrow (0,0)$\\
    $f(\alpha t, \beta t) = \frac{\alpha \beta^2 t^3}{\alpha^2 t^2 + \beta^4 t^4}=$\\
    $\frac{\alpha \beta^2 t}{\alpha^2 + \beta^4 t^2} \rightarrow \frac{0}{\alpha^2} = 0 = f(0,0)$\\
    $\lim_{t\rightarrow 0} f(t^2,t) = \lim_{t\rightarrow 0} \frac{t^2 t^2}{t^4 + t^4} = \frac{1}{2}$\\
    $f$ nie jest ciągła $0=f(0,0)\neq \frac{1}{2}$, czyli nie tylko liniowa ale też dowolna\\\\
    Kolejny przykład obalający dla zdef. funkcji $\left(\frac{1}{n^2} , \frac{1}{n}\right) \rightarrow (0,0)$, ale już\\
    $f\left(\frac{1}{n^2} + \frac{1}{n}\right) = - \frac{1}{2} \neq f(0,0)$
\end{pk}

\begin{pk}
    $f: \mathbb{R} \times \mathbb{R} \rightarrow \mathbb{R}$
    $f(x,y) = x+y$\\
    $g: \mathbb{R} \times \mathbb{R} \rightarrow \mathbb{R}$
    $g(x,y) = xy$\\\\
    $(x_n,y_n) \rightarrow (x_0, y_0) \iff (x_n\rightarrow x_0 \land y_n \rightarrow y_0)$\\
    $\lim_{n\rightarrow \infty} f(x_n, y_n) = \lim_{n\rightarrow \infty} (x_n + y_n) = x_0 + y_0 = f(x_0, y_0)$\\
    $\lim_{n\rightarrow \infty} g(x_n, y_n) = \lim_{n\rightarrow \infty} (x_n, y_n) = x_0 y_0 = g(x_0, y_0)$\\
\end{pk}

\begin{pk}
    $\lim_{x\rightarrow 0} (\lim_{y\rightarrow 0} f(x,y)) =$\\
    $\lim_{x\rightarrow 0} (\lim_{y\rightarrow 0} \frac{xy}{x^2+y^2}) = \lim_{x\rightarrow 0} (0) = 0$
    $\lim_{y\rightarrow 0} (\lim_{x\rightarrow 0} f(x,y)) =$\\
    $\lim_{y\rightarrow 0} (\lim_{x\rightarrow 0} \frac{xy}{x^2+y^2})$\\
    Nie istnieje\\
    $(x_n', y_n') = (\frac{1}{n}, \frac{1}{n}) \rightarrow (0,0)$\\
    $(x_n'', y_n'') = (-\frac{1}{n}, \frac{1}{n}) \rightarrow (0,0)$\\
    $f(x_n', y_n') = \frac{1/n 1/n}{(1/n)^2 + (1/n)^2} = 1/2$\\
    $f(x_n'', y_n'') = \frac{-\frac{1}{n}\frac{1}{n}}{(1/n)^2 + (1/n)^2}=-1/2$\\
    Ergo rozbieżny - granica podwójna nie istnieje.
\end{pk}

\subsection{Granica podwójna}

\begin{de}
    $\lim_{(x,y)\rightarrow(x_0,y_0) f(x,y)}$
\end{de}

\subsection{Granice iterowane}

\begin{de}
    $\lim_{x\rightarrow x_0} (\lim_{y\rightarrow y_n} f(x,y))$\\
    $\lim_{y\rightarrow y_0} (\lim_{x\rightarrow x_n} f(x,y))$
\end{de}

\subsection{Różniczkowanie}

$f: \mathbb{R} \rightarrow \mathbb{R}$:
$f'(x) = a \iff \lim_{h\rightarrow 0} \frac{f(x+h)-f(x)}{h}=a$\\
$\lim_{h\rightarrow 0} \frac{f(x+h)-f(x)}{h} - a = 0$\\
$\lim_{h\rightarrow 0} \frac{f(x+h)-f(x)-ah}{h} = 0$\\
$\lim_{h\rightarrow 0} \left|\frac{f(x+h)-f(x)-ah}{h}\right| = 0$\\
$\lim_{h\rightarrow 0} \frac{|f(x+h)-f(x)-ah|}{|h|} = 0$\\\\
$L: \mathbb{R} \rightarrow \mathbb{R}, L(h)= ah, h\in\mathbb{R}$\\
$L(h_1+h_2) = L(h_1) + L(h_2)$\\
$L(ch)=c\cdot L(h)$ ($L$ jest odwzorowaniem liniowym)\\

\begin{de}
    (Pochodna funkcji) $n, m \in \mathbb{N}-\{0\}$, $f: \mathbb{R}^n \rightarrow \mathbb{R}^n, x \in \mathbb{R}^n$\\
    Mówimy że funkcja $f$ jest różniczkowalna w punkcie $x$\\
    jeśli istnieje odwzorowaniem liniowe $f'(x): \mathbb{R}^n \rightarrow \mathbb{R}^m$\\
    takie że $h\in\mathbb{R}^n 0_n = (0,0,\dots, 0)$

    \begin{center}
        $\lim_{h\rightarrow 0} \frac{||f(x+h)-f(x)-f'(x)(h)||}{||h||} = 0_{\mathbb{R}}$
    \end{center}
\end{de}


\section{Wykład III}

\subsection{Pochodne cząstkowe}

\begin{de}
$f: \mathbb{R}^2\rightarrow\mathbb{R}$
    \begin{center}
    \[\frac{\partial f}{\partial x} (x_0,y_0) = \lim_{h\rightarrow 0} \frac{f(x_0+h,y_0)-f(x_0,y_0)}{h}\]
    \[\frac{\partial f}{\partial y} (x_0,y_0) = \lim_{h\rightarrow 0} \frac{f(x_0,y_0+h)-f(x_0,y_0)}{h}\]
    \end{center}
\end{de}

\begin{pk}
    Policzmy następującą pochodne cząstkowe dla funkcji:\\
    $f(x,y)=x\cdot y^2, f: \mathbb{R}^2\rightarrow \mathbb{R}$\\
    $\frac{\partial}{\partial x} (xy^2) = y^2$\\
    $\frac{\partial}{\partial y} (xy^2) = 2xy$\\
\end{pk}

\begin{de}
    $f:\mathbb{R}^n \rightarrow \mathbb{R}^m, x \in \mathbb{R}^n$ jest różniczkowalna w $x$ jeśli istnieje odwzorowanie liniowe:
        $f'(x):\mathbb{R}^n\rightarrow\mathbb{R}^m$ taka, że:
    \begin{center}
        \[\lim_{h\rightarrow 0} \frac{f(x+h)-f(x)-f'(x)(h)}{|h|}=0\]
    \end{center}
\end{de}

\begin{tw}
    Zakładamy, że $f:\mathbb{R}^n\rightarrow\mathbb{R}^m$ jest różniczkowalna. 
    Niech: $f=(f_1,f_2,\dots,f_m), f_{i}: \mathbb{R}^n\rightarrow\mathbb{R}, i = 1,2,\dots,m$:
    $a_{ij} = \frac{\partial f_i}{\partial x_j} (x), j = 1,2,\dots,n$. Wówczas macierz pochodnej wynosi:
    \begin{center}
        \[M_{f'(x)}=\begin{bmatrix}
            a_{1,1} & a_{1,2} & \dots & a_{1,n} \\
            a_{2,1} & a_{2,2} & \dots & a_{2,n} \\
            \dots & \dots & \dots & \dots \\
            a_{m,1} & a_{m,2} & \dots & a_{m,n} \\
            \end{bmatrix}\]
    \end{center}
\end{tw}

Uwaga. Istnienie wszystkich pochodnych cząstkowych nie wystarcza aby funkcja była różniczkowalna.
$$
f(x,y)=\begin{cases} 
    \frac{x^2y}{x^2+y^2} \text{ dla } x^2+y^2>0\\
    0 \text{ dla } (x,y)=(0,0) 
\end{cases}
$$
Cel. pokażmy że $f$ nie jest różniczkowalna w $(0,0)$\\
$\frac{\partial f}{\partial x} (0,0) = \lim_{h\rightarrow 0} \frac{f(h,0)-f(0,0)}{h} = \lim_{h\rightarrow 0} \frac{0}{h} = 0$\\
$\frac{\partial f}{\partial y} (0,0) = 0$\\
Kandydat na pochodną:
$$
M_{f'(0,0)} = \left[\frac{\partial f}{\partial x} (0,0),\frac{\partial f}{\partial y} (0,0)\right]
$$
Warunek różniczkowania: $h=(h_1,h_2)$\\
$$
\lim_{(h_1,h_2)\rightarrow(0,0)}
\frac{\left|f(h_1,h_2) - f(0,0) - [0,0]
    \left[\begin{matrix}
    h_1\\h_2\end{matrix}\right]
    \right|}{\sqrt{h_1^2 + h_2^2}} = 0
$$ (??)
$
\lim_{(h_1,h_2)\rightarrow (0,0)} \frac{|h_1^2 h_2|}{(h_1^2+h_2^2)^{\frac{3}{2}}}=
\lim_{n\rightarrow \infty} \frac{\left(\frac{1}{n}\right)^2\frac{1}{n}}{\left((\frac{1}{n})^22\right)^{\frac{3}{2}}}=0
$

\begin{tw}
    Tw. $f: \mathbb{R}^n \rightarrow \mathbb{R}^m, x\in\mathbb{R}^n$. Zakładamy, że pochodne cząstkowe: $\frac{\partial f_i}{\partial x_j} (x)$
    istnieją w otoczeniu punktu $x$ i są ciągłe w punkcie $x$. Wtedy $f$ jest różniczkowalna w punkcie $x$.
\end{tw}

\begin{pk}
    $f:\mathbb{R}^n \rightarrow \mathbb{R}$ $f(x)=\langle x, x \rangle$\\
    $x=(x_1,x_2,\dots,x_n)$\\
    $f(x)=x_1^2+x_2^2+\dots+x_n^2$\\
    $\frac{\partial f}{\partial x_1} = 2x_1, \frac{\partial f}{\partial x_2} = 2x_2, \dots$\\
    $$M_{f'(x)} = \left[2x_1, 2x_2, \dots, 2x_n\right]=2x$$
    $$M_{f'(x)}(h) = 2\langle x,h\rangle$$
\end{pk}

\begin{pk}
    Z definicji $\frac{\left|f(x+h)-f(x)-M_{f'(x)}(h) \right|}{|h|}=\frac{\langle h,h \rangle}{|h|} = \frac{|h||h|}{|h|}$\\
    Jednak algebraicznie $f(x+h)-f(x)-M_{f'(x)}(h)=\langle x+h,x+h \rangle - \langle x, x \rangle - 2 \langle x, h \rangle=\langle h,h \rangle$
\end{pk}

\begin{pk}
    $f:\mathbb{R}\rightarrow \mathbb{R}^2, f(t)=(\sin(t),\cos(t))$\\
    $$
    M_{f'(t)} = \left[
        \begin{matrix}
        sin(t)'\\
        cos(t)'
        \end{matrix}
    \right]=\left[
        \begin{matrix}
            cos(t)\\
            -sin(t)
        \end{matrix}
    \right]
    $$
\end{pk}

\begin{de}
    Pochodne kierunkowe: $f: \mathbb{R}^n\rightarrow \mathbb{R}$. 
    Pochodną kierunkową funkcji $f$ w punkcie $x_0$ 
    w kierunku wektora $\overline{a}$ nazywamy granicę:
    \begin{center}
        \[(D_a f)(x_0)=\lim_{t\rightarrow 0} \frac{f(x_0 + at) - f(x_0)}{t}\]
    \end{center}
    $\varphi(t)=f(x_0+at)$\\
    $\varphi'(0)=\lim_{h\rightarrow 0} \frac{\varphi(h)-\varphi(0)}{h}=\lim_{h\rightarrow 0} \frac{f(x_0+ah)-f(x_0)}{h}=(D_a f)(x_0)$
\end{de}

\begin{pk}
    $f(x,y) = sin(x)\cdot y$\\
    $x_0 = (0,0)$\\
    $a=(\frac{\sqrt{2}}{2},\frac{\sqrt{2}}{2})$\\
    \[D_a f (0,0) = \lim_{t\rightarrow 0} \frac{f\left(\frac{\sqrt{2}}{2}t,\frac{\sqrt{2}}{2}\right)-f(0,0)}{t}=\lim_{t\rightarrow 0} \frac{\sin(\frac{\sqrt{2}}{2})}{t}=\frac{\sqrt{2}}{2}\]
\end{pk}

\begin{tw}
    Jeżeli $f$ jest różniczkowalna w punkcie $x_0$ oraz $a\in\mathbb{R}-\{0\}$, to:
    \[(D_a f)(x_0) = f'(x_0) a^{T}\]
    \[f'(x_0)=\left[\frac{\partial f}{\partial x_1} (x_0),\dots,\frac{\partial f}{\partial x_n} (x_n)\right]\]
    \[f'(x_0)a = \frac{\partial f}{\partial x_1}(x_0)a_1 + \dots + \frac{\partial f}{\partial x_n} (x_0) a_n\]
\end{tw}

\begin{de}
    Gradient funkcji $f: \mathbb{R}^n\rightarrow\mathbb{R}$ w punkcie $x_0\in\mathbb{R}^n$:
    \[\nabla_{x_0} f = \left(\frac{\partial f}{\partial x_1} (x_0), \frac{\partial f}{\partial x_2} (x_0),\dots, \frac{\partial f}{\partial x_n} (x_0)\right)\]
    \[(D_a f) (x_0) = \langle \nabla_{x_0} f, a \rangle\]
\end{de}

\begin{pk}
    Policzmy gradient dla: $f(x,y)=x^2+y^2$.
    \[\nabla_(x_0,y_0) f(x,y) = (2x_0, 2y_0)\]
    \[\nabla f(x,y) = (2x,2y)\]
\end{pk}
D-d. Zakładamy, że $f$ jest różniczkowalna w $x_0$. 
\[0=\lim_{h\rightarrow 0} \frac{|f(x+h)-f(x_0)-f'(x_0)h|}{|h|}=\]
\[=\lim_{h\rightarrow 0} \frac{1}{|a|} \left|\frac{f(x_0+a)-f(x_0)-f'(x_0)t(ta)}{t}\right|=\]
\[=\lim_{t\rightarrow 0} \frac{1}{|a|} \left|\frac{f(x_0+ta)-f(x_0)}{t}-f'(x_0)(a)\right|\]

\subsection{Gradient}

$f:\mathbb{R}^n \rightarrow \mathbb{R}$

\[\nabla_{x_0} f = \left(\frac{\partial f}{\partial x_1} (x_0), \dots, \frac{\partial f}{\partial x_n} (x_0)\right)\]

\subsection{Własności gradientu}

Niech $\alpha \in \mathbb{R}$
\begin{enumerate}
    \item $\nabla_{x_0} (\alpha f) = \alpha \nabla_{x_0} (f)$
    \item $\nabla_{x_0} (f+g) = \nabla{x_0} (f) + \nabla{x_0} (g)$
\end{enumerate}
Wniosek liniowość (dla $\alpha,\beta \in \mathbb{R}$):
\[\nabla_{x_0} (\alpha f + \beta g) = \alpha \nabla_{x_0} (f) + \beta \nabla_{x_0} (g)\]

\subsection{Gradient iloczynu}
\[\nabla_{x_0} (f\cdot g) = \nabla_{x_0} (f) \cdot g(x_0) + f(x_0) \nabla_{x_0} (g)\]
D-d. z liniowości gradientu.

\begin{tw}
Zakładamy, że $f: \mathbb{R}^n \rightarrow \mathbb{R}^m$ jest różniczkowalna w $x\in\mathbb{R}^n$.
Wówczas $f$ jest ciągła w $x\in\mathbb{R}^n$.\\\\
D-d. (metryka w $\mathbb{R}^n$)\\
\[|f(x+h)-f(x)|=\]
\[=|f(x+h)-f(x)-f'(x)h + f'(x)h|\]
\[\leq |f(x+h)-f(x)-f'(x)h|+|f'(x)h|=\]
\[=\frac{|f(x+h)-f(x)-f'(x)h|}{|h|} |h| = |f'(x)h|\]
\end{tw}

\subsection{Minimum lokalne właściwe}

\begin{de}
    $f:\mathbb{R}^n \rightarrow \mathbb{R}, a\in\mathbb{R}^n$. 
    Mówimy, że $f$ ma w punkcie $a$ minimum lokalne (właściwe)
    jeśli istnieje $r>0$:
    \[\left(\forall x\in K(a,r) - \{a\}\right) f(a) \leq_{(<)} f(x)\] 
    Gdzie $K(a,r)$ - kula towarta o środku w $a$ i promieniu $r$.
\end{de}

\begin{tw}
    (Warunek konieczny) $f: \mathbb{R}^n \rightarrow \mathbb{R}, a\in\mathbb{R}^n$.
    Jeżeli $f$ ma w punkcie $a$ ekstremum lokalne to:
    \[\nabla_{x_0} f = (0,0,\dots, 0)\]
    Inaczej:
    $\frac{\partial f}{\partial x_1} (a) = \dots = \frac{\partial f}{\partial x_n} (a) = 0$
\end{tw}

\begin{pk}
    Rozważmy następujące przykłady:
    \begin{enumerate}
        \item $f(x,y)=x^2-y^2$ - sprawdźmy warunek konieczny\\
        $\frac{\partial f}{\partial x} = 2x = 0, \frac{\partial f}{\partial y} = 2y = 0$\\
        $(0,0)$ - punkt podejrzany\\
        $f(\frac{1}{n},0) = \frac{1}{n^2} > 0$
        $f(0, \frac{1}{n}) = -\frac{1}{n^2} < 0$\\
        Nie istnieje taka $K((0,0),r)$, że $f$ ma w tej kuli stały znak.
        \item $f(x)=x^3$, $f\left(\frac{1}{n}\right) = \frac{1}{n^3}$, $f\left(-\frac{1}{n}\right)\dots$
        \item $f(x,y,z) = -x^2 - (y-1)^2 - (z+1)^2$\\
        Sprawdźmy warunek: $\nabla f = (-2x, -2(y-1), -2(z+1))=(0,0,0)$\\
        Zobaczmy $f(0,1,-1)=0$\\
        $f(a,b+1,c-1) - f(0,1,-1) = -a^2 -b^2 -c^2 = -(a^2+b^2+c^2) < 0$ maksimum lokalne
        $f(a,b+1,c-1) < f(0,1,-1)$ (liczby sześcianu opisane na kuli)\\
    \end{enumerate}
\end{pk}

\begin{de}
    Niech $f(x,y): \mathbb{R}^2 \rightarrow \mathbb{R}$ $x,y\in\mathbb{R}$.
    Zakładamy, że $\frac{\partial f(x,y)}{\partial x} = f_x(x,y)$ oraz $\frac{\partial f(x,y)}{\partial y} = f_y(x,y)$.
    Wtedy pochodne cząstkowe pochodnych $f_x(x,y), f_y(x,y)$ nazywamy pochodnymi cząstkowymi drugiego rzędu:

    \begin{enumerate}
        \item $\frac{\partial^2 f}{\partial x^2} = \frac{\partial}{\partial x} \left(\frac{\partial f}{\partial x}\right) = f_{xx} (x,y)$
        \item $\frac{\partial^2 f}{\partial y^2} = \frac{\partial}{\partial y} \left(\frac{\partial f}{\partial y}\right) = f_{yy} (x,y)$
        \item $\frac{\partial^2 f}{\partial x \partial y} = \frac{\partial}{\partial x} \left(\frac{\partial f}{\partial y}\right) = f_{xy} (x,y)$
        \item $\frac{\partial^2 f}{\partial y \partial x} = \frac{\partial}{\partial y} \left(\frac{\partial f}{\partial x}\right) = f_{yx} (x,y)$
    \end{enumerate}
\end{de}

\begin{tw}
  Jeżeli pochodne cząstkowe istnieją w pewnym obszarze i obie są, w pewnym punkcie ciągłe, to w tym punkcie są równe
  \[ f_{xy} = f_{yx} \]
\end{tw}

\[\frac{\partial^{p+q} f}{\partial x^p \partial y^q}\]\\

$\frac{\partial^3 f}{\partial x^2 \partial y}$
$f(x,y) = x^3 y^2$\\

\[\frac{d}{dy} (x^3 y^2) = x^3 2y\]
\[\frac{d}{dx} (x^3 2y) = 3x^2 2y\]
\[\frac{d}{dx} (3x^2 2y) = 6x 2y\]

\[D\left[x^3 \cdot y^2, (x,2), (y,1)\right] \text {(wolframalpha)}\]

\subsection{Różniczkowanie złożenia funkcji}

\begin{tw}
    $f:\mathbb{R}^n \rightarrow \mathbb{R}^m, g: \mathbb{R}^m \rightarrow \mathbb{R}^k$ 
    $a\in\mathbb{R}^n, b=f(a)\in \mathbb{R}^m$. Zakładamy, że $f$ jest różniczkowalna w punkcie $a$
    oraz $g$ jest różniczkowalna w punkcie $b$. Wtedy $g\circ f$ jest różniczkowalna w punkcie $a$ i zachodzi wzór:
    \[(g\circ f)(a) = g'\left(f(a)\right) \circ f'(a)\]
    Złożenie odwzorowań liniowych - mnożenie macierzy.
\end{tw}

\begin{pk}
    $U: \mathbb{R} \rightarrow \mathbb{R}^3, f: \mathbb{R}^3 \rightarrow \mathbb{R}, g=f\circ u : \mathbb{R} \rightarrow \mathbb{R}\\$
    $u(t) = \left(u_1 (t), u_2 (t), u_3 (t)\right)$\\
    $g(t) = (f\circ u) (t) = f(u_1(t),u_2(t),u_3(t))$\\
    Zobaczmy:

    $M_{u'(t)} = \begin{bmatrix}
        u_1'(t) \\
        u_2'(t) \\
        u_3'(t)
    \end{bmatrix}  
    $
    $M_{f'(b)} = \left(\frac{\partial f}{\partial x_1}, \frac{\partial f}{\partial x_2}, \frac{\partial f}{\partial x_3}\right)_{x=b}$\\
    Wykonajmy mnożenie macierzy:
    \[M_{f'(b)}\cdot M_{u'(b)} = \frac{\partial f}{\partial x_1} \frac{d u_1}{dt} + \frac{\partial f}{\partial x_2} \frac{d u_2}{dt} + \frac{\partial f}{\partial x_3} \frac{d u_3}{dt}\]
\end{pk}

\begin{fakt}
    Uogólniona reguła łańcuchowa. ($x_i = u_i(t)$)
    \[\frac{d}{dt} f\left(u_1(t), u_2(t), \dots, u_n(t)\right) = \sum_{i=1}^{n} \frac{\partial f}{\partial x_i} \frac{d u_i}{dt}\]
\end{fakt}

\begin{pk}
    Niech $u(t) = (t^2-t, 2t, 4t), f(x,y,z)=xy+z, g=f\circ u, g(t)=(t^2-t)2t+4t$\\\\
    $g'(t) = (t^2-1)'2t + (t^2-t)(2t)' + (4t)' = (2t-1)(2t) + 2(t^2-t) + 4=$\\
    $= 4t^2 - 2t + 2t^2 - 2t + 4 = 6t^2 - 4t + 4$\\
    Zobaczmy z reguły łańcuchowej:\\
    \[g(t)=(f\circ u)(t)\]
    \[g'(t)=\frac{\partial f}{\partial x_1} \frac{du_1}{dt} + \frac{\partial f}{\partial x_2} \frac{du_2}{dt} + \frac{\partial f}{\partial x_3} \frac{du_3}{dt}\]
    \[y\cdot (2t-1) + 2x + 1\cdot 4 = 2t (2t-1) + 2(t^2-t) + 4 = 4t^2 - 2t + 2t^2 - 2t + 4 = 6t^2 - 4t + 4\]
\end{pk}
W zastosowaniu algorytmu back propagation.

\section{Wykład IV}

$f(x)=\sqrt{x^2+\exp(x^2)} + \cos(x^2+\exp(x^2))$\\
$\frac{df}{dx}$ ... można policzyć
Graf, jakby liczył komputer:
\begin{align}
    (x) \rightarrow (\text{ })^2 \rightarrow &a \rightarrow +\\
    & a\rightarrow exp(\text{ }) \rightarrow b \rightarrow + \rightarrow c \rightarrow \sqrt{\text{ }}\rightarrow d \rightarrow + \rightarrow f\\
    & c \rightarrow cos(\text{ }) \rightarrow e \rightarrow +
\end{align}
Rozpiszmy $a=x^2, b=\exp(a), c = a+b, d=\sqrt{c}, e=\cos(c), f=d+e$

$\frac{\partial a}{\partial x} = 2x$,
$\frac{\partial b}{\partial a} = \exp(a)$,
$\frac{\partial c}{\partial a} = 1 = \frac{\partial c}{\partial b}$,
$\frac{\partial d}{\partial c} = \frac{1}{2\sqrt{c}}$,
$\frac{\partial e}{\partial c} = -\sin(c)$,
$\frac{\partial f}{\partial d} = 1 \frac{\partial f}{\partial e}$,
\[\frac{\partial f}{\partial b} = \frac{\partial f}{\partial c} + \frac{\partial c}{\partial b}\]
\[\frac{\partial f}{\partial c} = \frac{\partial f}{\partial d}\frac{\partial d}{\partial c} + \frac{\partial f}{\partial e}\frac{\partial e}{\partial c}\]
\[\frac{\partial f}{\partial a} = \frac{\partial f}{\partial b}\frac{\partial b}{\partial a} + \frac{\partial f}{\partial c}\frac{\partial c}{\partial a}\]

\begin{pk}
    Problem. $f: \mathbb{R}^n \rightarrow \mathbb{R}$ przyzwoita - różniczkowalna co najmniej 2 razy, jakie jest maksimum lokalne.
    \begin{enumerate}
        \item $n=1$ analiza 1.
        \item $n=2$
        \item $n\geq 2$
    \end{enumerate}
\end{pk}

\begin{tw}
    Niech $f(x,y)$ ma w otoczeniu punktu $(x_0,y_0)$ pierwsze i drugie pochodne cząstkowe ciągłe oraz $f_x(x_0,y_0)=f_y(x_0,y_0)$. Wtedy:
    \[W(x_0,y_0)= \begin{bmatrix}
        f_{xx} (x_0,y_0)  \text{  }  f_{xy} (x_0,y_0) \\
        f_{xy} (x_0,y_0)  \text{  }  f_{yy} (x_0,y_0) 
        \end{bmatrix}  \]
    Z ciągłości $f_{xy} (x_0,y_0) = f_{yx} (x_0,y_0)$
\end{tw}

\begin{enumerate}
    \item Jeśli $W(x_0, y_0) > 0$ i $f_{xx} (x_0,y_0) > 0$, to $f$ ma w punkcie $(x_0,y_0)$ minimum lokalne
    \item Jeśli $W(x_0, y_0) > 0$ i $f_{xx} (x_0,y_0) < 0$, to $f$ ma w punkcie $(x_0,y_0)$ maksimum lokalne
    \item Jeśli $W(x_0, y_0) < 0$ to $f$ nie ma ekstremum lokalnego w $(x_0,y_0)$
    \item Jeśli $W(x_0, y_0) = 0$ kryterium nie działa
    \item (??) $W(x_0, y_0) > 0, f_{xx} (x_0, y_0) = 0$, -$f_{xy}^2 (x_0,y_0) \leq 0$, sprzeczność (z $\det$ macierzy)
\end{enumerate}
Jak to liczyć w wolframalpha

\begin{enumerate}
    \item solve(grad(f(x,y), {x,y})=0, (x,y)), (Hessian)
    \item det(...)
\end{enumerate}

\begin{pk}
    Rozważmy przykład \\
    \[ f(x,y) = (2x+y^2)e^x \]
    Policzmy najpierw pierwsze pochodne cząstkowe:
    \[ f_x(x,y) = 2e^x (2x+y^2)e^x\]
    \[ f_y(x,y) = 2y e^x\]
    Rozwiążmy powyższe równianie:
    \[ y = 0, x = -1\]
    Punkt podejrzany o ekstremum $P=(-1,0)$\\
    Policzmy drugie pochodne cząstkowe:
    \[ f_{xx} = e^x (4 + 2x + y^2) \]
    \[ f_{yy} = 2 e^x\]
    \[ f_{xy} = f_{yx} = 2y e^x\]
    Zobaczmy:
    \[ f_{xx} (-1,0) = 2e^{-1}\]
    \[ f_{yy} (-1,0) = 2e^{-1}\]
    \[ f_{xy} (-1,0) = f_{yx} (-1,0) = 0 \]
    Policzmy wyznacznik z powyższego twierdzenia:
    \[ W(-1, 0) = 4e^2 > 0, f_{xx} > 0 \]
    Wobec tego finalnie $f$ ma w punkcie $(-1,0)$ minimum lokalne.
\end{pk}

\begin{pk}
    Zobaczmy następny przykład:
    \[ f(x,y) = x^3 + y^3\]
    Policzmy następujące pochodne cząstkowe:
    \[ f_x(x,y)=3x^2, f_y(x,y)=3y^2, f_{xy}=0, f_{xx}(x,y) = 6x ,f_{yy}(x,y) = 6y\]
    Wyznacznik macierzy:
    \[ \det(W(0,0)) = 0\]
    Kryterium nie działa. Nic nam nie powie.
\end{pk}

\begin{fakt}
Wzór Taylora. Analiza I : $f:\mathbb{R} \rightarrow \mathbb{R}$
\[f(b+x) = f(b) + \frac{1}{1!} f'(b)x + \frac{1}{2!} f'(b) x^2 + \dots + \frac{1}{n!} f^{(n)}(b) x^n + R_n\]
\[R_n = \frac{1}{(n+1)!} f^{(n+1)} (b+\theta x) = x^{(n+1)} \]
dla pewnego $\theta\in(0,1)$\\
\end{fakt}

\begin{fakt}
Uogólnienie wzoru Taylora : $\mathbb{R}^n \rightarrow \mathbb{R}$\\
$a=(a_1,a_2,\dots,a_n)\in\mathbb{R}^n$\\ $x=(x_1,x_2,\dots, x_n)\in\mathbb{R}^n$
\[\varphi(t) = f(a+tx) = f(a_1+tx_1,a_2+tx_2,\dots,a_n+tx_n)\]
Dla funkcji $\varphi(t)$ stosujemy wzór Taylora z $b=0$ i $x=t$:
\[\varphi(t) = \varphi(0) + \frac{1}{1!} \varphi'(0)t + \frac{1}{2!} \varphi''(0)t^2 + \dots + \frac{1}{n!} \varphi^{(n)}(0) t^{n} + R_n\]
\[R_n = \frac{1}{(n+1)!} \varphi^{(n+1)} (0+\theta t) t^{n+1}\]
Zapiszmy $\varphi(0)$:
\[\varphi(0) = f(a_1,a_2,\dots, a_n)\]
\[\varphi'(t)(a_1+tx_1,a_2+tx_2,\dots,a_n+tx_n)=\]
\[=\frac{\partial f}{\partial x_1}(a_1+tx_1,\dots,a_n+tx_n) + \frac{\partial f}{\partial x_2}(a_1+tx_1,\dots,a_n+tx_n) + \dots + \frac{\partial f}{\partial x_n}(a_1+tx_1,\dots,a_n+tx_n)=\]
\[\varphi'(t) = \frac{\partial f}{\partial x_1} x_1 + \dots + \frac{\partial f}{\partial x_n} x_n\]
Wprowadźmy $c(t)=\frac{\partial f}{\partial x_1}(a_1+tx_1,\dots,a_n+tx_n)$. Zachodzi:
\[c'(t) = \frac{\partial^2 f}{\partial x_1 \partial x_1} x_1^2 + \frac{\partial^2 f}{\partial x_2 \partial x_1} x_2 x_1 + \dots + \frac{\partial^2 f}{\partial x_n \partial x_1} x_n x_1\]
Zatem patrząc na $\varphi''(t)$:
\[\varphi''(t)=\sum_{j=1}^{n}\sum_{i=1}^{n} \frac{\partial^2 f}{\partial x_j \partial x_i} x_j x_i\]
Oznaczenie:
\[\left(\frac{\partial}{\partial x_1} x_1 + \frac{\partial}{\partial x_2} x_2 + \dots + \frac{\partial}{\partial x_n} x_n\right) \left(f\right) = \frac{\partial f}{\partial x_1} x_1 + \dots + \frac{\partial f}{\partial x_n} x_n \]
\[\left\{\frac{\partial}{\partial x_1} x_1 + \frac{\partial}{\partial x_2} x_2\right\}^2 = \frac{\partial}{\partial x_1} x_1 \frac{\partial}{\partial x_1} x_1 + \frac{\partial}{\partial x_1} x_1 \frac{\partial}{\partial x_2} x_2 + \frac{\partial}{\partial x_2} x_2  \frac{\partial}{\partial x_1} x_1 + \frac{\partial}{\partial x_2} x_2 \frac{\partial}{\partial x_2} x_2=\]
\[=\frac{\partial^2}{\partial x_1 \partial x_1} x_1^2 + \frac{\partial^2}{\partial x_1 \partial x_2} x_1 x_2 + \frac{\partial^2}{\partial x_2 \partial x_1} x_2 x_1\frac{\partial^2}{\partial x_1 \partial x_2} x_1 x_2 + \frac{\partial^2}{\partial x_2 \partial x_2} x_2 x_2\]
\end{fakt}

\subsection{Ogólny wzór Taylora}

\begin{fakt}
Wzór Taylora. $f: \mathbb{R}^n \rightarrow \mathbb{R}, x=(x_1,x_2,\dots,x_n), a=(a_1,a_2,\dots,a_n)$\\
Zdefiniujmy $D_x = \left\{\frac{\partial}{\partial x_1}x_1 + \dots + \frac{\partial}{\partial x_n} x_n\right\}$
\[f(a+x) = f(a) + D_x(f(a)) + \frac{1}{2!} D_x^2 (f(a)) + \dots + \frac{1}{n!} D_x^n (f(a)) + R_n \]
\[R_n = \frac{1}{(n+1)!} (D_x)^{n+1} (f(a+\theta x))\]
\end{fakt}

\begin{pk}
Weźmy sobie taką $f: \mathbb{R}^2 \rightarrow \mathbb{R}$, $f(x_1,x_2)=x_1^2+x_2^2$\\
Wyznaczmy $D_x(f(a_1,a_2))$:
\[D_x(f(a_1,a_2)) = \left(\frac{\partial}{\partial x_1} x_1 + \frac{\partial}{\partial x_2} x_2\right) f(a_1,a_2) = \]
\[=\frac{\partial}{\partial x_1} f(a_1,a_2) x_1 + \frac{\partial}{\partial x_2} f(a_1,a_2) x_2 = 2a_1x_1 + 2a_2x_2\]
\[(D_x)^2(f(a_1,a_2)) = \frac{\partial^2}{\partial x_1 \partial x_1} f(a_1,a_2) x_1 x_1 + \frac{\partial^2}{\partial x_1 \partial x_2} f(a_1,a_2) x_1 x_2 + \frac{\partial^2}{\partial x_2 \partial x_1} f(a_1,a_2) x_2 x_1 + \frac{\partial^2}{\partial x_2 \partial x_2} f(a_1,a_2) x_2 x_2 =\]
\[2x_1x_1 + 0 x_1 x_2 + 0 x_2 x_1 + 1 x_2 x_2 = 2x_1^2 + 2x_2^2\]
\[(a_1+x_1)^2+(a_2+x_2)^2=a_1^2+a_2^2+2a_1x_1+2a_2x_2+2x_1^2+2x_2^2\]
\end{pk}

\section{Wykład V}

$f: \mathbb{R} \rightarrow \mathbb{R}$ ekstrema lokalne
Weźmy $f: \mathbb{R}^n \rightarrow \mathbb{R}, n\geq 2$ - cel. ekstrema lokalne

\subsection{Forma Kwadratowa}

\[\sum_{k=1}^{n} \sum_{i=1}^{n} a_{ik} y_{i} y_{k}\] zmiennych $y_1, y_2, \dots, y_n$ 
jest określona dodatnio (ujemnie) jeżeli przybiera wartości dodatnie (ujemne) 
dla wszystkich $y_1, y_2, \dots, y_n \neq 0$.\\

Przykład:

$n=3$ (dziel 2, bo liczymy dwa razy)\\
\[6y_1^2+5y_2^2+14y_3^2+4y_1y_2+8y_1y_3-2y_2y_3\]

\begin{itemize}
    \item $a_{11}=6, a_{22}=5, a_{33}=14$
    \item $a_{12}=a_{21}=\frac{4}{2}=2$
    \item $a_{13}=a_{31}=\frac{-8}{2}=-4$
    \item $a_{23}=a_{32}=-1$
\end{itemize}

Sprowadźmy do postaci kanonicznej i przyrównajmy do 0.

\[0=(y_1-y_3)^2+2(y_1+y_2+y^3)^2+3(y_2-y_3)^2=0\]
\[y_1=y_2=y_3\dots=y_n=0\]

\begin{tw}
    Twierdzenie Sylvestera.
    \begin{enumerate}
        \item Warunek konieczny i dostateczny na to, aby forma (*) była określona dodatnio (ujemnie) wyraża się ciągiem nierówności:
        \begin{itemize}
        \item $n=2$: $\det(a_{11})> (<) 0, \det\left(\begin{bmatrix}
            a_{11} & a_{12} \\
            a_{21} & a_{22} 
            \end{bmatrix} \right) > (>) 0$\\
        \item $n=3$: $\det(a_{11})> (<) 0, \det\left( \begin{bmatrix}
            a_{11} & a_{12} & a_{13} \\
            a_{21} & a_{22} & a_{23} \\
            a_{31} & a_{32} & a_{33} 
            \end{bmatrix} \right)  > (<) 0$\\
        \item $n$ ... itd (pojedyńczy wyraz < i det naprzemiennie dla ujemnych)
        \end{itemize}
    \end{enumerate}
\end{tw}

\begin{de}
    Forma kwadratowa (*).
    \begin{enumerate}
        \item jest nieokreślona, jeśli przybiera wartości różnych znaków i zeruje się tylko dla \\ $y_1=y_2=\dots=y_n=0$
        \item półokreślona, jeśli przybiera wartości różnych znaków i zeruje się nie tylko dla \\ $y_1=y_2=\dots=y_n=0$
    \end{enumerate}
\end{de}

\begin{tw}
    $f: \mathbb{R}^n \rightarrow \mathbb{R}$. Niech $f(x_1, x_2, \dots, x_n)$ ma w otoczeniu (kuli otwartej) punktu $x_0=(x_1^0,x_2^0,\dots,x_n^0)$ 
    pierwsze i drugie pochodne cząstkowe ciągłe. Niech $f_{x_1}(x_0)=\frac{\partial f}{\partial x_1}(x_0)=0, f_{x_2}(x_0)=0, \dots$ (gradient jest zero).
    Oznaczamy $a_{ik} = f_{x_i x_k} (x_0) = f_{x_1 x_k} (x_1^0, x_2^0,\dots x_n^0)$
    \begin{enumerate}
        \item Jeżeli forma kwadratowa \[\sum_{k=1}^{n} \sum_{i=1}^{n} a_{ik} y_{i} y_{k}\]
        jest dodatnio (ujemnie) określona, to $f$ ma w punkcie $x_0=(x_1^0,x_2^0,\dots,x_n^0)$ minimum (maksimum) lokalne.
        \item Jeżeli ww. forma kwadratowa jest nieokreślona to $f$ nie ma w punkcie $x_0=(x_1^0,x_2^0,\dots,x_n^0)$ ekstremum.
        \item Jeżeli ww. forma kwadratowa jest półokreślona to kryterium nic nie powie
    \end{enumerate}
\end{tw}
Ilustracja twierdzenia dla
$f(x_1,x_2,x_3)=x_1^2+x_2^2+x_3^2$\\
$f_{x_1} (x_1,x_2,x_3) = 2x_1 =0$\\
$f_{x_2} (x_1,x_2,x_3) = 2x_2 =0$\\
$f_{x_3} (x_1,x_2,x_3) = 2x_3 =0$\\
$(x_1^0, x_2^0, x_3^0) = (0,0,0)$ znaleźliśmy punkt podejrzany.\\
$a_{11}=2, a_{22}=2, a_{33}=2, a_{12}=a_{21}=a_{23}=a_{32}=a_{31}=a_{13}=0$\\\\
$ \begin{bmatrix}
a_{11} & a_{12} & a_{13} \\
a_{21} & a_{22} & a_{23} \\
a_{31} & a_{32} & a_{33} 
\end{bmatrix} = $
$ \begin{bmatrix}
2 & 0 & 0 \\
0 & 2 & 0 \\
0 & 0 & 2
\end{bmatrix} $\\
\begin{enumerate}
    \item $\det\left(a_{11}\right) = 2 > 0$
    \item $\det\begin{bmatrix}
    a_{11} & a_{12} \\
    a_{21} & a_{22} 
    \end{bmatrix} = 4 > 0$
    \item $\det\begin{bmatrix}
        a_{11} & a_{12} & a_{13} \\
        a_{21} & a_{22} & a_{23} \\
        a_{31} & a_{32} & a_{33} 
        \end{bmatrix}=8>0$
\end{enumerate}

\begin{de}
    Otoczeniem punktu $P\in\mathbb{R}^n$ nazywamy dowolny zbiór otwarty $U$, taka, że $P\in U$
\end{de}

\subsection{Jacobian funkcji}

\begin{de}
    Jacobian funkcji $F: \mathbb{R}^n \rightarrow \mathbb{R}^n$. $F=(f_1,f_2,\dots,f_n): \mathbb{R}^n\rightarrow\mathbb{R}$
    \[M_{F'(a)} = \begin{bmatrix}
        \frac{\partial f_1}{\partial x_1} (a) & \frac{\partial f_1}{\partial x_2} (a) & \dots & \frac{\partial f_1}{\partial x_n} (a) \\
        \frac{\partial f_2}{\partial x_1} (a) & \frac{\partial f_2}{\partial x_2} (a) & \dots & \frac{\partial f_2}{\partial x_n} (a) \\
        \dots & \dots & \dots & \dots \\
        \frac{\partial f_n}{\partial x_1} (a) & \frac{\partial f_n}{\partial x_2} (a) & \dots & \frac{\partial f_n}{\partial x_n} (a) \\
        \end{bmatrix}\]
    Jacobian: $J_f (a) = \det (M_{F'(a)})$
\end{de}

\begin{tw}
    (O funkcji odwrotnej)
    Zakładamy, że $F: \mathbb{R}^n\rightarrow \mathbb{R}^n, a\in \mathbb{R}^n$.
    Zakładamy, że $F$ ma ciągłe pochodne cząstkowe w pewnym otoczeniu punktu $a$.
    Zakładamy, że $J_{F}(a)\neq 0$.
    Wówczas istnieje otoczenie $U$ punktu $a$ oraz $V$ punktu $F(a)$ oraz funkcja rózniczkowalna $\varphi: V\rightarrow U$, taka że:
    \[\forall_{x\in U} (\varphi \circ F) (x) = x\]
    (Przy powyższych założeniach lokalnie funkcje daje się odwrócić)
\end{tw}

\begin{pk}
    (Jeden wymiar) \\
    Weźmy $F:\mathbb{R} \rightarrow \mathbb{R}$, $F(x)=x^2, a>0$\\
    $F'(a) = 2a, J_F(a)=2a\neq 0$\\
    $\varphi \circ F(x) = x$\\
    $\varphi = \sqrt{x}$\\
    $\varphi\left(F(x)\right)=\sqrt{F(x)}$\\
    Powyższe twierdzenie stanowi uogólnienie dla $n$ wymiarów.
\end{pk}

Niech $f: \mathbb{R}^2\rightarrow\mathbb{R}$, $f(x_0,y_0)=0$\\
$\{(x,y)\in\mathbb{R}^2 : f(x,y)=0\}$. $f(x,y)=x^2+y^2-4$, $f:\mathbb{R}^2\rightarrow \mathbb{R}$\\
$x^2+y^2=2$ równanie okręgu.\\
$y=\pm \sqrt{4-x^2}$\\
$\varphi(t)=\sqrt{4-x^2}$\\
Szukamy funkcji (istnieje $\varepsilon > 0$)
$\varphi(t): \mathbb{R}\rightarrow\mathbb{R}$ takiej, że:
\begin{enumerate}
    \item $\varphi(x_0) = y_0$
    \item $\left(\forall t\in(x_0 - \varepsilon, x_0 + \varepsilon)\right) f(t,\varphi(t))=0$
\end{enumerate}

\begin{tw}
    (O funkcji uwikłanej) Jeśli pochodne cząstkowe funkcji $f$ są ciągłe oraz $\frac{\partial f}{\partial y} (x_0,y_0) \neq 0$ to istnieje funkcja $\varphi$.
\end{tw}
Liczmy:\\
$f(t,\varphi(t)) = 0$\\
$\varphi'(t)=??$\\
$\frac{d}{dt} \left(f(t,\varphi(t)) \right) = \frac{d}{dt} (0) = 0$\\
$\frac{\partial}{\partial x} \frac{d}{dt} (t) + \frac{\partial f}{\partial y} \frac{d}{dt} (\varphi(t)) = 0$\\
$\frac{\partial f}{\partial x} 1 + \frac{\partial f}{\partial y} \varphi'(t)= 0$\\
$\varphi'(t) = \frac{\frac{\partial f}{\partial x}}{\frac{\partial f}{\partial x}}$

Przypomnienie\\
$\frac{d}{dt} F(x(t),y(t)) = \frac{\partial F}{\partial x} x'(t) + \frac{\partial F}{\partial y} y'(t)$

\begin{tw}
    $f: \mathbb{R}^2 \rightarrow \mathbb{R}$. Niech $f$ ma ciągłe pochodne cząstkowe w punkcie $(x_0,y_0)$. Równanie płaszczyzny
    stycznej do wykresu funkcji $f$ w punkcie $(x_0,y_0,f(x_0,y_0))$:
    \[z-f(x_0,y_0)=\frac{\partial f}{\partial x} (x_0,y_0) (x-x_0) + \frac{\partial f}{\partial y} (x_0,y_0) (y-y_0)\]
    ($z-z_0=A(x-x_0)+B(y-y_0)$) - równanie płaszczyzny przechodzącej przez $(x_0,y_0,z_0)$
\end{tw}

\section{Wykład VI}

\subsection{Mnożniki Lagrange'a}

$f, g : \mathbb{R}^2 \rightarrow \mathbb{R}$
Max/Min $f(x_1,x_2)$ pod warunkiem $g(x_1,x_2)$\\
Np. Znaleźć Max/Min $x^2+y^2$ pod warunkiem $xy=2$\\
$f(x,y)=x^2+y^2$\\
$g(x,y)=xy-2$\\
$\Phi(x,y,\lambda)=f(x,y)-\lambda g(x,y) = x^2+y^2 - \lambda(xy) + 2\lambda$\\
$\frac{\partial \Phi}{\partial x} = 2x - \lambda y = 0$\\
$\frac{\partial \Phi}{\partial x} = 2y - \lambda x = 0$\\
$ \begin{bmatrix}
    2 & -\lambda \\
    -\lambda & 2 
    \end{bmatrix}
\begin{bmatrix}
    x\\
    y
\end{bmatrix}
=
$
$
\begin{bmatrix}
0\\
0
\end{bmatrix}
$
\begin{enumerate}
    \item $\det \begin{bmatrix}
        2 & -\lambda \\
        -\lambda & 2 
        \end{bmatrix} \neq 0$, nie spełnia $xy=2$
    \item $\det \begin{bmatrix}
        2 & -\lambda \\
        -\lambda & 2 
        \end{bmatrix} = 0$\\
        $4-\lambda^2 =0$\\
        $\lambda = \pm 2$
    \item $\lambda = -2$\\
    $2x+2y=0\iff x=-y$\\
    Wtedy $2=(-y)y = -y^2\leq 0$, nie ma ekstremów
    \item $\lambda = 2$\\
    $2x-2y = 0 \iff x=y$\\
    $x^2=2$ dwa punkty podejrzane $(\pm\sqrt{2},\pm\sqrt{2})$\\
    Sprawdzamy warunek dostateczny dla funkcji $f(x,y)=x^2+y^2$\\
    Policzmy Hessian:\\
    $f_{xx}(x,y)=2, f_{xx} (\sqrt{2},\sqrt{2}) > 0$\\
    $W(\sqrt{2},\sqrt{2})=4>0$\\
    W punktach $(\sqrt{2},\sqrt{2}), (-\sqrt{2},-\sqrt{2})$ mamy minima lokalne.
\end{enumerate}

Teoria ($n=2$)
\begin{de}
    Punkt $a\in\mathbb{R}^n$ jest punktem regularnym zbioru $\{x\in\mathbb{R}^n:  g(x)=0\}$ jest $g(a)=0$ oraz gradient funkcji $g$ w punkcie $a$ jest różny od $0$.
\end{de}

\begin{tw}
    Jeśli $f:\mathbb{R}^n\rightarrow\mathbb{R}$ ma ekstremum lokalne w punkcie $a\in\mathbb{R}^n$ pod warunkiem, że $g(a)=0$, to istnieje
    $\lambda$, takie że $\nabla(f - \lambda g)_{x=a} = 0$ o ile $a$ jest punktem regularnym.
\end{tw}

Przykład:\\
$f(x,y)=xy$\\
$g(x,y)=\frac{x^2}{8}+\frac{y^2}{2}-1=0$\\
$\Phi(x,y,\lambda) = xy - \lambda (\frac{x^2}{8} + \frac{y^2}{2}-1)$\\
$\frac{\partial \Phi}{\partial x} = y - \frac{\lambda x}{4} =0$\\
$\frac{\partial \Phi}{\partial y} = x - \lambda y = 0 $\\
$\begin{bmatrix}
-\frac{\lambda}{4} & 1\\
1 & -\lambda
\end{bmatrix}$
$\begin{bmatrix}
    x\\
    y
\end{bmatrix}
=
$
$
\begin{bmatrix}
0\\
0
\end{bmatrix}$\\
\begin{enumerate}
    \item $\det A \neq 0$\\
    $x=y=0$, nie ma rozwiązań
    \item $\det A = 0 \left(\lambda = 2 \lor \lambda = -2\right)$\\
    $\lambda = 2 \implies \left(y=\frac{1}{2} x \land \frac{x^2}{8} + \frac{y^2}{2} - 1 =0\right)$\\
    $\left(2,1\right), \left(-2,-1\right)$\\
    \item $\lambda = -2 \implies \left(y=-\frac{1}{2}x \land \frac{x^2}{8}+\frac{y^2}{2} - 1 = 0\right)$\\
    $\left(2,-1\right), \left(-2,1\right)$\\
    Badamy max/min:
    \begin{enumerate}
        \item $f(2,1)=2$ max lok
        \item $f(-2,-1)=2$ max lok
        \item $f(2,-1)=2$ min lok
        \item $f(-2,1)=-2$ min lok
    \end{enumerate}
\end{enumerate}

\subsection{Całka Lebesgue'a}

Przedziały $a_1 \leq b_1, a_2 \leq b_2, \dots, a_n \leq b_n$.
$\Pi = [a_1, b_1] \times [a_2, b_2] \times \dots \times [a_n, b_n]$
Przedział domknięty na $\mathbb{R}^n$. 

\begin{de}
    Wnętrze przedziału oznaczamy:
    \[\text{Int}([a_1,b_1]\times\dots\times[a_n,b_n]) = (a_1,b_1)\times\dots\times(a_n,b_n)\]
\end{de}

\begin{de}
    Niech $\Pi, \Delta$ będą przedziałami. Mówimy, że $\Pi, \Delta$ nie zachodzą na siebie
    i piszemy, że $\Pi \bot \Delta$ jeśli:
        \[\text{Int} \Pi \cap \text{Int} \Delta = \emptyset\]
\end{de}

\begin{de}
    Rodzina przedziałów $P$ jest rozbiciem przedziałów $\Pi$ jeśli:
    \begin{enumerate}
        \item $\bigcup P = \Pi$
        \item $\forall P,Q \in P \left(P\neq Q =\implies P \bot Q \right)$
    \end{enumerate}
\end{de}

\begin{de}
    $\text{Vol}\left(\prod_{i=1}^{n} [a_i,b_i]\right) = \prod_{i=1}^{n} (b_1-a_1)$
\end{de}

\begin{pk}
    \begin{enumerate}
        \item $\text{Vol}([a,b]) = b-a$
        \item $\text{Vol}([a,b]\times[c,d]) = (b-a)(c-d)$
        \item $\text{Vol}([a,b]\times[c,d]\times[e,f]) = (b-a)(c-d)(e-f)$
    \end{enumerate}
\end{pk}

\begin{tw}
    Jeśli $P$ jest rozbiciem $\Pi$ to:
    \[\text{Vol}(\Pi) = \sum_{p\in P} \text{Vol}(p)\]
\end{tw}

Własności:
\begin{enumerate}
    \item $\Pi \subset \Delta \implies \text{Vol}(\Pi) \leq \text{Vol}(\Delta)$
    \item $\Pi \subset \Delta_1 \cup \Delta_2 \cup \dots \cup \Delta_n$
    to $\text{Vol}(\Pi) \leq \sum_{i=1}^{n} \text{Vol}{\Delta_i}$
\end{enumerate}

\begin{de}
    Miara zewnętrzna Lebesgue'a. Ustalamy $n$, $A\subseteq \mathbb{R}^n$, $A$ - ograniczony.
    \[\lambda^{*}(A) = \inf \left\{\sum \text{Vol}(\pi_i) \left(\forall_i \right)(\pi_i \text{ jest produktem} \land A\subset \bigcup_{n\geq 0} \pi_n)\right\}\]
    Oznaczamy $|A|=\lambda^{*}(A)$
\end{de}

\begin{enumerate}
    \item $0 \leq |A| \leq \infty$
    \item $A\subseteq B \implies |A| \leq |B|$
    \item $\bigcup_{n=1}^{\infty} A_n \leq \sum_{n=1}^{\infty} |A_n|$
\end{enumerate}

\begin{tw}
    $\Pi$ jest przedziałem:
    \[\lambda^{*}(\Pi) = \text{Vol}(\Pi)\]
\end{tw}

\begin{de}
    Zbiór $A\subseteq \mathbb{R}^n$ jest zbiorem miary 0, jeśli $\lambda^{*}(A)=0$.
    \begin{enumerate}
        \item $n=2$, $A = (a,b) \subset \mathbb{R}^2$\\
        $\lambda^{*}(B)=(b-a)2\varepsilon$\\
        $\lambda^{*}(A)=0$
        \item $\qed\in \mathbb{R}^3$ jest miary zero\\
        $\lambda^{*}(\mathbb{Q})=\lambda^{*}(\bigcup_{n\in\mathbb{N}} \{P_n\}) \leq \sum_{n\in\mathbb{N}} \{P_n\} = \sum_{n\in\mathbb{N}} 0 = 0$
    \end{enumerate}
\end{de}

\begin{de}
    Zbiór $A\subset \mathbb{R}^n$ jest mierzalny wedug Lebesgue'a jeśli:
    \[\forall Z\subseteq \mathbb{R}^n \left(\lambda^{*}(Z\cap A) + \lambda^{*}(Z\cap A^C) = \lambda^{*}(Z)\right)\]
\end{de}

\begin{de}
    $M_n$ - rodzina zbiorów mierzalnych. Miarą Lebesgue'a na $M_n$ nazywamy funkcję:
    \[\lambda(A) = \lambda^{*}(A)\]
\end{de}
Funkcje proste $A\subseteq \mathbb{R}^n$, Niech $B\subseteq A$.
\[
\textbf{1} =
\begin{cases}
    1 \text{ dla } x\in B\\
    0 \text{ dla } x\in A-B
\end{cases}
\]

\begin{de}
    Niech $A\in M_n$. Funkcję prostą o nośniku $A$ nazywamy funkcję postaci:
    \[ f(x) = \sum_{i=1}^{n} a_i \cdot \textbf{1}_{A,B} (x), a_i\in\mathbb{R}\]
    Gdzie $(B_i)_{i=1}^{n}$ jest rodziną zbiorów parami rozłącznych, mierzalnych, zawartych w $A$.
\end{de}

\section{Wykład VII}

\begin{de}
    $f$ prosta $f=\sum_{i=1}^{n} a_i \textbf{1}_{A,B_i}$.
    \[\int_{A} f = \sum{i=1}^{n} a_i \lambda(B_i)\]
\end{de}

\subsection{Przykład całki Lebesgue'a}

Przykład:
$A\subset \mathbb{R}^m, m=1$\\
$f(x)=1 \cdot$
$\textbf{1}_{A,B_1} (x)$
$+ 2 \cdot \textbf{1}_{A,B_2} (x)$
$+ 4 \cdot \textbf{1}_{A,B_3} (x)$\\
\[\int_{A} f = 1\lambda(B_1) + 2\lambda(B_2) + 4\lambda(B_3)\]
$D(x)=\begin{cases}
    1, x\in \mathbb{Q} \cap [0,1]\\
    0, x\in [0,1] - \mathbb{Q}
\end{cases}$
$A=[0,1], B_1 = \mathbb{Q} \cap [0,1], B_2=[0,1] - \mathbb{Q}$\\
$D(x) = 1 \cdot \textbf{1}_{A,B_1} (x) + 0 \cdot \textbf{1}_{A,B}$
\[\int_{[0,1]} 1\lambda(\mathbb{Q}\cap[0,1]) + 0\lambda(B_2)=0\]
Całkowalna w sensie Lebesgue'a, nie istnieje cała Riemanna.\\
Corollary:
$f: \mathbb{R}^m \rightarrow \mathbb{R}, g: \mathbb{R}^m \rightarrow \mathbb{R}$.
$\{x\in\mathbb{R}^n f(x)\neq g(x)\}$ jest miary zero. Wówczas $g$ jest całkowalna:
$\int_{\mathbb{R}^m} g = \int_{\mathbb{R}^m} f$\\
\[\int g = \int f\]

\subsection{Całki wielokrotne}

Całkowanie po wielu zmiennych można wykonać po kolei.

\begin{align}
    \int_{0}^{1} \int_{0}^{1} xy dx dy =\\
    =\int_{0}^{1} \left(\int_{0}^{1} xy dx \right) dy=\\
    =\int_{0}^{1} \left(\left[\frac{x^2y}{2}\right]_{x=0}^{x=1}\right) dy=\\
    =\int_{0}^{1} \left(\frac{y}{2}\right) dy = \left[\frac{1}{4} y^2\right]_{0}^{1}=\\
    =\frac{1}{4} 
\end{align}

\subsection{Twierdzenie Fubiniego}

\begin{tw}
    Zakładamy, że funkcja $f:\mathbb{R}^n \times \mathbb{R}^m \rightarrow \mathbb{R}$
    jest nieujemna lub całkowalna. Niech $x\in\mathbb{R}^n$, $y\in\mathbb{R}^m$. Dla $x\in\mathbb{R}^n$ definiujemy:
    \[f_x(y) = f(x,y)\]
    Wtedy:
    \begin{itemize}
        \item dla prawie wszystkich $x$ funkcje $f_x$ są całkowalne
        \item $\int_{\mathbb{R}^n\times\mathbb{R}^m} = \int_{\mathbb{R}^n} \left(\int_{\mathbb{R}^m} f\right)$
    \end{itemize}
\end{tw}

Przykład:
$x=(x_1,x_2,\dots,x_n), y=(y_1,y_2,\dots, y_m)$\\
$\int \dots \int_{(n+m) R^n\times R^m} f(x_1,x_2,\dots,x_n,y_1,y_2,\dots,y_n) dx_1, dx_2 \dots, dy_1, dy_2, \dots, dy_n =$\\
$=\int \int_{\mathbb{R}^n} \left(\int \int_{\mathbb{R}^m} f(x_1,\dots, y_n) dy_1, dy_2, \dots, dy_n\right) dx_1, dx_2, \dots, dx_n$

\subsection{Suma wielokrotna}
\[\sum_{k=1}^{\infty}\sum_{k=1}^{\infty} a_{ik} = A\]
\[A=\sum_{k=1}^{\infty}\left(\sum_{i=1}^{\infty} a_{ik}\right)= \sum_{i=1}^{\infty}\left(\sum_{k=1}^{\infty} a_{ik}\right)\]

\begin{tw}
    Inna wersja twierdzenia Fubiniego.\\
    Niech $A\subset \mathbb{R}^n, B\subseteq \mathbb{R}^m$
    $A, B$ - mierzalne,
    $f: A\times B \rightarrow \mathbb{R}$ - nieujemna lub całkowalna.\\
    Dla $x\in A$ definiujemy:
    \[f_x(y) = f(x,y)\]
    Wtedy:
    \begin{itemize}
        \item dla prawie wszystkich $x$ funkcje $f_x$ są całkowalne
        \item $\int \int_{A\times B} f(x,y) dx dy = \int_A \left(\int_B f_x(y) \right) dy dx$
    \end{itemize}
\end{tw}

Funkcja dwóch zmiennych jest całkowalna jeżeli istnieje oszacowanie:
\[|f(x,y)| \leq g(x,y)\]
Prowadzący nt. powyższego "Lepiej żeby to po prostu nie było napisane"

\begin{pk}
    \[\int_{[0,1]} \int_{[0,1]} \int_{[0,1]} (2x+2y+2z) dx dy dz\]
    Rozbijmy to na całki iterowane:
    \begin{align}
        \int_0^1 \left( \int_0^1 \left( \int_0^1 (2x+2y+2z) dx \right) dy \right) dz =\\
        =\int_0^1 \left( \int_0^1 \left[x^2 + (2y+2z)x\right]_0^1 \right) dy dz= \\
        =\int_0^1 \left( \int_0^1 1+2y+2z dy \right) dz=\\
        =\int_0^1 \left[(1+2z)y + y^2\right]_0^1 dz=\\
        =\int_0^1 (2+2z) dz = \left[2z+z^2\right]_0^1 = 3
    \end{align}
    Integrate[2x+2y+2z,(x,0,1),(y,0,1),(z,0,1)]
\end{pk}
$\text{vol}([a,b]\times[c,d]\times[e,f]) = \lambda_3(\dots) = \int\int\int_P 1 dx dy dz$

$\int \dots \int_{[0,1]^n} (x_1^2 + x_2^2 + \dots + x_n^2) dx_1dx_2 \dots dx_n = n \frac{1}{3}$\\
$=\sum_{i=1}^{\infty} \int \dots \int_{[0,1]^2} (x_i^2) dx_1 dx_2 \dots dx_n = n \frac{1}{3}$\\
$\int_0^1 x_1^2 dx_1 = \left[\frac{x_1^2}{3}\right]_0^1 = \frac{1}{3}$\\
$(\frac{b_n^3}{3} - \frac{a_n^3}{3})(b_{n-1} - a_{n-1})(b_{n-2} - a_{n-1})\dots(b_1-a_1)$\\

\subsection{Pole koła}

$K=\{(x,y): x^2 + y^2 \leq r^2\}$\\
$y=\pm\sqrt{r^2-x^2}$\\
$-r \leq x \leq r$\\
$-\sqrt{r^2-x^2} \leq y \leq \sqrt{r^2-x^2}$
\[\int \int 1 dx dy = \lambda(k) = \pi r^2 dx_n = \int_{-r}^{r} \left(\int_{-\sqrt{r^2-x^2}}^{\sqrt{r^2-x^2}} 1 dy \right) dx =\]
\[=\int_{-r}^{r} 2\sqrt{r^2-x^2} dx = 2 \int_{-r}^{r} \sqrt{r^2-x^2} dx = 2 \frac{pi}{2} r^2 = \pi r^2\]

Integrate($sqrt(r^2-x^2)$,$(x,-r,r)$)\\

$\int_{-r}^{r} \sqrt{r^2-x^2} = 2\int_{0}^{r} \sqrt{r^2-x^2} dx$\\
Podstawmy $x=rt, dx = r dt$, zatem:
$=2r^2 \int_0^1 \sqrt{1-t^2} dt = r^2 \frac{\pi}{2}$

\begin{pk}
Policzmy następującą całkę:\\
$\int \int_{\Delta} (x+y) dx dy, x=0,y=0, x+y=1$\\
$0 \leq x \leq 1, 0 \leq y \leq 1-x$\\
zatem:
\[A=\int_0^1 \left( \int_0^1 (x+y) dy\right) dx = \frac{1}{3}\]
Obliczenia:
\[\int_0^1 \left(x(1-x) + \frac{(1-x)^2}{2}\right) dx=\]
\[=\int_0^1 (x-x^2+\frac{1}{2} + \frac{x^2}{2} - x) dx=\]
\[=\int_0^1 \frac{1}{2} - \frac{x^2}{2} dx = \left[\frac{1}{2} x - \frac{x^3}{3}\right]_0^1 = \frac{1}{2} - \frac{1}{6} = \frac{1}{3}\]
\end{pk}

$A=\int_0^1 \int_0^{1-y} (x+y) dx dy$

\section{Wykład VIII}

\subsection{Twierdzenie o zamianie zmiennych}

\begin{tw}
    O zamianie zmiennych $:\varphi: \mathbb{R}^n \rightarrow \mathbb{R}^n$ jest $1-1$ i $\mathbb{C}^{1}$ (różniczkowalna i ciągłe pochodne cząstkowe)
    $\varphi(x_1,x_2,\dots,x_n)=\left(\varphi_1(x_1,x_2,\dots,x_n), \varphi_2(x_1,x_2,\dots,x_n), \dots \varphi_n(x_1,x_2,\dots,x_n)\right)$.
    Rozważamy macierz Jacobiego:
    \[\mathbf {J} ={\begin{bmatrix}{\dfrac {\partial \varphi_{1}}{\partial x_{1}}}&\cdots &{\dfrac {\partial \varphi_{1}}{\partial x_{n}}}\\\vdots &\ddots &\vdots \\{\dfrac {\partial \varphi_{n}}{\partial x_{1}}}&\cdots &{\dfrac {\partial \varphi_{m}}{\partial x_{n}}}\end{bmatrix}}\]
    $\det\left(\mathbf {J}_{\varphi(x)}\right) \neq 0$
    Niech $f:\varphi(D)\rightarrow\mathbb{R}$ ciągła.
    \[\int_{\varphi(D)}^{n}\int f(y) dy_1 dy_2 \dots dy_n = \int_{D}^{n} \int f(x) \cdot \left|\det\mathbf{J}_{\varphi(x)}\right| dx_1 dx_2 \dots dx_n\]
\end{tw}

\subsection{Współrzędne biegunowe}

\begin{pk}
    Współrzędne biegunowe $(x,y)\in\mathbb{R}^2$, $x=r\cos(\alpha)$, $y=r\sin(\alpha)$\\
    $r\in\left[0,\infty\right), \alpha \in [0,2\pi]$\\
    $\varphi(r,\alpha) = \left(r\cos(\alpha),r\sin(\alpha)\right)$\\
    $\varphi : \mathbb{R}^2 \rightarrow \mathbb{R}^2$\\
    \[\mathbf{J}_{\varphi(r,\alpha)} = \begin{bmatrix}\frac{\partial \varphi_1}{\partial r}&\frac{\partial \varphi_1}{\partial \alpha}\\\frac{\partial \varphi_2}{\partial r}&\frac{\partial \varphi_2}{\partial \alpha}\end{bmatrix}\]
    \[\mathbf{J}_{\varphi(r,\alpha)} = \begin{bmatrix}\cos(\alpha)&r(-\sin(\alpha))\\\sin(\alpha)&r\cos(\alpha)\end{bmatrix} |\det(\mathbf{J}_{\alpha(r,\alpha)})| = |r\cos^2\alpha + r\sin^2 \alpha| = |r| = r\]
    \[\int_{\varphi{D}} \int f(x,y) dx dy = \int_D \int f(r,\alpha) r dr d\alpha\]
\end{pk}

\begin{pk}
    Pole koła o promieniu $R$
    \[\int\int_{x^2+y^2\leq R^2} 1 dx dy =\]
    Podstawmy $x=r\cos(\alpha)$, $y=r\sin(\alpha)$, $r$: mamy:\\
    \[\int_{0}^{2\pi}\left(\int_{0}^{2\pi} r d\alpha\right) dr = \int_{0}^{R} 2\pi r dr = \left|\frac{2\pi r^2}{2}\right|_{0}^{R} = \pi R^2\]
    \[\int\int_{r^2\leq R^2} r dr d\alpha = \int_0^{2\pi} \int_0^R r dr d\alpha = \int_0^{2\pi} \left[\frac{r^2}{2}\right]_0^R d\alpha = \int_0^{2\pi} \frac{R^2}{2} d\alpha = \pi R^2\]
\end{pk}

\begin{pk}
    \[\int_{\varphi(D)}\int f(x,y) dxdy = \int_{D}\int f(r,\alpha) r dr d\alpha\]
    \[\int\int_{x^2+y^2\leq R^2} xy dx dy = \int_{r\in[0,R]}\int_{\alpha[0,2\pi]} r\cos(\alpha) r\sin(\alpha) r dr d\alpha\]
\end{pk}

\begin{pk}
    $D, B$ - jakiś przedział,  $\int_D f(y) dy$. Podstawmy $y=\varphi(x), y\in D, x\in B$\\
    \[\int_D f(y) dy = \int_B f(\varphi(x)) \varphi'(x) dx\]
    $\int_{a}^{b} (x^+1)^2 dx$ (zob. $x^2+1=y$, $B=[(a^2+1),(b^2+1)]$, $D=[a,b]$)
\end{pk}

\begin{pk}
    Objętość elipsoidy obrotowej.
    \[E: \frac{x^2}{a^2} + \frac{y^2}{b^2} + \frac{z^2}{c^2} \leq 1\]
    \[\text{Vol}(E) = \int\int\int_{E} dx dy dz\]
    \[\varphi(x_1,y_1,z_1) = (x_1 a, y_1 b, z_1 c)\]
    \[\mathbf{J}_{\varphi(x_1,y_1,z_1)} = \begin{bmatrix}\frac{d}{dx_1}(x_1,a) & \frac{d}{dy_1}(x_1,a) & \frac{d}{dz_1}(x_1,a)\\ \frac{d}{dx_1}(y_1,b) & \frac{d}{dy_1}(y_1,b) & \frac{d}{dz_1}(y_1,b)\\ \frac{d}{dx_1}(z_1,c) & \frac{d}{dy_1}(z_1,c) & \frac{d}{dz_1}(z_1,c)\end{bmatrix}\]
    \[\mathbf{J}_{\varphi(x_1,y_1,z_1)} = \begin{bmatrix}a & 0 & 0\\0 & b & 0\\0 & 0 & c\end{bmatrix}\]
    \[\int\int\int_{E} dx dy dz = \int\int\int abc dx_1 dy_1 dz_2 = abc \int\int\int dx_1dy_1dz_1= abc \text{Vol}\left(K(0,0,0),1\right)\]
    Oszukujemy $V=\frac{4}{3}\pi r^3$, zatem:
    \[\text{Vol}=\frac{4}{3}abc \pi\]
\end{pk}

\begin{pk}
    Error function. 
    \[\int_{0}^{\infty} e^{-x^2} = \frac{\sqrt{\pi}}{2}, \mathbb{R}^{+} \in [0,\infty)\]
    \begin{align}
        \left(\int_{0}^{\infty} e^{-x^2} dx\right)\left(\int_{0}^{\infty} e^{-y^2} dy\right)=\\
        =\left(\int_{\mathbb{R}^{+}} e^{-x^2} dx\right)\left(\int_{\mathbb{R}^{+} e^{-y^2}} dy\right)=\\
        =\int\int_{\mathbb{R}^{+}\times\mathbb{R}^{+}} e^{-x^2} e^{-y^2} dxdy=\\
        =\int_{\mathbb{R}^{+}} \left(\int_{\mathbb{R}^{+}} e^{-x^2} e^{-y^2} dx\right) dy=\\
        =\int_{\mathbb{R}^{+}} e^{-y^2} \left(\int_{\mathbb{R}^{+}} e^{-x^2} dx\right) dy=\\
        =\int\int_{r\in[0,\infty]} e^{-r^2} r d\alpha dr\\
        = \frac{\pi}{2} \int_{0}^{\infty} e^{-r^2} r dr=\\
        \left|\frac{\pi}{2} \frac{-e^{-r^2}}{2} \right|_{0}^{\infty} = \frac{\pi}{4}\\
    \end{align}
    Następnie:
    \begin{align}
        \int_{0}^{\infty} \int_{0}^{\frac{\pi}{2}} e^{-r^2} r d\alpha dr=\\
        =\int_{0}^{\infty} e^{-r^2} r \int_{0}^{\frac{\pi}{2}} d\alpha =\\
        =\int_{0}^{\infty} e^{-r^2} \cdot r \frac{\pi}{2}\\
        \int_{-\infty}^{\infty} e^{-x^2} dx =\\
        =\sqrt{\pi}
    \end{align}
\end{pk}

\subsection{Współrzędne walcowe}

\begin{pk}
    Współrzędne walcowe $(x,y,z)\in\mathbb{R}^3$, $x=r\cos(\alpha)$, $y=r\sin(\alpha)$, $z=z$\\
    $r\in[0,\infty)$, $\alpha\in[0,2\pi]$, $z\in\mathbb{R}$\\
    $\varphi(r,\alpha,z) = (r\cos(\alpha),r\sin(\alpha),z)$\\
    $\mathbf{J}_{\varphi(r,\alpha,z)} = \begin{bmatrix}\cos(\alpha)&r(-\sin(\alpha))&0\\\sin(\alpha)&r\cos(\alpha)&0\\0&0&1\end{bmatrix}$\\
    $|\det(\mathbf{J}_{\varphi(r,\alpha,z)})| = |1\left(r\cos^2\alpha + r\sin^2\alpha\right)| = |r| = r$\\
    $\int\int\int_{K} f(x,y,z) dx dy dz = \int\int\int_{K} f(r,\alpha,z) r dr d\alpha dz$
\end{pk}

\begin{pk}
    Objętość walca (o promieniu $R$ i wysokości $H$):
    \[\text{Vol}(W) = \int\int\int_{W} dx dy dz = \int\int\int_{W} r dr d\alpha dz =\] 
    \[\int_{0}^{2\pi} \int_{0}^{R} \int_{0}^{H} r dr d\alpha dz = \left(\int_{0}^{2\pi}\right) \left(\int_{0}^{R} rdr\right) \left(\int_{0}^{H} 1dz\right) = \pi R^2 H\]
\end{pk}

\begin{pk}
    Wielokrotnie zastosujmy twierdzenie Fubiniego:
    \[\int\int\int_{x\in A, y\in B, z\in C} f(x)g(x)h(z) dxdydz = \left(\int_A f(x)dx\right)\left(\int_B g(y) dy\right)\left(\int_C h(z) dz\right)\]
\end{pk}

\begin{pk}
    Weźmy punkt $P(x,y,z)$. $\Theta $ - kąt z osią OZ.\\
    $\cos\Theta = \frac{z}{r} \implies z=r\cos\Theta$,
    $\sin\Theta = \frac{\text{rzut}}{r} \implies \text{rzut} = r\sin\Theta$\\
    $\cos\alpha = \frac{x}{r\sin\Theta}$,
    $\sin\alpha = \frac{y}{r\sin\Theta}$\\
    $z=r\cos\Theta, x=r\sin\Theta, y=r\sin\Theta \sin\alpha$,
    $\alpha\in(0,2\pi), \Theta\in(0,\pi)$\\
    Są to współrzędne sferyczne.    
\end{pk}

\begin{tw}
    (prawdopodobnie)
    Współrzędne sferyczne $(x,y,z)\in\mathbb{R}^3$, $x=r\sin\Theta\cos\alpha$, $y=r\sin\Theta\sin\alpha$, $z=r\cos\Theta$\\
    $r\in[0,\infty)$, $\alpha\in[0,2\pi]$, $\Theta\in[0,\pi]$\\
    $\varphi(r,\alpha,\Theta) = (r\sin\Theta\cos\alpha,r\sin\Theta\sin\alpha,r\cos\Theta)$\\
    $\mathbf{J}_{\varphi(r,\alpha,\Theta)} = \begin{bmatrix}\sin\Theta\cos\alpha&r\cos\Theta\cos\alpha&-r\sin\Theta\sin\alpha\\ \sin\Theta\sin\alpha&r\cos\Theta\sin\alpha&r\sin\Theta\cos\alpha\\ \cos\Theta&-r\sin\Theta&0\end{bmatrix}$\\
    $|\det(\mathbf{J}_{\varphi(r,\alpha,\Theta)})| = |r^2\sin^2\Theta\cos\Theta\cos^2\alpha + r^2\sin^2\Theta\cos\Theta\sin^2\alpha + r^2\sin^2\Theta\sin^2\alpha\sin^2\Theta + r^2\cos^2\Theta\sin^2\Theta\cos^2\alpha + r^2\cos^2\Theta\sin^2\Theta\sin^2\alpha + r^2\cos^2\Theta\cos^2\Theta| = r^2\sin\Theta$\\
    $\int\int\int_{K} f(x,y,z) dx dy dz = \int\int\int_{K} f(r,\alpha,\Theta) r^2\sin\Theta dr d\alpha d\Theta$
\end{tw}

\section{Wykład IX}

\subsection{Współrzędne sferyczne}
$z=r\cos(\Theta), \Theta\in[0,\pi]\\$
$y=r\sin(\Theta)\sin(\alpha), \alpha\in[0,2\pi)\\$
$x=r\sin(\Theta)\cos(\alpha) r\in[0,\infty)$\\

\[\varphi(r,\alpha,\Theta) = (r\sin(\Theta)cos(\alpha), r\sin(\Theta)\sin(\alpha), r\cos(\Theta))\]
Jakobian $J_{\alpha}=(r,\alpha, \Theta) = r^2\sin(\Theta)$ Zapisane współrzędne:\\
\[\frac{d}{dr} (r\sin(\Theta)\cos(\alpha)), \frac{d}{d\alpha} (r\sin(\Theta)\cos(\alpha)), \frac{d}{d\Theta} (r\sin(\Theta)\cos(\alpha))\]
\[\frac{d}{dr} (r\sin(\Theta)\sin(\alpha)), \frac{d}{d\alpha} (r\sin(\Theta)\sin(\alpha)), \frac{d}{d\Theta} (r\sin(\Theta)\sin(\alpha))\]
\[\frac{d}{dr} (r\cos(\alpha)), \frac{d}{d\alpha} (r\cos(\Theta)), \frac{d}{d\Theta} (r\cos(\Theta))\]

\[\int\int\int_{\varphi(D)} f(x,y,z)dxdydz = \int\int\int_{D(r,\alpha,\Theta)} f(r\sin\Theta\cos\alpha, r\sin\Theta\sin\alpha, r\cos\Theta) r^2 \sin\Theta dr d\alpha d\Theta\]

Objętość kuli $K_3((0,0,0),R)$ o promieniu $R$.
\begin{align}
    \int\int\int 1dxdydz =\\
    \int\int\int_{r\in[0,R]} r^2\sin\Theta drd\alpha d\theta =\\
    \int\left(\int\left(\int r^2 \sin\Theta dr\right) d\alpha\right) d\Theta =\\
    \int_{0}^{\pi}\left(\int_{0}^{2\pi} \left(\frac{R^3}{3} \sin\Theta\right) d\alpha\right)d\Theta =\\
    \int_{0}^{\pi}\left(\frac{2\pi R^3}{3} \sin\Theta\right) d\Theta =\\
    \frac{2\pi R^3}{3}(-\cos\Theta)|_{0}^{\pi} =\\
    \frac{4}{3}\pi R^3
\end{align}

Objętość $n$-wymiarowej kuli $K_n=\{(x_1,x_2,\dots,x_n)\in\mathbb{R}^n, x_1^2+x_2^2+\dots+x_n^2 \leq r^2\}$\\
Miara Lebesgue'a $\lambda(K_1)=2r, \lambda(K_2)=\pi r^2, \lambda(K_3)=\frac{4}{3}\pi r^3$
\[\lambda(K_n) = \frac{\pi^{\frac{n}{2}}r^n}{\Gamma(\frac{n}{2}+1)}\]
Gdzie $\Gamma(a)=\int_{0}^{\infty} t^{a-1}e^{-t}dt, \Gamma(a+1)=a\Gamma(a),\Gamma(a+1)=a!$

\subsection{Funkcja Beta}

Niech $a,b>0$:
\[B(a,b)=\int_{0}^{1} x^{a-1} (1-x)^{b-1} dx\]
\[B(a,b)=\frac{\Gamma(a)\Gamma(b)}{\Gamma(ab)}\]

Policzmy:

\[\Gamma\left(\frac{1}{2}\right) = \sqrt{\pi}\]
\begin{align}
    \Gamma\left(\frac{1}{2}\right) = \int_{0}^{\infty} t^{-\frac{1}{2}} e^{-t} dt=\\
    \text{Podstawmy } t=s^2, dt=2sds \text{, mamy:}\\
    =2\int_{0}^{\infty} e^{-s^2} ds = 2 \frac{\sqrt{\pi}}{2} = \sqrt{\pi}
\end{align}
Pokażmy, że faktycznie:
\[\lambda(K_n) = \frac{\pi^{\frac{n}{2}}r^n}{\Gamma(\frac{n}{2}+1)}\]
D-d. Indukcja po $n$:
\begin{enumerate}
    \item $n=1$:
    $\lambda(K_1)=\frac{\pi^{\frac{1}{2}}r}{\Gamma\left(\frac{3}{2}\right)}=2r$ super
    \item $n_0=n$. Popatrzmy na $K_{n+1}$:
    \begin{align}
        x_1^2+x_2^2+\dots+x_{n+1}^2 &\leq r^2\\
        x_1^2+x_2^2+\dots+x_n^2 &\leq r^2-x_{n+1}^2 = \left(\sqrt{r^2-x_{n+1}^2}\right)
    \end{align}
    Wyznaczmy $\lambda(K_{n+1})$: 
    \begin{align}
        \lambda(K_{n+1}) &= \int_{-r}^{r} \left(\int \dots \int_{K x_1,x_2,\dots,x_n} dx_1 dx_2 \dots dx_n\right) dx_{n+1}\\
        &=\int_{-r}^{r} \frac{\pi^{\frac{n}{2}} \left(\sqrt{r^2-x_{n+1}^2}\right)^n}{\Gamma(\frac{n}{2}+1)}\\
        &=\frac{\pi^{\frac{n}{2}}}{\Gamma(\frac{n}{2}+1)} \int_{-r}^{r} (r^2-x_{n+1}^2)^{\frac{n}{2}} dx_{n+1}\\
        &\text {Podstawmy } x_{n+1}=rt, dx_{n+1} = rdt \text{, mamy:}\\
        &=\frac{\pi^{\frac{n}{2}}}{\Gamma(\frac{n}{2}+1)}r^{n+1} \int_{-1}^{1} (1-t^2)^{\frac{n}{2}} dt\\
        &=\frac{2r^{n+1}\pi^{\frac{n}{2}}}{\Gamma(\frac{n}{2}+1)} \int_{0}^{1} (1-t^2)^{\frac{n}{2}} dt\\
        &\text {Podstawmy } t^2=y, dy=2t dt\\
        &=\frac{2r^{n+1}\pi^{\frac{n}{2}}}{\Gamma(\frac{n}{2}+1)} \int_{0}^{1} (1-y)^{\frac{n}{2}-1+1} y^{-\frac{1}{2}} dy\\ 
        &=\frac{2r^{n+1}\pi^{\frac{n}{2}}}{\Gamma(\frac{n}{2}+1)} B\left(\frac{n}{2}+1,\frac{1}{2}\right)\\
        &=\frac{2r^{n+1}\pi^{\frac{n}{2}}}{\Gamma(\frac{n}{2}+1)} \cdot \frac{\Gamma(\frac{n}{2}+1)\Gamma(\frac{1}{2})}{\Gamma(\frac{n}{2} + 1 + \frac{1}{2})}\\
        &=\frac{r^{n+1} \pi^{\frac{n}{2}} \sqrt{\pi}}{\Gamma(\frac{n+1}{2}+1)}\\
        &=\frac{r^{n+1} \pi^{\frac{n+1}{2}}}{\Gamma(\frac{n+1}{2}+1)}\qed
    \end{align}
\end{enumerate}

\subsection{Symplesks}

\[S_n=\{(x_1,x_2,\dots,x_n)\in\mathbb{R}^n, x_1+x_2+\dots+x_n\leq a, x_1,x_2,\dots,x_n\geq 0\}\]
\[\lambda{S_n}=\frac{a^n}{n!}\]
D-d. indukcyjny:
\begin{enumerate}
    \item $n=1$, $0\leq x_1\leq a$, $\lambda(S_1)=a$ 
    \item $\lambda(S_n)=\frac{a^n}{n!}$
    \begin{align}
        \lambda(S_{n+1})&=\\
        &= \int\int dx_1 dx_2\dots dx_{n+1}\\
        &= \int_{0}^{a} \frac{(a-x_{n+1})^n}{n!} dx_{n+1}\\
        &= \frac{(-1)(a-x_{n+1})^{n+1}}{(n+1)!}|_{0}^{a}\\
        &= \frac{a^{n+1}}{(n+1)!}
    \end{align}
\end{enumerate}

\begin{de}
    Splot funkcji. Dla odpowiednich funkcji $f,g:\mathbb{R}^n\rightarrow \mathbb{R}$
    definiujemy splot:
    \[(f*g)(x) = \int\dots\int_{\mathbb{R}^n} f(t)g(x-t) dt_1 dt_2 \dots dt_n\]
    Dla $x=(x_1,x_2,\dots,x_n), t=(t_1,t_2,\dots,t_n), x-t=(x_1-t_1,x_2-t_2,\dots,x_n-t_n)$
\end{de}

\subsection{Wstęp do równań różniczkowych}

Przykład równania różniczkowego:
\[m'(t)=(-1)k \cdot m(t)\]
Rozwiązanie - każda funkcja postaci:
\[m(t)=Ce^{-kt}\]
Zobaczmy:
\[m'(t)=C (-k) e^{-kt} = (-k) m(t)\]

\subsection{Krzywe całkowe równania różniczkowego}

Wykres :cry:

\subsection{Równania różniczkowe z warunkiem początkowym}

$$\begin{cases}
    m'(t)=(-k)m(t)\\
    m(t_0)=m_0
\end{cases}$$
Rozwiązanie:
$m(t)=m_0 e^{-k(t-t_0)}$

\section{Wykład X}

\subsection{Równanie różniczkowe}

\begin{de}
    (Równanie różniczkowe) Zwyczanje pierwszego rzędu:
    \[y' = f(t,y)\]
    Ogólne:
    \[F(t,y,y')=0\]
\end{de}

\begin{de}
    $y(t)$ jest rozwiązaniem rr. $y'=f(t,y)$ na przedziale $(a,b)$ jeśli:
    $y'(t)$ istnieje na przedziale $(a,b)$:
    \[y'(t)=f(t,y(t))\]
\end{de}

\begin{de}
    Wykres rozwiązania rr. to jego krzywa całkowa.
\end{de}

\begin{pk}
    Rozwiążmy RR:
    \begin{enumerate}
        \item $y(t) = \frac{1}{1+t}$ jest rozwiązaniem rr. $y' + 2ty^2 = 0$ na $\mathbb{R}$
        \begin{align}
            y'(t) &= ((1+t^2)^{-1})' = (-1)2t (1+t^2)^{-2} = \frac{-2t}{(1+t^2)^2}\\
            2t y^2(t) &= \frac{2t}{(1+t^2)^2}\\
            y'(t) + 2ty^2(t) &= 0
        \end{align}
        Jak widać rozwiązanie działa.
        \item $y(t)=\ln(t)$ jest rozwiązaniem rr. $y'=e^{-y}$ na $(0,\infty)$.
        \begin{align}
            y'(t) &= \frac{1}{t}\\
            e^{-\ln(t)} &= \frac{1}{e^{\ln(t)}} = \frac{1}{t}
        \end{align}
    \end{enumerate}
\end{pk}

\subsection{Równanie różniczkowe z warunkiem}

\begin{de}
    Równanie różniczkowe wraz z warunkiem $y(t_0)=y_0$ nazywamy zagadnieniem początkowym.
    Mówimy, że $y(t)$ jest rozwiązaniem zagadnienia początkowego, jeżeli jest rozwiązaniem równania różniczkowego 
    $y'(t)=f(t,y)$ na pewnym przedziale zawierającym punkt $t_0$ i spełnia warunek $y(t_0)=y_0$:
    $$
    \begin{cases}
        y'(t) = f(t,y(t))\\
        y(t_0) = y_0
    \end{cases}
    $$
\end{de}

\begin{de}
    RR, które można zapisać w postaci: 
    \[y'(t)=g(t)h(y)\]
    nazywamy rr. o zmiennych rozdzielonych.\\
    Zakładamy, że funkcje $g(t), h(y)$ są ciągłe oraz $h(y)\neq 0$ dla każdego $y$. Wówczas całka rr. o rozdzielonych zmiennych dana jest wzorem:
    \[\int \frac{dy}{h(y)} = \int g(t) dt + C\]
    ($y'(t)=\frac{dy}{dt}$)\\
    D-d.\\
    \begin{align}
        \frac{dy}{dt} &= g(t)h(y)\\
        \frac{1}{h(y)} dy &= g(t)dt\\
        \int \frac{dy}{h(y)} &= \int g(t) dt + C\qed
    \end{align}
    Jeżeli $h(y)=0$ dla pewnego $y_0$ to funkcja stała $y(t)=y_0$ jest jednym z rozwiązań rr. $y'(t)=g(t)h(y)$.
    $y'=0, h(y(t))=h(y_0)=0\qed$
\end{de}

\begin{pk}
    \[(1+e^y)yy'=e^t\]
    \begin{align}
        y' = e^t \frac{1}{(1+e^y)y}\\
        (1+e^y)y\frac{dy}{dt} = e^t\\
        (1+e^2)y dy = e^t dt\\
        \int (1+e^y) y dy = \int e^t dt + C\\
        \int (y + ye^y) dy = \int e^t dt + C\\
        \frac{y^2}{2} + \int ye^y dt = e^t + C\\
        \frac{y^2}{2} + e^y(y-1) = e^t + C
    \end{align}
    Metodami elementarnymi rozwikłanie tej funkcji jest niemożliwe. Wyjdzie prawdopodobnie specjalna funkcja d'Alemberta.
    Wobec tego, w tym przypadku nie trzeba tego rozwikływać do funkcji $y(t)$
\end{pk}

\begin{tw}
    Zakładamy, że funkcje $g(t), h(y)$ są ciągłe odpowiednio na przedziałach $(a,b), (c,d)$ oraz $h(y)\neq 0$.
    dla każdego $y\in(c,d)$. Niech $t_0\in(a,b), y_0\in(c,d)$. Wówczas zagadnienie początkowe:
    $$
    \begin{cases}
        y' = g(t)h(y)\\
        y(t_0) = y_0
    \end{cases}
    $$
    Ma dokładnie jedno rozwiązanie.
\end{tw}

\subsection{Równiaine różniczkowe jednorodne}

\begin{de}
    RR, które można zapisać w postaci:
    \[y' = f\left(\frac{y}{t}\right)\]
    nazywamy rr. jednorodnym.
\end{de}

\begin{pk}
    $y'=f\left(\frac{y}{t}\right)$\\
    Przykład:\\
    \begin{align}
        ty y' = y^2 - t^2\\
        y' = \frac{x}{y} - \frac{1}{\frac{y}{t}}\\
        f(u)=u-\frac{1}{u}
    \end{align}
\end{pk}

\subsection{Równanie różniczkowe o rozdzielonych zmiennych}

\begin{fakt}
    Jeżeli $f(u)=u$, to rr. jednorodne przyjmuje następującą postać:
    \[y' = \frac{y}{t}\]
    jest to rr. o rozdzielonych zmiennych, którego rozwiązanie dane jest wzorem:
    \[y(t)=Ct\]
    D-d.\\
    \begin{align}
        y' &= \frac{y}{t}\\
        \frac{dy}{dt} &= \frac{y}{t}\\
        \frac{dy}{y} &= \frac{dt}{t}\\
        \int \frac{dy}{y} &= \int \frac{dt}{t} + C\\
        \ln(|y|) &= \ln(|t|) + C\\
        |y| &= e^{\ln|t| + C}\\
        |y| &= |t|e^{C}, e^{C} \in \mathbb{R^{+}}\\
        y &= e^{C}t \lor y=(-e^{C}) t\\
        y(t) &= C_1 t, C_1 \in \mathbb{R}-\{0\}
    \end{align}
\end{fakt}

\subsection{Równanie różniczkowe przez zamianę zmiennych}

\begin{tw}
    RR. jednorodne przez zamianę zmiennych $y=ut$:
    \[y(t)=u(t)t\]
    Sprowadzamy do równania różniczkowego o zmiennych rozdzielonych:
    \begin{align}
        y'&=f\left(\frac{y}{t}\right)\\
        y&=ut\\
        y'&=u't+u(t)' =u't+u\\
        u't+u &= f(u)\\
        u't &= f(u)-u\\
        \frac{du}{dt}t &= f(u)-u\\
        \frac{du}{f(u)-u} &= \frac{dt}{t}\\
        \int \frac{du}{f(u)-u} &= \int \frac{dt}{t}
    \end{align}
\end{tw}

\begin{pk}
    \begin{align}
        ty' &= t + y\\
        y'&=1+\frac{y}{t}\\
        f(\frac{y}{t}) &= 1 + \frac{y}{t}\\
        \text {podstawmy } y=ut&, y'=u't + u \cdot 1\\
        ty' &= u't + u\\
        t+y &= t+ut\\
        u't^2 + ut &= t + ut\\
        u't^2 &= t\\
        \frac{du}{dt} &= \frac{1}{t}\\
        du &= \frac{1}{t} dt\\
        \int du &= \int \frac{dt}{t}\\
        u(t) &= \ln|t| + C
    \end{align}
    Kolejno: $x(t)=t\left(\ln|t|+C\right)$
\end{pk}

\begin{tw}
    Zakładamy, że funkcja $g(u)$ jest ciągła na $(a,b)$ 
    oraz dla każdego $u\in(a,b),$ $g(u)\neq u$.
    Wówczas zagadnienie początkowe:
    $$
    \begin{cases}
        y' = g\left(\frac{y}{t}\right)
        y(t_0)=y_0
    \end{cases}
    $$
    $a < \frac{y_0}{t_0} < b$ ma dokładnie jedno rozwiązanie.
\end{tw}

\subsection{Równanie róniczkowe liniowe}

\begin{de}
    RR, które można zapisać w postaci:
    \[y' + p(t)y = q(t)\]
    Nazywamy równaniem liniowym pierwszego rzędu.
    Jeżeli $q(t)\neq 0$ to równanie różniczkowe jest równaniem różniczkowym liniowym niejednorodnym.
    Jeżeli $q(t)=0$: $y'+p(t)=0$ jest równaniem różniczkowym liniowym jednorodnym.
\end{de}

\begin{pk}
    \begin{align}
        y' + p(t)y(t) = 0\\
        \frac{dy}{dt} = -p(t)y(t)\\
        \frac{dy}{y} = -p(t)dt\\
        \int \frac{dy}{y} = \int -p(t) dt\\
        \ln|y| = \int -p(t) dt + C\\
        |y| = e^{\int -p(t) dt} e^{c}\\
        y=\pm e^C e^{\int -p(t) dt}\\
        y(t) = c_1 e^{\int -p(t) dt} 
    \end{align}
\end{pk}

\begin{pk}
    $y' + ty = 0 \implies y(t) = c_1 e^{\int -t dt} = c_1 e^{-\frac{t^2}{2}}$
\end{pk}

\section{Wykład XI}

\begin{de}
    (Całka krzywoliniowa nieskierowana) Niech \( f:\mathbb{R}^n \rightarrow \mathbb{R} \), \( \gamma [a,b]\rightarrow\mathbb{R}^n\), $\gamma$ ma ciągłą pochodną.
    Wtedy:
    \[ \int_{\gamma} f dt = \int_{a}^{b} f(\gamma(t)) \cdot ||\gamma'(t)|| dt \]
\end{de}

\begin{pk}
    Rozważmy przykłady krzywych gamma:
    \begin{enumerate}
        \item \( \gamma (t) = (t,t^2), t\in[0,1] \)
        \item Okrąg o środku w punkcie $(x_0,y_0)$ i promieniu $r$. \( \gamma(t) = (x_0 + r\cos(t), y_0 + r\sin(t)), t \in [0,2\pi]\)
        \item Odcinek łączący punkty $P(x_1,y_1,z_1), Q(x_2,y_2,z_2)$ \( \gamma(t) = (1-t)\cdot P + t\cdot Q\) ($tQ = (tx_2, ty_2, tz_2)$)
    \end{enumerate}
\end{pk}

\begin{pk}
    Policzmy następujące Całki:
    \begin{enumerate}
        \item \( f=1, \gamma(t)=(r\cos(t),r\sin(t)), t\in[0,2\pi], \gamma'(t) = ((r\cos(t))',(r\sin(t))') = (-\sin(t)r)\)
        \[ \int_{\gamma} 1 dl = \int_{0}^{2\pi} \sqrt{(-\sin(t)r)^2 + (\cos(t)r)^2} dt \]
        \item \(\gamma: [0,1]\rightarrow \mathbb{R}^2, \gamma(t) = (t,t^2), f(x,y)=3x + \sqrt{y}, \gamma'(t) = (1,2t), |\gamma'(t)|=\sqrt{1+4t^2}\)
        \[ \int_{\gamma} f dl = \int_{0}^{1} (3t + \sqrt{t})\sqrt{1 + 4t^2} dt \]
        \item \(\gamma [0,4] \rightarrow \mathbb{R}^2 \) 
        $$\gamma(t)=
        \begin{cases}
            \gamma_1(t) = (t,0) \text{ dla } 0\leq t\leq 1\\
            \gamma_2(t) = (1,t-1) \text{ dla } 1<t<2\\
            \gamma_3(t) = (3-t,1) \text{ dla } 2\leq t <3\\
            \gamma_4(t) = (0,4-t) \text{ dla } 3\leq t \leq 4
        \end{cases}
        $$
        $\gamma = \gamma_1 \cup \gamma_2 \cup \gamma_3 \cup \gamma_4$
        \[ \int_{\gamma} dl = \int_{\gamma_1} x^2 d + \int_{\gamma_2} x^2 dl + \int_{\gamma_3} x^2 dl + \int_{\gamma_4} x^2 dl \]
        \( \gamma_1'(t) = (1,0), |\gamma_1'(t)| = \sqrt{1^2 + 0^2} = 1, \gamma_2'(t)=(0,1), |\gamma_2'(t)|=1, |\gamma_3'(t)=(-1,0), |\gamma_3'(t)|  = 1 ... \)
    \end{enumerate}
\end{pk}

\begin{de}
    (Pole wektorowe) \( F: \mathbb{R}^n \rightarrow \mathbb{R}^n \)
    \[ F(x,y) = (x,y) \]
    \[ G(x,y) = (-y,x) \]
    \[ F = (F_1, F_2, \dots, F_n) \]
    \[ F_i: \mathbb{R}^2 \rightarrow \mathbb{R} \]
\end{de}

\begin{de}
    (Całka krzywoliniowa skierowana)
    \[ \int_{\gamma} \vec{F} \circ \vec{dl} = \int_{\gamma} (F_1 dx_1 + F_2 dx_2 + \dots + F_n dx_n) = 
    \int_{a}^{b} \left(\vec{F} (\gamma(t)) \circ \gamma'(t)\right) dt
    \] gdzie $\circ$ to iloczyn skalarny.
\end{de}

\begin{pk}
    Przykład:
    \(\vec{F}(x,y) = (x,y), \gamma(t) = (\cos(t), \sin(t)), t\in[0,2\pi], \gamma'(t) = (-\sin(t),\cos(t))\)
    \[\int_{\gamma} \vec{F} \circ \vec{dl} = \int_{0}^{2\pi} (\cos(t), \sin(t)) \circ (-\sin(t), \cos(t)) =\]
    \[= \int_{0}^{2\pi} (-\cos(t)(\sin(t))) + \sin(t)\cos(t) dt = \int_{0}^{2\pi} 0 dt = 0\]
    Dla $\vec{G}$
    \[\int_{\gamma} \vec{G} \circ \vec{dl} = \int_{0}^{2\pi} (-\sin(t), \cos(t)) \circ (-\sin(t), \cos(t)) dt\]
    \[= \int_{0}^{2\pi} (\sin^2(t) + \cos^2(t)) dt = \int_{0}^{2\pi} 1 dt = 2\pi\]
\end{pk}

Obszar jednospójny nie ma dziury w środku - Prowadzący.

\begin{tw}
    (Twierdzenie Greena) Jeśli $B$ jest obszarem jednospójnym, ograniczonym krzywą gładką $\vec{F}=(F_1,F_2)$ - pole wektorowe, które ma ciągłą pochodną na pewnym otoczeniu $B$.
    \( \partial B \) - brzeg zorientowany dodatnio:
    \[ \int_{\partial B} \vec{F} \circ \vec{dl} = \int\int_{B} \left(-\frac{\partial F_1}{\partial y} + \frac{\partial F_2}{\partial x}\right) dx dy\]
\end{tw}

\begin{pk}
    Zastosujmy Tw. Greena. \( F(x,y) = \left((1-x^2)y, x(1+y^2)\right), (F_1(x,y),F_2(x,y))\)
    , $B$ - koło $x^2+y^2\leq R^2$ zorientowane dodatnio:
    \[ \int_{\partial B} \vec{F} \circ \vec{dl} = \int\int_{x^2+y^2\leq R^2} \left(- \frac{\partial F_1}{\partial y} + \frac{\partial F_2}{\partial x}\right) dx dy = \]
    \[ \int\int_{x^2+y^2\leq R^2} (-(1-x^2)+(1+y^2)) dxdy= \int\int_{x^2+y^2\leq R^2} (x^2+y^2) dxdy =\]
    \[ = \int_{0}^{2\pi} \int_{0}^{R} r^3 dr d\alpha = \int_{0}^{2\pi} \frac{R^4}{4} d\alpha = \frac{\pi}{2} R^4\]
\end{pk}


\begin{de}
    (Całka powierzchniowa skierowana) $\sigma$ - mały element powierzchni
    \[\int_{\partial B} \vec{F}\circ\vec{d\sigma} =_{def} \text{całka podwójna}\]  
\end{de}

\begin{tw}
    (Twierdzenie Gaussa) (zamiana całki powierzchniowej skierowanej na całkę potrójną)
    \[ \int_{\partial B} \vec{F} \circ \vec{d\sigma} = \int\int\int_{B} (\nabla \circ \vec{F}) dx dy dz \]
\end{tw}

\begin{pk}
    $B: x^2 + y^2 + z^2 \leq R^2, OB: x^2+y^2+z^2 = R^2$
    \[ \int_{\partial B} \vec{F} \circ \vec{d\sigma} = \int\int\int_{x^2+y^2+z^2 \leq R^2} \frac{\partial}{\partial x} (x) \frac{\partial}{\partial y} (y) \frac{\partial}{\partial z} (x) dxdydz = \int\int\int_{x^2+y^2+z^2 \leq R^2} 3 dx dy dz =\]
    \[ = 3\cdot \frac{4}{3} \pi R^3 = 4\pi R^3\]
\end{pk}


\end{document}
